\documentclass[12pt, titlepage, oneside]{amsbook}
\makeatletter \@addtoreset{equation}{chapter}
\@addtoreset{figure}{chapter} \@addtoreset{theorem}{chapter}
\makeatother \sloppy \pagestyle{plain} \setcounter{tocdepth}{1}

\renewcommand{\theequation}{\thesection.\arabic{equation}}

\usepackage[russian]{babel}

\usepackage{pstcol}
\usepackage{pstricks, pst-node}
\usepackage[matrix, arrow,curve]{xy}
\usepackage{xypic,amscd}
\usepackage{amsmath}
\usepackage{amssymb}
\usepackage{latexsym}
\usepackage{graphicx}
\usepackage{wasysym}
\binoppenalty=10000 \relpenalty=10000
\parskip = 3pt
\parindent = 0.4cm

%\newcommand{\char}{\operatorname{char}}
\newcommand{\Div}{\operatorname{div}}
\newcommand{\ddef}{\operatorname{def}}
\newcommand{\rot}{\operatorname{rot}}
\newcommand{\grad}{\operatorname{grad}}
\newcommand{\Pic}{\operatorname{Pic}}
\newcommand{\Aut}{\operatorname{Aut}}
\newcommand{\PSL}{\operatorname{PSL}}
\newcommand{\GL}{\operatorname{GL}}
\newcommand{\Cr}{\operatorname{Cr}}
\newcommand{\St}{\operatorname{St}}
\newcommand{\Orb}{\operatorname{Orb}}
\newcommand{\Tr}{\operatorname{Tr}}
\newcommand{\Irr}{\operatorname{Irr}}
\newcommand{\Bs}{\operatorname{Bs}}
\newcommand{\mld}{\operatorname{mld}}
\newcommand{\cov}{\operatorname{cov}}
\newcommand{\cor}{\operatorname{cor}}
\newcommand{\sign}{\operatorname{sign}}
\newcommand{\mult}{\operatorname{mult}}
\newcommand{\Sing}{\operatorname{Sing}}
\newcommand{\Diff}{\operatorname{Diff}}
\newcommand{\Supp}{\operatorname{Supp}}
\newcommand{\Ext}{\operatorname{Ext}}
\newcommand{\Exc}{\operatorname{Exc}}
\newcommand{\Cl}{\operatorname{Cl}}
\newcommand{\discrep}{\operatorname{discrep}}
\newcommand{\discr}{\operatorname{discr}}
\newcommand{\di}{\operatorname{di}}
\renewcommand\gcd{\operatorname{\text{{\rm НОД}}}\ }
%\newcommand{\GL}{\operatorname{GL}}


\newcommand{\muu}{{\boldsymbol{\mu}}}
\newcommand{\OOO}{{\mathcal O}}
\newcommand{\LL}{{\mathcal L}}
\newcommand{\EE}{{\mathcal E}}
\newcommand{\PP}{\mathbf{P}}
\newcommand{\MM}{\mathbf{M}}
\newcommand{\NN}{\mathbb{N}}
\newcommand{\ZZ}{\mathbb{Z}}
\newcommand{\FF}{\mathbb{F}}
\newcommand{\RR}{\mathbb{R}}
\newcommand{\CC}{\mathbb{C}}
\newcommand{\QQ}{\mathbb{Q}}
\newcommand{\AAA}{\mathfrak{A}}
\newcommand{\DDD}{\mathfrak{D}}
\newcommand{\SSS}{\mathfrak{S}}
\newcommand{\MMM}{\mathfrak{M}}
\newcommand{\RRR}{\mathfrak{R}}
\newcommand{\BBB}{\mathfrak{B}}
\newcommand{\aaa}{\mathfrak{a}}
\newcommand{\ppp}{\mathfrak{p}}
\newcommand{\mmm}{\mathfrak{m}}
\newcommand{\DD}{\mathbb{D}}
\newcommand{\DDDD}{\mathbf{D}}

\newtheorem{theorem}{Теорема}[chapter]
\newtheorem{proposition}[theorem]{Предложение}
\newtheorem{lemma}[theorem]{Лемма}
\newtheorem{corollary}[theorem]{Следствие}
\newtheorem{claim}[theorem]{Утверждение}

\theoremstyle{definition}
\newtheorem{example}[theorem]{Пример}
\newtheorem{definition}[theorem]{Определение}
\newtheorem{notation}[theorem]{Обозначения}
\newtheorem{construction}[theorem]{Конструкция}
\newtheorem{pusto}[theorem]{}
\newtheorem{remark}[theorem]{Замечание}
\newtheorem{Exercise}[theorem]{Упражнение}
\newtheorem{zero}[theorem]{}
\newtheorem{case}[theorem]{}

\theoremstyle{remark}


\date{}

\begin{document}

\begin{titlepage}
\begin{center}
\large{\textbf{Лекции по курсу "Теория Галуа"}} \quad \\
\quad
\\ \quad
\\ \quad
\large{\textbf{Белоусов Григорий Николаевич}} \quad \\ \quad

\end{center}
\end{titlepage}

\tableofcontents

Мы будем придерживаться следующих обозначений.

\begin{notation}
\label{not}  Мы будем придерживаться следующих обозначений:
\begin{itemize}
\item $\NN$ --- множество натуральных чисел.
\item $\ZZ$ --- множество целых чисел.
\item $\QQ$ --- множество рациональных чисел.
\item $\RR$ --- множество вещественных чисел.
\item $\forall$ --- для любого.
\item $\exists$ --- существует.
\item $\in$ --- принадлежит.
\item $\infty$ --- бесконечность.
\end{itemize}
\end{notation}

\chapter{Группы}



\section{Коммутант группы}

Пусть $G$ --- группа. \emph{Коммутатором} двух элементов $a,b\in G$ называется произведение $$[a,b]=aba^{-1}b^{-1}.$$

Непосредственно из определения следуют следующие свойства
\begin{enumerate}
\item $ab=[a,b]ba$;
\item $ab=ba[a^{-1},b^{-1}]$;
\item $[a,b]^{-1}=[b,a]$;
\item $[a,b]=e$ тогда и только тогда, когда $ab=ba$.
\end{enumerate}

\begin{definition}
\emph{Коммутантом} группы $G$ называется подгруппа $K$, порожденная всеми коммутантами. Обозначается $[G,G]=K$.
\end{definition}

\begin{theorem}
\label{Kom1}
Пусть $H$ --- нормальная подгруппа в группе $G$. Тогда $[H,H]\triangleleft G$.
\end{theorem}

\begin{proof}
Заметим, что $$g[a,b]g^{-1}=gaba^{-1}b^{-1}g^{-1}=(gag^{-1})(gbg^{-1})(ga^{-1}g^{-1})(gb^{-1}g^{-1})=$$ $$(gag^{-1})(gbg^{-1})(gag^{-1})^{-1}(gbg^{-1})^{-1}=[gag^{-1},gbg^{-1}].$$
Пусть $h\in [H,H]$ и $g\in G$. Тогда $$h=[h_1,h'_1][h_2,h'_2]\cdots[h_n,h'_n],$$ где все $h_i,h'_i\in H$. Следовательно, $$ghg^{-1}=g[h_1,h'_1][h_2,h'_2]\cdots[h_n,h'_n]g^{-1}=$$ $$=(g[h_1,h'_1]g^{-1})(g[h_2,h'_2]g^{-1})\cdots(g[h_n,h'_n]g^{-1})=$$ $$=[gh_1g^{-1},gh'_1g^{-1}][gh_2g^{-1},gh'_2g^{-1}]\cdots[gh_ng^{-1},gh'_ng^{-1}].$$ Поскольку $H$ --- нормальная подгруппа в группе $G$, то все $gh_ig^{-1},gh'_ig^{-1}\in H$. Отсюда, $g^{-1}hg\in [H,H]$
\end{proof}

\begin{corollary}
\label{Kom1a}
Коммутант $K$ группы $G$ является нормальной подгруппой.
\end{corollary}

\begin{theorem}
\label{Kom2}
Пусть $K$ --- коммутант группы $G$. Тогда группа $G/K$ абелева.
\end{theorem}

\begin{proof}
Действительно, $$(aK)(bK)=abK=ba[a^{-1},b^{-1}]K=baK=(bK)(aK).$$
\end{proof}

\begin{theorem}
\label{Kom3}
Пусть $H$ --- нормальная подгруппа группы $G$, и пусть $G/H$ абелева. Тогда $K\subset H$, где $K$ --- коммутант группы $G$.
\end{theorem}

\begin{proof}
Если $G/H$ абелева, то $abH=baH$, или, поскольку $H$ --- нормальная подгруппа, $Hab=Hba$. Следовательно, существует $h\in H$ такой, что $ab=hba$, т.е. $h=aba^{-1}b^{-1}=[a,b]$.
\end{proof}

\section{Разрешимые и нильпотентные группы}

\begin{definition}
Последовательность подгрупп $$G=G_0\supset G_1\supset G_2\supset\cdots\supset G_n=\{e\}$$ называется \emph{рядом подгрупп}. Если для любого $i$ $G_{i}\triangleleft G$, то ряд называется \emph{нормальным}.
Группа $G$ называется \emph{разрешимой}, если существует нормальный ряд подгрупп $$G=G_0\supset G_1\supset G_2\supset\cdots\supset G_n=\{e\}$$ такой, что для любого $i$, $G_i/G_{i+1}$ --- абелева группа.
\end{definition}

Пусть $K_0=G$, $K_1=[K_0,K_0]$, $K_2=[K_1,K_1]$ и т.д. Тогда имеет место следующая теорема.

\begin{theorem}
\label{Raz1}
Группа $G$ разрешима тогда и только тогда, когда существует $n$ такое, что $K_n=\{e\}$.
\end{theorem}

\begin{proof}
Предположим, что $K_n=\{e\}$. Из теорем \ref{Kom1} \ref{Kom2} следует, что $G=K_0\supset K_1\supset K_2\supset\cdots\supset K_n=\{e\}$ --- нормальный ряд подгрупп, и $K_i/K_{i+1}$ --- абелева группа. Следовательно, в одну сторну утверждение доказано. Предположим, что существует нормальный ряд подгрупп $$G=G_0\supset G_1\supset G_2\supset\cdots\supset G_m=\{e\},$$ и для любого $i$, $G_i/G_{i+1}$ --- абелева группа. Из теоремы \ref{Kom3} следует, что $K_1\subset G_1$. Докажем по индукции, что $K_i\subset G_i$. Предположим, что $K_{i-1}\subset G_{i-1}$. Тогда $[K_{i-1},K_{i-1}]\subset[G_{i-1},G_{i-1}]$. С другой стороны, по теореме \ref{Kom3} $[G_{i-1},G_{i-1}]\subset G_i$. Отсюда, $K_i\subset G_i$.
\end{proof}

\begin{example}
Рассмотрим группу $S_n$ ($n\geq 5$). Заметим, что $[\sigma_1,\sigma_2]=\sigma_1\sigma_2\sigma^{-1}_1\sigma^{-1}_2$ --- четная перестановка. Следовательно, $[S_n,S_n]\subset A_n$. С другой стороны, $$[(ij),(jk)]=(ij)(jk)(ij)(jk)=(ikj).$$ Следовательно, любые циклы длины три лежат в $[S_n,S_n]$. Тогда $[S_n,S_n]=A_n$. Теперь посчитаем коммутант $A_n$. Рассмотрим перестановки $(ij)(kl)$ и $(ij)(km)$. Получаем $$[(ij)(kl),(ij)(km)]=(ij)(kl)(ij)(km)(ij)(kl)(ij)(km)=(klm).$$ Следовательно, $[A_n,A_n]=A_n$. Таким образом, группы $S_n$ и $A_n$ не разрешимы при $n\geq 5$.
\end{example}

\begin{example}
Рассмотрим группу $S_4$. Аналогично, $[S_4,S_4]=A_4$. Теперь посчитаем коммутант $A_4$. Для этого рассмотрим подгруппу $$V_4=\{e,(12)(34),(13)(24),(14)(23)\}.$$ Заметим, что $V_4$ --- нормальная подгруппа в $A_4$, и $A_4/V_4\simeq\ZZ_3$, т.е. $A_4/V_4$ --- абелева группа. Согласно теореме \ref{Kom3} $[A_4,A_4]\subset V_4$. С другой стороны, $$[(ijk),(ijl)]=(ijk)(ijl)(ikj)(ilj)=(ij)(kl).$$ Таким образом, $[A_4,A_4]=V_4$. Поскольку $V_4$ --- абелева группа, то $S_4$ и $A_4$ разрешимы.
\end{example}

\begin{definition}
Пусть $G$ --- группа. Тогда множество $Z(G):=\{a\in G\mid ag=ga
\text{ для всех } g\in G\}$ называется \emph{центром} группы $G$.
\end{definition}

Очевидно, что центр группы является нормальной подгруппой.

\begin{definition}
\emph{Центральным рядом подгрупп} называется нормальный ряд подгрупп $$G=G_0\supset G_1\supset G_2\supset\cdots\supset G_n=\{e\}$$ такой, что $G_i/G_{i+1}\subset Z(G/G_{i+1})$. Группа, обладающая центральным рядом подгрупп, называется \emph{нильпотентной группой}.
\end{definition}

Поскольку центр --- абелева группа, то нильпотентная группа является разрешимой. Обратное неверно (см. группу $S_4$).

\begin{definition}
Группа $G$ называется \emph{конечной $p$-группой}, если $p$ --- простое и $|G|=p^n$.
\end{definition}

\begin{lemma}
\label{Nil1}
Пусть $G$ --- конечная $p$-группа. Тогда $Z(G)\neq\{e\}$.
\end{lemma}

\begin{proof}
Пусть группа $G$ действует на себе посредством сопряжения. Тогда
$|G|=|Z(G)|+\sum |\Orb(g_i)|$, где все $|\Orb(g_i)|>1$. Заметим, что $|\Orb(g_i)|$ является делителем порядка группы
$G$. Следовательно, $|\Orb(g_i)|$ делится на $p$. Отсюда, $|Z(G)|$
делится на $p$. Следовательно, $Z(G)\neq\{e\}$.
\end{proof}

\begin{theorem}
\label{Nil2}
Любая конечная $p$-группа нильпотентна.
\end{theorem}

\begin{proof}
Пусть $G$ --- конечная $p$-группа. Согласно лемме \ref{Nil1}, $Z(G)\neq\{e\}$. Пусть $G_1=Z(G)$. Рассмотрим $G'_1=G/G_1$. Тогда $G'_1$ --- также конечная $p$-группа. Следовательно, $Z(G'_1)\neq\{e\}$. Пусть $G_2$ --- прообраз $Z(G'_1)$ при естественном гомоморфизме $f\colon G\rightarrow G/G_1$. Заметим, что $G_2$ --- нормальная подгруппа в $G$. Действительно, пусть $h\in G_2$, $g\in G$. Тогда $$f(ghg^{-1})=f(g)f(h)(f(g))^{-1}=f(h)f(g)(f(g))^{-1}=f(h).$$ Следовательно, $ghg^{-1}\in G_2$. Рассмотрим $G'_2=G/G_2$. Тогда $G'_2$ --- также конечная $p$-группа. Следовательно, $Z(G'_2)\neq\{e\}$. Пусть $G_3$ --- прообраз $Z(G'_2)$ при естественном гомоморфизме $f\colon G\rightarrow G/G_2$. Аналогично, $G_3$ --- нормальная подгруппа в $G$. Продолжая эти действия, мы получим искомый центральный ряд.
\end{proof}

\section{Теоремы Силова}

\begin{theorem}[1-я теорема Силова]
\label{1Sil} Пусть $G$ --- произвольная конечная группа и $|G|=mp^k$, где
$(m,p)=1$. Тогда существует подгруппа $H\subset G$ такая, что
$|H|=p^k$ (такие подгруппы называются \emph{силовскими подгруппами}).
\end{theorem}

\begin{proof}
Докажем утверждение индукцией по порядку группы. Предположим,
утверждение верно для всех порядков меньших $n=mp^k$. Рассмотрим два
случая:

1) $p\mid |Z(G)|$. Тогда существует
подгруппа $H\subset Z(G)$ такая, что $|H|=p^s$. Заметим, что
$H\triangleleft G$. Пусть $G'=G/H$. Тогда, по предположению
индукции, в $G'$ существует подгруппа $A$ порядка $p^{k-s}$. Пусть
$B$ --- прообраз $A$ в $G$. Тогда $B$ --- подгруппа $G$ и
$|A|=|B|/|H|$. Отсюда, $|B|=p^k$.


2) Порядок центра $|Z(G)|$ не делится на $p$. Рассмотрим разбиение
группы $G$ на классы сопряженности. Мы получим $$|G|= |Z(G)| + \sum
|\Orb(g_i)|.$$ Поскольку $|Z(G)|$ не делится на $p$, то существует
$\Orb(g_i)$ такая, что $|\Orb(g_i)|$ не делится на $p$ тогда
$|\St(g_i)|=lp^k$, где $l<m$. Следовательно, по предположению
индукции, существует подгруппа $H\subset\St(g)\subset G$ такая, что
$|H|=p^k$.
\end{proof}

\begin{theorem}[2-я теорема Силова]
\label{2Sil} Пусть $G$ --- конечная группа и $|G|=mp^k$, где
$(m,p)=1$. Тогда любая $p$-подгруппа $H$ содержится в силовской
подгруппе $S$ и все силовские подгруппы сопряжены.
\end{theorem}

\begin{proof}
Согласно теореме \ref{1Sil} существует силовская $p$-подгруппа
$S_1$. Рассмотрим полигон $\{S_1,g_1 S_1,\dots\}$ левых смежных
классов. Пусть $H$ действует на нем умножением (т.е. $(h,gS_1)\mapsto
hgS_1$). Все орбиты состоят либо из одного элемента, либо число
элементов делится на $p$. Поскольку число смежных классов по
подгруппе $S_1$ равно $m$ и $(m,p)=1$, то существует одноэлементная
орбита $g_i S_1$. Таким образом $Hg_iS_1=g_iS_1$. Отсюда, $H\subset
g_iS_1g_i^{-1}$. Заметим, что $g_iS_1g_i^{-1}$ --- силовская
подгруппа. Таким образом, первое утверждение теоремы доказано.
Возьмем в качестве $H$ силовскую подгруппу, мы получим второе
утверждение теоремы.
\end{proof}

\begin{remark}
Из 2-й теоремы Силова следует, что силовская $p$-подгруппа нормальна тогда и только тогда, когда она единственна.
\end{remark}

\begin{theorem}[3-я теорема Силова]
\label{3Sil} Пусть $G$ --- произвольная группа и $|G|=mp^k$, где
$(m,p)=1$. Пусть $l$ --- число силовских подгрупп порядка $p^k$.
Тогда $l$ является делителем $m$ и $l=1+qp$.
\end{theorem}

\begin{proof}
Пусть $\{S_1,\dots,S_l\}$ --- множество силовских подгрупп. Пусть
группа $G$ действует на $\{S_1,\dots,S_l\}$ сопряжением. Тогда
$\{S_1,\dots,S_l\}$ образуют одну орбиту. Отсюда, $l$ является
делителем порядка группы $G$. Теперь, пусть $S_1$ действует на
$\{S_1,\dots,S_l\}$ сопряжением. Заметим, что у этого действия только один неподвижный элемент, сама $S_1$. Действительно, предположим, что $S_2$ --- тоже неподвижный элемент. Тогда $S_1$ лежит в нормализаторе $S_2$. Рассмотрим $S_1S_2$. Заметим, что $S_2$ --- нормальная подгруппа в $S_1S_2$. Следовательно, $$S_1S_2/S_2\simeq S_1/(S_1\cap S_2).$$ Отсюда, $S_1S_2$ --- конечная $p$-группа. Противоречие. Следовательно, множество $\{S_1,\dots,S_l\}$
разбивается на одну орбиту из одного элемента и некоторого числа
орбит, порядок которых делится на $p$. Отсюда, $l=1+qp$.
\end{proof}

\section{Простые группы}

\begin{definition}
Группа $G$ называется \emph{простой}, если в ней нет нормальных подгрупп, кроме тривиальной (состоящей из единицы) и самой группы.
\end{definition}

\begin{theorem}
\label{Pr1} Пусть $G$ --- конечная группа и $|G|=pq$, где
$p$ и $q$ --- простые числа. Тогда группа $G$ не простая.
\end{theorem}

\begin{proof}
Пусть $q>p$. Рассмотрим силовские $q$-подгруппы. Согласно 3-й теореме Силова их число делит $pq$ и дает, при делении на $q$, в остатке $1$. Следовательно, такая подгруппа одна. По второй теореме Силова она нормальна.
\end{proof}

\begin{theorem}
\label{Pr2} Пусть $G$ --- конечная группа и $|G|=p^2q$, где
$p$ и $q$ --- простые числа. Тогда группа $G$ не простая.
\end{theorem}

\begin{proof}
Предположим  $p>q$. Рассмотрим силовские $p$-подгруппы. Согласно 3-й теореме Силова их число делит $p^2q$ и дает, при делении на $p$, в остатке $1$. Следовательно, такая подгруппа одна. По второй теореме Силова она нормальна.

Предположим  $q>p$. Рассмотрим силовские $q$-подгруппы. Согласно 3-й теореме Силова их число делит $p^2q$ и дает, при делении на $q$, в остатке $1$. Следовательно, такая подгруппа либо одна, либо таких подгрупп $p^2$. Если силовская $q$-подгруппа одна, то все доказано. Следовательно, мы можем предполагать, что существуют $p^2$ таких подгрупп. Заметим, что любая силовская $q$-подгруппа изоморфна $\ZZ_q$. Тогда любые две из них имеют тривиальное пересечение (т.е. пересекаются по единицы), и все их элементы, кроме единицы, имеют порядок $q$. Посчитаем число элементов порядка $q$ в группе $G$, получаем $p^2(q-1)=p^2q-p^2$. С другой стороны силовская $p$-подгруппа состоит из $p^2$ элементов и не содержит элементы порядка $q$. Отсюда следует, что существует единственная силовская $p$-подгруппа. Тогда она нормальна.
\end{proof}

\begin{theorem}
\label{Pr3} Пусть $G$ --- конечная группа и $|G|=pqr$, где
$p,q,r$ --- простые числа. Тогда группа $G$ не простая.
\end{theorem}

\begin{proof}
Пусть $r>q>p$. Рассмотрим силовские $r$-подгруппы. Согласно 3-й теореме Силова их число делит $pqr$ и дает, при делении на $r$, в остатке $1$. Следовательно, такая подгруппа либо одна, либо таких подгрупп $pq$. Если силовская $r$-подгруппа одна, то все доказано. Следовательно, мы можем предполагать, что существуют $pq$ таких подгрупп. Заметим, что любая силовская $r$-подгруппа изоморфна $\ZZ_r$. Тогда любые две из них имеют тривиальное пересечение (т.е. пересекаются по единицы), и все их элементы, кроме единицы, имеют порядок $r$. Посчитаем число элементов порядка $r$ в группе $G$, получаем $pq(r-1)=pqr-pq$. Рассмотрим силовские $q$-подгруппы. Мы можем предполагать, что такая подгруппа не единственна. Согласно 3-й теореме Силова их, как минимум, $r$ штук. Заметим, что каждая силовская $q$-подгруппа изоморфна $\ZZ_q$. Тогда все они имеют тривиальное пересечения. Аналогично, посчитаем число элементов порядка $q$. Получаем, что их, как минимум, $r(q-1)$. Суммируя число элементов порядка $r$ и порядка $q$, получаем $$pqr-pq+r(q-1)>pqr.$$ Противоречие.
\end{proof}

\begin{theorem}
\label{Pr4} Группа $A_n$ проста при $n\geq 5$.
\end{theorem}

Сначала, докажем следующую лемму.

\begin{lemma}
\label{PrL}
Пусть нормальная подгруппа $H$ группы $A_n$ содержит цикл дины три. Тогда $H=A_n$.
\end{lemma}

\begin{proof}
Если $n=3$, то утверждение очевидно. Пусть $n>3$ и $H$ содержит перестановку $(ijk)$. Пусть $\sigma=(ij)(km)$, Тогда $$\sigma(ijk)\sigma^{-1}=(ij)(km)(ijk)(ij)(km)=(imj).$$ Более того, $(ijk)(imj)=(imk)$. Отсюда легко видеть, что все циклы длины три содержаться в $H$. Следовательно, $H=A_n$.
\end{proof}

Теперь докажем теорему \ref{Pr4}. Пусть $H$ --- нормальная подгруппа группы $A_n$. Пусть $\sigma\in H$ --- элемент подгруппы $H$, содержащий минимальное количество номеров, при разложении в произведение циклов. Предположим, что $\sigma$ содержит цикл, длины больше трех, т.е. $\sigma=(i_1i_2\cdots i_{n-3}i_{n-2}i_{n-1}i_{n})\cdots$. Пусть $\tau=(i_{n-2}i_{n-1}i_{n})$. Тогда $$\sigma_1=\tau\sigma\tau^{-1}=(i_{n-2}i_{n-1}i_{n})(i_1i_2\cdots i_{n-3}i_{n-2}i_{n-1}i_{n})(i_{n-2}i_{n}i_{n-1})\cdots=$$ $$=(i_1i_2\cdots i_{n-3}i_{n-1}i_{n}i_{n-2})\cdots.$$ С другой стороны, $$\sigma^{-1}\sigma_1=(i_1i_ni_{n-1}\cdots i_3i_2)(i_1i_2\cdots i_{n-3}i_{n-1}i_{n}i_{n-2})\cdots$$ оставляет неподвижным номер $i_{n-1}$. Таким образом, мы можем считать, что $\sigma$ состоит из циклов длины два и три. Предположим, что $\sigma$ содержит цикл длины три. Возводя в квадрат мы можем считать, что $\sigma$ состоит из циклов длины три. Предположим, что $\sigma$ содержит два таких цикла, т.е. $\sigma=(ijk)(lmp)\cdots$. Пусть $\tau=(ij)(kl)$. Тогда $$\sigma_1=\tau\sigma\tau^{-1}=(ij)(kl)(ijk)(lmp)(ij)(kl)\cdots=(ilj)(kmp)\cdots.$$ С другой стороны, $$\sigma_1\sigma^{-1}=(ilj)(kmp)(ikj)(lpm)\cdots$$ оставляет на месте номер $p$. Следовательно, $\sigma$ содержит лишь один цикл длины три. Тогда по лемме \ref{PrL} получаем, что $H=A_n$. Предположим, что $\sigma$ состоит из циклов длины два (транспозиций). Поскольку $\sigma$ четная, то $\sigma$ состоит из четного числа транспозиций, т.е. $\sigma=(ij)(kl)\cdots$. Поскольку $n\geq 5$, рассмотрим $\tau=(ij)(km)$. Тогда $$\sigma_1=\tau\sigma\tau^{-1}=(ij)(km)((ij)(kl)\cdots)(ij)(km)=(ij)(lm)\cdots.$$ Отсюда, $$\sigma \sigma_1=((ij)(kl)\cdots)((ij)(lm)\cdots)$$ оставляет на месте номера $i,j$. Противоречие.

\section{Транзитивные и примитивные группы}

\begin{definition}
Группа перестановок $G$ множества $M$ называется \emph{транзитивной} над $M$, если в $M$ существует элемент $a$ такой, что для любого $x\in M$ существует $g\in G$ такое, что $x=g a$. Также говорят, что $G$ действует \emph{транзитивно} на $M$.
\end{definition}

\begin{claim}
\label{Tr1}
Пусть группа $G$ действует транзитивно на множестве $M$. Тогда для любых двух элементов $x,y\in M$ существует элемент $g\in G$ такой, что $g x=y$.
\end{claim}

\begin{proof}
Пусть $x=g_1 a$, $y=g_2 a$. Тогда $$(g_2g^{-1}_1)x=g_2 a=y.$$
\end{proof}

Если группа $G$ действует не транзитивно на $M$, то множество $M$ можно разбить на непересекающиеся множества $M_\alpha$ так, что группа $G$ действует транзитивно на каждом множестве $M_\alpha$. Это разбиение осуществляется по следующему принципу. Два элемента $x,y\in M$ относятся в одно подмножество тогда и только тогда, когда существует $g\in G$ такой, что $y=g x$. Это отношение рефлексивно, симметрично и транзитивно. Действительно,
\begin{enumerate}
\item $x=g x$ при $g=e$ (рефлексивность);
\item если $y=g x$, то $x=g^{-1} y$ (симметричность);
\item если $y=g_1 x$ $z=g_2 y$, то $z=(g_2 g_1)x$ (транзитивность).
\end{enumerate}

\begin{definition}
Разбиение множества $M$ на непересекающиеся подмножества $M_\alpha$ называется \emph{разбиением на блоки относительно} $G$, если для любого $M_\alpha$ и любого $g\in G$ существует $M_\beta$ такое, что $M_\beta=g M_\alpha$. Очевидно, что всегда существует два тривиальных разбиения --- на одноэлементные блоки, и на единственный блок в виде всего множества $M$. Если нетривиального разбиения на блоки не существует, то группа $G$ называется \emph{примитивной}. В противном случае группа называется \emph{импримитивной}. Множества $M_\alpha$ называются \emph{областями импримитивности}.
\end{definition}

\begin{theorem}
\label{Tr2} Пусть $G$ действует транзитивно на множестве $M$, состоящим из $n$ элементов, и пусть задано неизмельчимое разбиение $M$ на блоки. Тогда стабилизатор любого блока $M'$ является подгруппой группы $G$, примитивно действующей на $M'$.
\end{theorem}

\begin{proof}
Напомним, что $$\St M'=\{g\mid g\in G, x\in M', gx\in M'\}.$$ Очевидно, что $H$ --- подгруппа. Предположим, что она не примитивна, т.е. существует нетривиальное разбиение $M'$ на блоки $M'_1,\ldots, M'_k$. Пусть $g_1,\ldots,g_l$ --- элементы смежных классов $g_1H,\ldots,g_lH$. Тогда исходное разбиение $M$ можно измельчить до разбиения $g_iM'_j$.
\end{proof}

\begin{theorem}
\label{Tr3} Пусть $G$ действует транзитивно на множестве $M$. Пусть $H=\St(x)$, $x\in M$. Тогда если $G$ импримитивна, то существует подгруппа $K\neq H,G$ такая, что $H\subset K\subset G$. Обратно, если существует такая группа, то $G$ импримитивна.
\end{theorem}

\begin{proof}
Предположим, что $G$ импримитивна. Тогда она разбивается на нетривиальные блоки $M_1, M_2,\ldots$. Пусть $x\in M_1$ и $K=\St(M_1)$. Тогда $H\subset K\subset G$. Поскольку $G$ действует транзитивно на множестве $M$, то существует элемент $g_1\in G$ такой, что $g_1$ переводит $x$ в другой элемент множества $M_1$. Тогда $g_1\in K$, $g_1\not\in H$. С другой стороны, существует элемент $g_2\in G$ такой, что $g_2$ переводит $M_1$ в $M_2$, т.е. $g_2\not\in K$ и $K\neq G$.

Обратно, пусть существует подгруппа $K$ такая, что $H\subset K\subset G$ и $K\neq H,G$. Пусть $M_1$ --- орбита элемента $x$ при действии подгруппы $K$. Поскольку $K\neq H$, то $M_1$ состоит не только из элемента $x$. Рассмотрим левые смежные классы по подгруппе $K$, $g_1K,g_2K\ldots$. Пусть $M_i=g_i M_1$. Докажем, что эти множества не пересекаются. Предположим, что $y\in M_i\cap M_j$. Тогда $y=g_i x_1=g_j x_2$, где $x_1,x_2\in M_1$. Следовательно, $y=g_i h_1 x=g_j h_2 x$, где $h_1,h_2\in K$. Тогда $h^{-1}_2g^{-1}_j g_ih_1 x=x$. Отсюда, $h^{-1}_2g^{-1}_j g_ih_1\in H\subset K$. Следовательно, $g^{-1}_j g_i\in K$. Тогда $g_i K= g_j K$. Противоречие. Таким образом, мы получили разбиение на нетривиальные блоки. Следовательно, группа $G$ импримитивна.
\end{proof}

\begin{corollary}
\label{Tr4}
Для того, чтобы группа $G$ была примитивной необходимо и достаточно, чтобы стабилизатор точки был максимальной подгруппой.
\end{corollary}

\begin{theorem}
\label{Tr5} Пусть $G$ действует примитивно на множестве $M$. Пусть $H$ --- нормальная подгруппа в $G$. Тогда либо $H$ действует транзитивно на $M$, либо $H$ действует тривиально (т.е. оставляет все элементы $M$ на месте).
\end{theorem}

\begin{proof}
Предположим, что $H$ действует нетранзитивно. Тогда множество $M$ можно разбить на множества $M_1,M_2,\ldots$ орбит группы $H$. Докажем, что любой элемент $g\in G$ переводит одну орбиту в другую. Пусть $x\in M_1$. Предположим, что $g x=y\in M_2$. Рассмотрим $z=h_1y$, где $h_1\in H$. Тогда $$t=g^{-1}z= g^{-1}h_1y=g^{-1}h_1g x=h_2 x,$$ где $h_2\in H$. Следовательно, $t\in M_1$ и $gt=z$. Таким образом, мы получили разбиение $M$ на нетривиальные блоки.
\end{proof}


\chapter{Теория полей}

\section{Конечные и алгебраические расширения полей}

Пусть $E,k$ --- два поля, причем $k\subset E$. Тогда поле $E$ называется \emph{расширением} поля $k$.

\begin{definition}
Расширение $E$  поля $k$ называется \emph{конечным} (\emph{бесконечным}), если $E$ конечномерно (бесконечномерно), как линейное пространства над $k$. Другими словами, $E$ конечно над $k$, если существуют $a_1,a_2,\ldots, a_n\in E$ такие, что $\forall x\in E$, $x=\alpha_1 a_1+\alpha_2 a_2+\cdots+\alpha_n a_n$, где $\alpha_1,\alpha_2\ldots,\alpha_n\in k$. \emph{Степенью} $E$ над $k$ мы будем называть размерность $E$ как линейного пространства и обозначать $[E:k]$.
\end{definition}

\begin{theorem}
\label{Ras1} Пусть $E$ --- конечное расширение поля $k$, $F$ --- конечное расширение поля $E$. Тогда $F$ --- конечное расширение поля $k$ и $[F:k]=[E:k][F:E]$.
\end{theorem}

\begin{proof}
Пусть $x_1,x_2,\ldots, x_n$ --- базис $E$ над полем $k$, $y_1,y_2,\ldots, y_m$ --- базис $F$ над полем $E$. Тогда для любого элемента $a\in F$ существует разложение $$a=\alpha_1 y_1+\cdots+\alpha_m y_m,$$ где $\alpha_1,\ldots\alpha_m\in E$. Поскольку $E$ --- конечное расширение поля $k$, то $$\alpha_i=\beta_{i1}x_1+\cdots\beta_{in}x_n,$$ где $\beta_{ij}\in k$. Таким образом, $$a=\sum\limits_{i=1}^m\sum\limits_{j=1}^n\beta_{ij}x_jy_i.$$ Следовательно, $\{x_jy_i\}$ порождают $F$ над $k$. Таким образом, $F$ --- конечное расширение поля $k$. Осталось доказать равенство $[F:k]=[E:k][F:E]$. Для этого докажем линейную независимость $\{x_jy_i\}$. Предположим противное, т.е. существуют элементы $c_{ij}$ такие, что $$\sum\limits_{i=1}^m\sum\limits_{j=1}^n c_{ij}x_jy_i=0.$$ С другой стороны, $$\sum\limits_{i=1}^m\sum\limits_{j=1}^n c_{ij}x_jy_i=\left(\sum\limits_{j=1}^n c_{1j} x_j\right)y_1+\left(\sum\limits_{j=1}^n c_{2j} x_j\right)y_2+\cdots+\left(\sum\limits_{j=1}^n c_{mj} x_j\right)y_m.$$ Заметим, что $\sum\limits_{j=1}^n c_{ij} x_j\in E$. Поскольку $y_1,y_2,\ldots, y_m$ линейно независимы, то все $\sum\limits_{j=1}^n c_{ij} x_j=0$. Поскольку $x_1,x_2,\ldots, x_n$ линейно независимы, то все $c_{ij}=0$.
\end{proof}

\begin{remark}
Если $k\subset E\subset F$ и $F$ --- конечное расширение поля $k$, то очевидно, что $E$ --- конечное расширение поля $k$, а $F$ --- конечное расширение поля $E$.
\end{remark}

\begin{definition}
Элемент $x\in E$ называется \emph{алгебраическим}, если он является корнем многочлена с коэффициентами из $k$, т.е. существуют  $\alpha_0,\alpha_1\ldots,\alpha_n\in k$ такие, что $\alpha_0+\alpha_1 x+\alpha_2 x^2+\cdots\alpha_n x^n=0.$
Расширение $E$  поля $k$ называется \emph{алгебраическим}, если любой элемент $E$ является алгебраическим.
\end{definition}

\begin{theorem}
\label{Ras2} Любое конечное расширение является алгебраическим.
\end{theorem}

\begin{proof}
Пусть $E$ --- конечное расширение поля $k$, и пусть $a\in E$. Если $a\in k$, то он алгебраичен. Предположим, что $a\not\in k$. Рассмотрим $1,a,a^2\cdots a^n\cdots$. Поскольку $E$ --- конечное расширение поля $k$, то существует $n$ такое, что элементы $1,a,a^2\cdots a^n$ линейно зависимы. Тогда существуют  $\alpha_0,\alpha_1\ldots,\alpha_n\in k$ такие, что $\alpha_0+\alpha_1 a+\alpha_2 a^2+\cdots\alpha_n a^n=0.$
\end{proof}

Пусть $E$ --- расширение поля $k$, и $a_1,a_2,\ldots, a_n\in E$ обозначим через $k(a_1,a_2,\ldots, a_n)$ наименьшее подполе поля $E$, содержащее $a_1,a_2,\ldots,a_n$. Очевидно оно состоит из элементов вида $$\frac{f(a_1,a_2,\ldots,a_n)}{g(a_1,a_2,\ldots,a_n)},$$ где $f,g$ --- многочлены с коэффициентами из $k$ и $g(a_1,a_2,\ldots,a_n)\neq 0$.

\begin{theorem}
\label{Ras3} Пусть $E$ --- расширение поля $k$ и $a\in E$ алгебраичен над $k$. Тогда $k(a)$ --- конечное расширение поля $k$.
\end{theorem}

\begin{proof}
Пусть $f(x)$ --- многочлен с коэффициентами из $k$ такой, что $f(a)=0$. Предположим, что $f(x)$ приводим над $k$, т.е. $f(x)=f_1(x)f_2(x)$, где $f_1(x),f_2(x)$ --- многочлены над $k$, степени меньше степени $f(x)$. Тогда либо $f_1(a)=0$, либо $f_2(a)=0$. Таким образом, последовательно заменяя $f(x)$ на многочлены меньшей степени, мы можем считать, что $f(x)$ неприводим. Рассмотрим $k[x]$ --- множество многочленов от $x$ с коэффициентами из $k$. Пусть $g(x)\in k[x]$ такой, что $g(a)\neq 0$. Тогда $g(x)$ взаимно прост с $f(x)$. Следовательно, существуют многочлены $p(x), q(x)$ такие, что $f(x)p(x)+g(x)q(x)=1$. Подставляя $a$, получаем $g(a)q(a)=1$. Таким образом, $k[a]$ не только кольцо, но и поле. Очевидно, что размерность $k[a]$ как векторного пространства над $k$ не превышает степени многочлена $f(x)$.
\end{proof}

\begin{remark}
Заметим, что многочлен $f(x)$ единственен с точностью до умножения на константу. Мы можем считать, что коэффициент при старшей степени у этого многочлена равен 1. Действительно, пусть существует другой неприводимый многочлен $f'(x)$ такой, что $f'(a)=0$. Поскольку они оба неприводимы, то они взаимно просты. Тогда существуют многочлены $p(x), q(x)$ такие, что $f(x)p(x)+f'(x)q(x)=1$. Подставляем $a$, получаем противоречие. Таким образом, мы можем считать, что старший коэффициент многочлена $f(x)$ равен 1. Такой многочлен мы будем называть \emph{минимальным многочленом элемента} $a$ над $k$, и обозначать $\Irr(a,k,x)$.
\end{remark}

\begin{corollary}
\label{Ras4}
Пусть $E$ --- расширение поля $k$ и $a_1,a_2,\ldots,a_n\in E$ алгебраичны над $k$. Тогда $k(a_1,a_2,\ldots,a_n)$ --- конечное расширение поля $k$.
\end{corollary}

\begin{proof}
Заметим, что $$k\subset k(a_1)\subset k(a_1,a_2)\subset\cdots \subset k(a_1,a_2,\ldots,a_n).$$ Поскольку $k(a_1,a_2,\ldots,a_i, a_{i+1})=k(a_1,a_2,\ldots,a_i)(a_{i+1})$, то согласно теореме \ref{Ras3} каждое вложение является конечным расширением. Теперь утверждение следует из теоремы \ref{Ras1}.
\end{proof}

\begin{theorem}
\label{Ras5}
Пусть $E$ --- алгебраическое расширение поля $k$ и $F$ --- алгебраическое расширение поля $E$. Тогда $F$ --- алгебраическое расширение поля $k$.
\end{theorem}

\begin{proof}
Пусть $x\in F$. Тогда $$a_0+a_1x+\cdots+a_n x^n=0,$$ где $a_0,a_1,\ldots,a_n\in E$. Рассмотрим $E_0=k(a_0,a_1,\ldots,a_n)$. Согласно следствию \ref{Ras4} $E_0$ --- конечное расширение $k$. Рассмотрим $F_0=E_0(x)$. Аналогично, $F_0$ --- конечное расширение $E_0$. Следовательно, по теореме \ref{Ras1}, $F_0$ --- конечное расширение $k$. Заметим, что $x\in F_0$. С другой стороны, согласно теореме \ref{Ras2}, $F_0$ --- алгебраическое расширение поля $k$. Следовательно, $x$ алгебраичен.
\end{proof}

\begin{remark}
Если $k\subset E\subset F$ и $F$ --- алгебраическое расширение поля $k$, то очевидно, что $E$ --- алгебраическое расширение поля $k$, а $F$ --- алгебраическое расширение поля $E$.
\end{remark}

\begin{definition}
Пусть $E$ и $F$ --- произвольные поля, содержащиеся в поле $L$. Наименьшее подполе в $L$, содержащие $E$ и $F$ называется \emph{композитом} и обозначается $EF$. \emph{Композитом} семейства подполей $\{E_i\}$ в $L$ называется наименьшее подполе в $L$, содержащее все семейство $\{E_i\}$. Пусть $E=k(\alpha_1,\alpha_2,\ldots,\alpha_n)$ и $F$ --- расширение поля $k$. Предположим, что $E$ и $F$ содержатся в $L$. Тогда $EF=F(\alpha_1,\alpha_2,\ldots,\alpha_n)$. Мы будем называть расширение $EF$ поля $F$ \emph{подъемом} $E$ до $F$.
\end{definition}

\begin{theorem}
\label{Ras6}
Пусть $E$ --- конечное расширение поля $k$ и $F$ --- любое расширение поля $k$. Предположим, что существует поле $L$, содержащее $E$ и $F$. Тогда $EF$ --- конечное расширение поля $F$.
\end{theorem}

\begin{proof}
Пусть $E$ --- конечное расширение поля $k$. Тогда существуют элементы $\alpha_1,\alpha_2,\ldots,\alpha_n\in E$, алгебраичные над $k$ такие, что $E=k(\alpha_1,\alpha_2,\ldots,\alpha_n)$. Согласно \ref{Ras4} $EF=F(\alpha_1,\alpha_2,\ldots,\alpha_n)$ --- конечное расширение $F$.
\end{proof}

\begin{corollary}
\label{Ras7}
Пусть $E$ и $F$ --- конечные расширения поля $k$. Предположим, что существует поле $L$, содержащее $E$ и $F$. Тогда $EF$ --- конечное расширение поля $k$.
\end{corollary}

\begin{proof}
Следует из\ref{Ras1} и \ref{Ras6}.
\end{proof}

\begin{theorem}
\label{Ras8}
Пусть $E$ --- алгебраическое расширение поля $k$ и $F$ --- любое расширение поля $k$. Предположим, что существует поле $L$, содержащее $E$ и $F$. Тогда $EF$ --- алгебраическое расширение поля $F$.
\end{theorem}

\begin{proof}
Пусть $\alpha\in E$ --- алгебраический элемент над $k$. Тогда $\alpha$ алгебраичен над $F$ (любой многочлен из $k[x]$ является многочленом в $F[x]$).
\end{proof}

\begin{corollary}
\label{Ras9}
Пусть $E$ и $F$ --- конечные расширения поля $k$. Предположим, что существует поле $L$, содержащее $E$ и $F$. Тогда $EF$ --- конечное расширение поля $k$.
\end{corollary}

\begin{proof}
Следует из\ref{Ras5} и \ref{Ras8}.
\end{proof}

\section{Нормальные расширения полей}

Пусть $p(x)$ --- неприводимый многочлен над полем $k$. Рассмотрим кольцо многочленов $k[x]$. Тогда многочлен $p(x)$ порождает главный идеал $(p(x))$. Поскольку $p(x)$ неприводим, то $(p(x))$ --- максимальный идеал. Следовательно, $k[x]/(p(x))$ --- поле. Пусть $\sigma\colon k[x]\rightarrow k[x]/(p(x))$ --- естественный гомоморфизм. Заметим, что $\sigma$ сюръективен на $k$. Тогда $\sigma(k)$ --- подполе поля $k[x]/(p(x))$ изоморфное $k$. Мы можем отождествить его с $k$. Тогда $E=k[x]/(p(x))$ является расширением поля $k$. Рассмотрим $\xi=\sigma(x)$. Заметим, что $\xi$ является корнем многочлена  $p(x)$ в $E$. Таким образом, мы получили следующее утверждение.

 \begin{claim}
\label{Nor1}
Для любого многочлена $p(x)\in k[x]$ существует расширение поля $k$ в котором $p(x)$ имеет корень.
\end{claim}

\begin{definition}
Поле $E$ называется \emph{алгебраически замкнутым}, если любой многочлен $f(x)\in E[x]$ имеет корень.
\end{definition}

\begin{theorem}
\label{Zam}
Для любого поля $k$ существует алгебраическое расширение $\bar{k}$ такое, что $\bar{k}$ алгебраически замкнуто.
\end{theorem}

Пусть $k$ --- поле, $\sigma\colon k\rightarrow L$ --- вложение поля $k$ в алгебраически замкнутое поле $L$. Пусть $E=k(\alpha)$, $p(x)=\Irr(\alpha,k,x)$. Пусть $p^{\sigma}(x)$ --- образ многочлена $p(x)$ в $L$, $\beta$ --- корень $p^{\sigma}(x)$. Заметим, что любой элемент из $E$ можно записать в виде $f(\alpha)$, где $f(x)\in k[x]$. Определим продолжение $\sigma$ как отображение $f(\alpha)\rightarrow f^{\sigma}(\beta)$. Это отображение не зависит от $f(x)\in k[x]$. Действительно, пусть есть $g(x)\in k[x]$ такой, что $f(\alpha)=g(\alpha)$. Тогда $f(\alpha)-g(\alpha)=0$. Следовательно, $f(x)-g(x)$ делится на $p(x)$. Отсюда, $f^{\sigma}(x)-g^{\sigma}(x)$ делится на $p^{\sigma}(x)$. Тогда $f^{\sigma}(\beta)-g^{\sigma}(\beta)=0$ и $f^{\sigma}(\beta)=g^{\sigma}(\beta)$. Таким образом, мы получили продолжение $\sigma$ на $E=k(\alpha)$.

\begin{remark}
Данное продолжение не единственно и зависит от выбора $\beta$.
\end{remark}

\begin{lemma}
\label{Lem1}
Пусть $E$ --- алгебраическое расширение поля $k$, и $\sigma\colon E\rightarrow E$ --- гомоморфизм. Тогда $\sigma$ --- автоморфизм.
\end{lemma}

\begin{proof}
Очевидно, что $\sigma$ инъективен. Осталось доказать, что он сюръективен. Пусть $\alpha$ --- произвольный элемент из $E$, и $p(x)$ --- его минимальный многочлен. Рассмотрим подполе $E'$ порожденное всеми корнями $p(x)$, лежащими в $E$. Тогда $E'$ --- конечное расширение поля $k$. Поскольку $\sigma$ отображает каждый корень многочлена $p(x)$ в корень этого же многочлена, то $\sigma$ отображает $E'$ в себя. Тогда $\sigma(E')$ --- подпространство в $E'$, имеющее ту же размерность, что и $E'$. Следовательно, $\sigma(E')=E'$. Поскольку $\alpha\in E'$, то $\alpha$ лежит в образе $\sigma$.
\end{proof}

\begin{theorem}
\label{Nor2}
Пусть $k$ --- поле, $E$ --- алгебраическое расширение поля $k$, и $\sigma\colon k\rightarrow L$ --- вложение поля $k$ в алгебраически замкнутое поле $L$. Тогда существует продолжение $\sigma$ до вложения $E$ в $L$. Если $L$ алгебраически замкнуто, и $L$ алгебраично над $\sigma k$, то любое продолжение $\sigma$ будет изоморфизмом поля $E$ на $L$.
\end{theorem}

\begin{proof}
Рассмотрим множество $S$ пар $(F,\tau)$, где $F$ --- подполе $E$, содержащие $k$, $\tau$ --- продолжение $\sigma$ до вложения $F$ в $L$. Мы пишем $(F,\tau)<(F',\tau')$, если $F\subset F'$ и $\tau'$ совпадает с $\tau$ на $F$. Заметим, что $S$ не пусто ($(k,\sigma)\in S$). Рассмотрим линейно упорядоченое подмножество $(F_{i},\tau_{i})$. Пусть $F=\cup F_i$, и $\tau=\tau_i$ на каждом $F_i$. Тогда $(F,\tau)$ --- верхняя грань этого упорядоченного подмножества. Тогда существует максимальный элемент $(K,\lambda)$. Мы утверждаем, что $K=E$. Действительно, пусть $K\neq E$. Тогда существует $\alpha\in E$, $\alpha\not\in K$. Мы знаем, что $\lambda$ имеет продолжение на $K(\alpha)$ вопреки максимальности $(K,\lambda)$. Если $L$ алгебраически замкнуто, и $L$ алгебраично над $\sigma k$, то $\sigma E$ алгебраически замкнуто и $L$ алгебраично над $\sigma E$. Отсюда, $L=\sigma E$.
\end{proof}

\begin{example}
$\bar{\RR}=\CC=\RR[x]/(x^2+1)$. Поскольку существуют трансцендентные числа (например $e$ и $\pi$), то $\bar{\QQ}\neq\CC$.
\end{example}

\begin{definition}
Пусть $k$ --- поле, $f(x)\in k[x]$. \emph{Полем разложения} многочлена $f(x)$ мы будем называть расширение $K$ поля $k$, в котором $f(x)$ разлагается на линейные множители, т.е. $$f(x)=c(x-\alpha_1)(x-\alpha_2)\cdots(x-\alpha_n),$$ где все $\alpha_i\in K$ и $K=k(\alpha_1,\alpha_2,\ldots,\alpha_n)$.
\end{definition}

\begin{theorem}
\label{Nor3}
Пусть $K$ --- поле разложения многочлена $f(x)\in k[x]$, и $E$ --- другое поле разложения $f(x)$. Тогда существует изоморфизм $\sigma\colon E\rightarrow K$, индуцирующий тождественное отображение на $k$ (такой изоморфизм мы будем называть $k$-изоморфизмом). Более того, если $k\subset K\subset\bar{k}$, то любое вложение $E$ в $\bar{k}$ является $k$-изоморфизмом $E$ на $K$.
\end{theorem}

\begin{proof}
Пусть $\bar{K}$ --- алгебраическое замыкание поля $K$. Тогда $\bar{K}$ алгебраичен над $k$ и, следовательно, $\bar{K}=\bar{k}$. Согласно теореме \ref{Nor2} существует вложение $\sigma\colon E\rightarrow\bar{K}$, индуцирующее тождественное отображение на $k$. Заметим, что $$f(x)=c(x-\beta_1)(x-\beta_2)\cdots (x-\beta_n),$$ где $\beta_i\in E$, $c\in k$. Тогда $$f(x)=f^{\sigma}(x)=c(x-\sigma(\beta_1))(x-\sigma(\beta_2))\cdots (x-\sigma(\beta_n)).$$ С другой стороны, $f(x)$ имеет в $K[x]$ разложение $$f(x)=c(x-\alpha_1)(x-\alpha_2)\cdots (x-\alpha_n).$$ Поскольку разложение многочлена единственно в $\bar{K}[x]$, то $(\sigma(\beta_1),\sigma(\beta_2),\ldots,\sigma(\beta_n))$ отличается от $(\alpha_1,\alpha_2,\ldots,\alpha_n)$ только перестановкой. Отюсда, $\sigma(\beta_i)\in K$ для любого $i$. Поскольку $E=k(\beta_1,\beta_2,\ldots,\beta_n)$, то $\sigma E\subset K$. С другой стороны, $K=k(\alpha_1,\alpha_2,\ldots,\alpha_n)=k(\sigma(\beta_1),\sigma(\beta_2),\ldots,\sigma(\beta_n))$. Тогда $\sigma E=K$.
\end{proof}

\begin{corollary}
\label{Nor3-1}
Пусть $\tau\colon k_1\rightarrow k_2$ --- изоморфизм двух полей, $f(x)\in k_1[x]$ --- многочлен степени $n$, $\bar{f}(x)=\tau(f(x))\in k_2[x]$. Пусть $k'_1$ и $k'_2$ --- поля разложения над $k_1$ и $k_2$ многочленов $f(x)$ и $\bar{f}(x)$ соответственно. Тогда $\tau$ может быть продолжен до изоморфизма $\varrho\colon k'_1\rightarrow k'_2$, и любое такое продолжение переводит каждый корень многочлена $f(x)$ в корень многочлена $\bar{f}(x)$.
\end{corollary}

\begin{remark}
Заметим, что всякий многочлен $f(x)\in k[x]$ имеет поле разложения, а именно поле, порожденное всеми его корнями в $\bar{k}$.
\end{remark}

Пусть $\{f_i\}$ --- семейство многочленов из $k[x]$. \emph{Полем разложения} этого семейства мы будем называть расширение $K$ поля $k$ такое, что любой $f_i$ разлагается на линейные множители в $K[x]$, и $K$ порождается корнями многочленов $\{f_i\}$.

\begin{remark}
Если семейство $f_1,f_2,\ldots, f_n$ конечно, то полем разложения этих многочленов будет поле разложения одного многочлена $$f(x)=f_1(x)f_2(x)\cdots f_n(x).$$
\end{remark}

\begin{definition}
Расширение $K$ поля $k$ называется \emph{нормальным}, если $K$ --- алгебраическое расширение поля $k$, и любой неприводимый многочлен из $k[x]$, имеющий корень в $K$ разлагается на линейные множители.
\end{definition}

\begin{proposition}
\label{Nor4}
Пусть $K$ --- конечное нормальное расширение поля $k$. Тогда $K$ --- поле разложения некоторого многочлена $f(x)\in k[x]$.
\end{proposition}

\begin{proof}
Поскольку $K$ --- конечное расширение поля $k$, то $K=k(\alpha_1,\alpha_2,\ldots,\alpha_n)$. Пусть $f_i(x)=\Irr(\alpha_i,k,x)$ --- минимальный многочлен элемента $\alpha_i$. Поскольку $K$ --- нормальное расширение поля $k$, то $K$ содержит поле разложения $f_i(x)$. Тогда  $K$ содержит поле разложение многочлена $f(x)=f_1(x) f_2(x)\cdots f_n(x)$. Поскольку $K$ порождается корнями $f(x)$ (а именно $\alpha_1,\alpha_2,\ldots,\alpha_n$), то $K$ --- поле разложения многочлена $f(x)$.
\end{proof}

\begin{theorem}
\label{Nor5}
Поле разложения многочлена $f(x)\in k[x]$ над $k$ является конечным нормальным расширением поля $k$.
\end{theorem}

\begin{proof}
Пусть $K$ --- поле разложения многочлена $f(x)\in k[x]$ над $k$, и $g(x)$ --- любой неприводимый многочлен над $k$ имеющий корень $\alpha$ в поле $K$. Пусть $K'$ --- поле разложения многочлена $g(x)$ над $K$. Пусть $\beta\in K'$ --- корень $g(x)$. Поскольку $g(x)$ неприводим над $k$, то существует $k$-изоморфизм $\tau$ между $k(\alpha)$ и $k(\beta)$, переводящий $\alpha$ в $\beta$. Этот изоморфизм оставляет $f(x)$ на месте. Заметим, что $K$ и $K(\beta)$ --- поля разложения $f(x)$ над $k$ и $k(\beta)$ соответственно. Таки образом, согласно следствию \ref{Nor3-1}, изоморфизм $\tau$ продолжается до изоморфизма $\varrho$ поля $K$ на поле $K(\beta)$. Поскольку $\varrho$ --- $k$-изоморфизм и $f(x)$ разлагается на линейные множители в $K$, то $\varrho$ переводит множество корней $f(x)$ в себя. Заметим, что множество корней $f(x)$ порождают $K$. Следовательно, $\varrho$ --- автоморфизм поля $K$. Поскольку $\alpha\in K$, то $\varrho(\alpha)=\beta\in K$. Таким образом, $K$ содержит все корни многочлена $g(x)$. Отсюда следует, что $K$ --- нормальное расширение поля $k$.
\end{proof}

\begin{theorem}
\label{Nor6}
Пусть $K$ --- алгебраическое расширение поля $k$, и $k\subset K\subset\bar{k}$, где $\bar{k}$ --- алгебраическое замыкание $k$. Тогда $K$ --- нормальное расширение поля $k$ тогда и только тогда, когда всякое вложение $\sigma\colon K\rightarrow\bar{k}$ над $k$ является автоморфизмом поля $K$.
\end{theorem}

\begin{proof}
Предположим, что всякое вложение $\sigma\colon K\rightarrow\bar{k}$ над $k$ является автоморфизмом поля $K$.
Пусть $f(x)\in k[x]$ --- неприводимый многочлен над $k$, и $\alpha\in K$ --- его корень. Пусть $\beta\in \bar{k}$ --- другой корень этого многочлена. Тогда существует $k$-изоморфизм $\sigma$ полей $k(\alpha)$ и $k(\beta)$. Продолжим этот изоморфизм до вложения $K$ в $\bar{k}$. По предположению, это вложение является автоморфизмом поля $K$. Отсюда, $\beta\in K$.

Обратно, пусть $K$ --- нормальное расширение поля $k$. Пусть $\sigma\colon K\rightarrow\bar{k}$ --- вложение над $k$, и $\alpha\in K$. Пусть $p(x)$ --- минимальный многочлен $\alpha$ над $k$. Поскольку $\sigma$ --- вложение над $k$, то $\sigma$ отображает $\alpha$ в корень $\beta$ многочлена $p(x)$. Поскольку $K$ --- нормальное расширение поля $k$, то $\beta\in K$. Следовательно, $\sigma$ --- автоморфизм поля $K$ (см. \ref{Lem1}).
\end{proof}

\begin{proposition}
\label{Nor7}
Пусть $E$ --- расширение поля $k$ степени два. Тогда $E$ --- нормальное расширение.
\end{proposition}

\begin{proof}
Пусть $f(x)\in k[x]$ --- неприводимый многочлен над $k$, и $\alpha\in E$ --- корень $f(x)$. Тогда $E=k(\alpha)$. Пусть $K$ --- поле разложения $f(x)$. Заметим, что $E\subset K$. Рассмотрим минимальный многочлен $p(x)\in k[x]$ элемента $\alpha$. Поскольку $E$ --- расширение степени два, то $p(x)$ имеет степень два. Следовательно, существует $\bar{\alpha}\in E$ такой, что $p(\bar{\alpha})=0$. Пусть $\tau$ --- автоморфизм поля $E$ переводящий $\alpha$ в $\bar{\alpha}$. Поскольку $\tau$ --- $k$-изоморфизм, то он оставляет $f(x)$ на месте. Следовательно, он продолжается до автоморфизма поля $K$. Тогда $\bar{\alpha}=\tau(\alpha)$ является корнем $f(x)$. Отсюда следует, что $p(x)$ делит $f(x)$ (т.е. $f(x)=cp(x)$, $c\in k$). Тогда $E$ --- поле разложения многочлена $f(x)$.
\end{proof}

\begin{example}
Алгебраическое замыкание является нормальным расширением.
\end{example}

\begin{example}
Пусть $E=\QQ(\sqrt[4]{2})$. Тогда $E$ не является нормальным расширением поля $\QQ$, $E$ не содержит комплексные корни многочлена $x^2-2$. С другой стороны, пусть $F=\QQ(\sqrt{2})$. Тогда $\QQ\subset F\subset E$, при этом $F$ --- расширение поля $\QQ$ степени два, и $E$ расширение поля $F$ степени два, т.е. оба эти расширения нормальны.
\end{example}

\begin{theorem}
\label{Nor8}
Пусть $k\subset E\subset K$, и $K$ --- нормальное расширение поля $k$. Тогда $K$ --- нормальное расширение поля $E$.
\end{theorem}

\begin{proof}
Рассмотрим вложение полей $k\subset E\subset K$ в алгебраическое замыкание $\bar{k}$. Пусть $\sigma\colon K\rightarrow \bar{k}$ --- любое вложение $K$ над $E$. Тогда $\sigma$ является вложением и над $k$. По теореме \ref{Nor6} $\sigma$ является автоморфизмом поля $K$. По той же теореме, $K$ --- нормальное расширение поля $E$.
\end{proof}

\section{Сепарабельные расширения полей}

\begin{definition}
Пусть $k$ --- поле. Предположим, что существует такое число $p$, что $p\cdot 1=0$, т.е. $$\underbrace{1+1+\cdots+1}_{\text{$p$ слагаемых}}=0.$$ Пусть $p$ --- минимальное из таких чисел. Тогда говорят, что $p$ --- \emph{характеристика поля} $k$. Обозначается $char(k)$. Если не существует такого положительного числа $p$, то говорим, что поле имеет характеристику ноль.
\end{definition}

 \begin{claim}
\label{Har1}
Характеристика поля либо ноль, либо простое число.
\end{claim}

\begin{proof}
Предположим, что характеристика поля $p=mn$. Тогда $$\underbrace{1+1+\cdots+1}_{\text{$p$ слагаемых}}=\underbrace{(1+1+\cdots+1)}_{\text{$m$ слагаемых}}\cdot\underbrace{(1+1+\cdots+1)}_{\text{$n$ слагаемых}}=0.$$ Отсюда, либо $$\underbrace{1+1+\cdots+1}_{\text{$m$ слагаемых}}=0,$$ либо $$\underbrace{1+1+\cdots+1}_{\text{$n$ слагаемых}}=0.$$
\end{proof}

Рассмотрим поле $k$ характеристики $p$.

 \begin{claim}
\label{Har1}
Пусть $k$ --- поле характеристики $p$. Тогда $(a+b)^p=a^p+b^p$.
\end{claim}

\begin{proof}
Следует из формулы Бинома--Ньютона и того, что $C_p^i$ делится на $p$ для любого $i\neq 0,p$.
\end{proof}

\begin{definition}
Поскольку $(a+b)^p=a^p+b^p$ и $(ab)^p=a^p b^p$, то отображение $f\colon k\rightarrow k^p$ заданное $f(x)=x^p$ является гомоморфизмом. Он называется \emph{морфизмом Фробениуса}.
\end{definition}

\begin{definition}
Поле $k$ называется совершенным, если либо $k$ характеристики ноль, либо $k$ характеристики $p$ и совпадает с $k^p$.
\end{definition}

\begin{theorem}
\label{Sov}
Пусть $k$ --- конечное поле. Тогда $k$ совершенно.
\end{theorem}

\begin{proof}
Заметим, что $k^p$ --- подполе в $k$ и $k^p$ изоморфно $k$. Следовательно, $k^p$ и $k$ имеют одинаковое количество элементов. Тогда они совпадают.
\end{proof}

Рассмотрим $f(x)\in k[x]$, т.е. $$f(x)=a_n x^n+a_{n-1} x^{n-1}+\cdots+a_1x+a_0.$$ Пусть $f'(x)$ --- обычная производная, т.е. $$f'(x)=na_n x^{n-1}+(n-1)a_{n-1} x^{n-2}+\cdots+a_1.$$ Заметим, что если $k$ имеет характеристику ноль, то $f'(x)\neq 0$ при $n\geq 1$. Более того, если $k$ имеет характеристику $p$, то $f'(x)= 0$ тогда и только тогда, когда $f(x)=\tilde{f}(x^p)$, т.е. $f(x)$ --- многочлен от $x^p$.

\begin{definition}
Неприводимый многочлен $f(x)$ называется \emph{сепарабельным}, если $f'(x)\neq 0$ и \emph{несепарабельным}, если $f'(x)=0$. Произвольный многочлен $f(x)$ называется \emph{сепарабельным}, если сепарабельны все его неразложимые множители.
\end{definition}

\begin{remark}
Если $k$ имеет характеристику ноль, то любой многочлен сепарабелен.
\end{remark}

Рассмотрим более подробно связь между полем $k$ и сепарабельности многочленов $f(x)\in k[x]$.

\begin{theorem}
\label{Sov2}
Пусть $k$ --- поле характеристики $p$. Если $a\in k$, $\sqrt[p]{a}\not\in k$, то $x^{p^m}-a$ неразложим в $k[x]$ для любого $m$.
\end{theorem}

\begin{proof}
Докажем индукцией по $m$. Для $m=0$ утверждение очевидно. Пусть $\varphi(x)$ --- приведенный неразложимый множитель многочлена $x^{p^m}-a$ в $k[x]$, и $\varphi^l(x)$  --- наивысшая степень $\varphi(x)$, которая делит $x^{p^m}-a$. Таким образом, $$x^{p^m}-a=\varphi^l(x)\psi(x),$$ где $\varphi(x)$ и $\psi(x)$ взаимно просты. Возьмем производную от обеих частей, получим $$l\varphi^{l-1}(x)\varphi'(x)\psi(x)+\varphi^l(x)\psi'(x)=0.$$ Поделим на $\varphi^{l-1}(x)$, получим $$l\varphi'(x)\psi(x)+\varphi(x)\psi'(x)=0.$$ Заметим, что $\psi(x)$ должен делить $\varphi(x)\psi'(x)$. С другой стороны, $\varphi(x)$ и $\psi(x)$ взаимно просты, а $\psi'(x)$ имеет меньшую степень чем у $\psi(x)$. Тогда $\varphi(x)\psi'(x)=0$, и следовательно, $l\varphi'(x)\psi(x)=0$. Отсюда, $\psi'(x)=0$ и $l\varphi'(x)=0$. Из $\psi'(x)=0$ следует, что $\psi(x)=\psi_1(x^p)$. Предположим, что $l$ не делится на $p$. Тогда, из $l\varphi'(x)=0$ следует, что $\varphi(x)=\varphi_1(x^p)$. Отсюда, заменяя $x$ на $x^p$, получаем $$x^{p^{m-1}}-a=\varphi^l_1(x)\psi_1(x).$$ Противоречие с индуктивным предположением. Пусть $l$ делится на $p$. Тогда $\varphi^l(x)=\varphi_1(x^p)$. Следовательно, $$x^{p^{m-1}}-a=\varphi_1(x)\psi_1(x).$$ Отсюда, $\psi(x)=1$ и $x^{p^m}-a=\tilde{\varphi}^p(x)$. С другой стороны, все коэффициенты $\tilde{\varphi}^p(x)$ принадлежат $k^p$. Противоречие с условием $\sqrt[p]{a}\not\in k$.
\end{proof}

\begin{theorem}
\label{Sov3}
Поле $k$ совершенно тогда и только тогда, когда каждый многочлен положительной степени сепарабелен.
\end{theorem}

\begin{proof}
Мы можем считать, что поле $k$ имеет характеристику $p$. Предположим, что $k$ совершенно, и $f(x)\in k[x]$ --- многочлен такой, что $f'(x)=0$. Тогда $f(x)\in k[x^p]$, т.е. $$f(x)=a_nx^{pn}+a_{n-1}x^{p(n-1)}+\cdots+a_1x^p+a_0.$$ Поскольку $k$ совершенно, то существуют элементы $\alpha_i=\sqrt[p]{a_i}\in k$. Тогда $$f(x)=(\alpha_nx^{n}+\alpha_{n-1}x^{n-1}+\cdots+\alpha_1x+\alpha_0)^p.$$ Следовательно, $f(x)$ разложим.

Предположим, что $k$ несовершенно. Тогда существует элемент $a\in k$ такой, что $\sqrt[p]{a}\not\in k$. Согласно теореме \ref{Sov2} многочлен $f(x)=x^p-a$ неразложим, но $f'(x)=0$.
\end{proof}

\begin{definition}
Пусть $K$ --- расширение поля $k$ и $\alpha\in K$ --- алгебраический элемент. Мы говорим, что $\alpha$ \emph{сепарабелен}, если его минимальный многочлен сепарабелен.
\end{definition}


\begin{definition}
Алгебраическое расширение $K$ поля $k$ называется \emph{сепарабельным}, если каждый элемент поля $K$ сепарабелен над $k$.
\end{definition}

\begin{remark}
Из теоремы \ref{Sov3} следует, что если $k$ совершенно, то любое его алгебраическое расширение сепарабельно.
\end{remark}

\begin{claim}
\label{Sep1}
Пусть $\alpha$ --- алгебраический элемент над $k$, и $f(x)$ --- его минимальный многочлен. Тогда $\alpha$ несепарабелен тогда и только тогда, когда $f'(\alpha)=0$. Более того, если $g(x)\in k[x]$ --- многочлен такой, что $g(\alpha)=0$, то $g'(\alpha)=0$.
\end{claim}

\begin{proof}
Если $\alpha$ несепарабелен, то $f'(x)=0$. Обратно, если $f'(\alpha)=0$, то, поскольку $f(x)$ --- минимальный многочлен, а $f'(x)$ имеет степень на единицу меньше чем $f(x)$, то $f'(x)=0$. Пусть $\alpha$ несепарабелен, и $g(x)\in k[x]$ --- многочлен такой, что $g(\alpha)=0$. Тогда $g(x)$ делится на $f(x)$, т.е. $g(x)=h(x)f(x)$. Тогда $$g'(x)=h'(x)f(x)+h(x)f'(x)=h'(x)f(x).$$ Отсюда, $g'(\alpha)=h'(\alpha)f(\alpha)=0$.
\end{proof}

Пусть $f(x)\in k[x]$ такой, что $f(\alpha)=0$, где $\alpha\in K$, $K$ --- расширение поля $k$. Тогда $f(x)$ делится на $x-\alpha$. Пусть $s$ --- наибольшая степень $x-\alpha$ такая, что $f(x)=(x-\alpha)^sf_1(x)$. Заметим, что $f_1(\alpha)\neq 0$. Более того, поскольку $f(x)\in k[x]$, то $f_1(x)\in k(\alpha)[x]$. В силу единственности разложения $f(x)$ над $k(\alpha)\subset K$, получаем, что  $s$ и $f_1(x)$ не зависят от расширения. Число $s$ называется \emph{кратностью корня} $\alpha$ многочлена $f(x)$. Мы будем говорить, что $\alpha$ --- \emph{простой корень}, если $s=1$, и $\alpha$ --- \emph{кратный корень}, если $s>1$.

\begin{claim}
\label{Sep2}
Пусть $\alpha$ --- алгебраический элемент над $k$, и $f(x)\in k[x]$ --- многочлен такой, что $f(\alpha)=0$. Тогда $\alpha$ --- \emph{кратный корень} тогда и только тогда, когда $f'(\alpha)=0$.
\end{claim}

\begin{proof}
Пусть $f(x)=(x-\alpha)^sf_1(x)$. Тогда $$f'(x)=s(x-\alpha)^{s-1}f_1(x)+(x-\alpha)^sf'_1(x).$$ Если $s>1$, то $f'(\alpha)=0$. Обратно, если $s=1$, то $f'(\alpha)=f_1(\alpha)\neq 0$.
\end{proof}

 \begin{corollary}
\label{Sep3}
Пусть $\alpha$ --- алгебраический элемент над $k$, и $f(x)$ --- его минимальный многочлен. Тогда $\alpha$ несепарабелен тогда и только тогда, когда $\alpha$ --- кратный корень многочлена $f(x)$. Более того, если $g(x)\in k[x]$ --- многочлен такой, что $g(\alpha)=0$, то $\alpha$ --- кратный корень многочлена $g(x)$.
\end{corollary}

\begin{theorem}
\label{Sep4}
Пусть $\alpha$ алгебраичен над $k$, и $f(x)$ --- его минимальный многочлен. Если $char(k)=0$, то все корни $f(x)$ имеют кратность один. Если $char(k)=p>0$, то существует $e$ такое, что все корни $f(x)$ имеют кратность $p^e$.
\end{theorem}

\begin{proof}
Пусть $\alpha$ и $\beta$ --- корни многочлена $f(x)$ в замыкании $\bar{k}$. Тогда существует изоморфизм $\sigma\colon k(\alpha)\rightarrow k(\beta)$, который продолжается до автоморфизма $\bar{k}$. Следовательно, все корни имеют одинаковую кратность $m$. Рассмотрим $f'(x)$. Если $m>1$, то $\alpha$ является корнем многочлена $f'(x)$, степень которого меньше степени $f(x)$. Поскольку $f(x)$ --- минимальный многочлен, то $f'(x)=0$. Следовательно, если $f(x)$ имеет кратные корни, то $char(k)=p>0$  и $f(x)=g(x^p)$. Пусть $f(x)=(x-\alpha)^mf_1(x)$, где $f_1(x)\in k(x)$ и $f_1(\alpha)\neq 0$. Тогда $$f'(x)=m(x-\alpha)^{m-1}f_1(x)+(x-\alpha)^mf'_1(x)=0.$$ Поделив на $(x-\alpha)^{m-1}$, получим $$mf_1(x)+(x-\alpha)f_1(x)=0.$$ Поскольку $f_1(\alpha)\neq 0$, то $m$ делится на $p$, т.е. $m=m_1 p$. Применив морфизм Фробениуса, получаем $$f(x)=(x-\alpha)^mf_1(x)=(x^p-\alpha^p)^{m_1}g_1(x^p).$$ Таким образом, все корни многочлена $g(x)$ имеют кратность $m_1$. Повторяя наше рассуждение, получаем, что либо $m_1=1$, либо $m_1=pm_2$ и $g(x)=h(x^p)$. Продолжая этот процесс, получаем, что все корни имеют кратность $p^e$.
\end{proof}

Рассмотрим еще одно важное отличие сепарабельных и несепарабельных расширений.

Пусть $k$ --- поле и $k(\alpha)$ --- расширение, порожденное алгебраическим элементом $\alpha$, $f(x)$ --- минимальный многочлен элемента $\alpha$. Тогда число вложений $k(\alpha)$ в алгебраическое замыкание $\bar{k}$ равно числу различных корней многочлена $f(x)$. С другой стороны, степень $[k(\alpha):k]=\deg f$. Таким образом, число вложений $k(\alpha)$ в алгебраическое замыкание $\bar{k}$ не превосходит степени $[k(\alpha):k]$. Более того, равенство достигается тогда и только тогда, когда $\alpha$ сепарабелен. Пусть $E$ --- конечное расширение поля $k$. Пусть $[E:k]_S$ --- количество вложений поля $E$ в алгебраическое замыкание $\bar{k}$. Число $[E:k]_S$ будем называть \emph{сепарабельной степенью} $E$ над $k$.

\begin{theorem}
\label{Sep5}
Пусть $E$ --- конечное расширение поля $k$, и $F$ --- конечное расширение поля $E$. Тогда $$[E:k]_S[F:E]_S=[F:k]_S.$$
\end{theorem}

\begin{proof}
Пусть $\sigma_1,\sigma_2,\ldots,\sigma_n$ --- множество вложений $E$ в алгебраическое замыкание $\bar{k}$ над $k$, и $\tau_{i1},\tau_{i2},\ldots, \tau_{im}$ --- множество продолжений $\sigma_i$ до вложения $F$ в $\bar{k}$. Поскольку $\sigma_i\sigma^{-1}_j$ --- изоморфизм полей $\sigma_j E$ и $\sigma_i E$, то количество продолжений одинаково для любого $\sigma_i$. Таким образом, мы получили $nm$ вложений $F$ в алгебраическое замыкание $\bar{k}$. Обратно, пусть $\varrho$ --- вложение $F$ в $\bar{k}$ над $k$. Тогда ограничение $\varrho$ на $E$ совпадает с одним из $\sigma_i$. Следовательно, $\varrho=\tau_{ij}$.
\end{proof}

Теперь рассмотрим один важный критерий сепарабельности.

\begin{theorem}
\label{Sep6}
Пусть $E$ --- конечное расширение поля $k$. Тогда $[E:k]_S\leq [E:k].$ Более того, $[E:k]_S=[E:k]$ тогда и только тогда, когда $E$ --- сепарабельное расширение поля $k$.
\end{theorem}

\begin{proof}
Поскольку $E$ --- конечное расширение поля $k$, то существует башня полей $$k\subset k(\alpha_1)\subset k(\alpha_1,\alpha_2)\subset\cdots\subset k(\alpha_1,\alpha_2,\ldots,\alpha_n)=E.$$ Согласно теоремам \ref{Sep5} и \ref{Ras1}, получаем $$[E:k]_S=[k(\alpha_1):k]_S\cdots[k(\alpha_1,\alpha_2,\ldots,\alpha_n):k(\alpha_1,\alpha_2,\ldots,\alpha_{n-1})]_S,$$ $$[E:k]=[k(\alpha_1):k]\cdots[k(\alpha_1,\alpha_2,\ldots,\alpha_n):k(\alpha_1,\alpha_2,\ldots,\alpha_{n-1})].$$ Мы знаем, что $$[k(\alpha_1,\alpha_2,\ldots,\alpha_i):k(\alpha_1,\alpha_2,\ldots,\alpha_{i-1})]_S\leq$$ $$\leq[k(\alpha_1,\alpha_2,\ldots,\alpha_i):k(\alpha_1,\alpha_2,\ldots,\alpha_{i-1})].$$ Более того, равенство достигается, когда $\alpha_i$ сепарабелен.
\end{proof}

\begin{remark}
Из теоремы \ref{Sep4} следует, что $[E:k]=[E:k]_S p^r$.
\end{remark}

\begin{theorem}
\label{Sep7}
Пусть $E$ --- алгебраическое расширение поля $k$, и $F$ --- алгебраическое расширение поля $E$. Тогда для того, чтобы $F$ было сепарабельным расширением поля $k$ необходимо и достаточно, чтобы $E$ было сепарабельным расширением поля $k$ и $F$ было сепарабельным расширением поля $E$.
\end{theorem}

\begin{proof}
Пусть $F$ --- сепарабельное расширение поля $k$. Заметим, что все элементы поля $E$ являются элементами поля $F$, и следовательно, сепарабельны над $k$. Поскольку каждый элемент из $F$ сепарабелен над $k$, то он сепарабелен и над $E$. Обратно, пусть $E$ --- сепарабельное расширение поля $k$ и $F$ --- сепарабельное расширение поля $E$. Если $E$ и $F$ --- конечные расширения, то утверждение следует из теорем \ref{Sep5} и \ref{Sep6}. Пусть $\alpha\in F$, и $f(x)=a_nx^n+a_{n-1}x^{n-1}+\cdots+a_1x+a_0$ --- его минимальный многочлен. Положим $E_0=k(a_0,a_1,\ldots,a_n)$, $F_0=E_0(\alpha)$. Заметим, что $E_0$ --- конечное расширение поля $k$, $F_0$ --- конечное расширение поля $E_0$. Тогда $F_0$ сепарабельно над $k$. Следовательно, $\alpha\in F$ --- сепарабельный над $k$ элемент.
\end{proof}

\begin{definition}
Пусть $k$ --- поле характеристики $p$. Элемент $\alpha$ называется \emph{чисто несепарабельным} над $k$, если существует целое $l\geq 0$ такое, что $\alpha^{p^l}\in k$. Расширение $K$ поля $k$ называется \emph{чисто несепарабельным}, если все элементы $K$ чисто несепарабельны.
\end{definition}

\begin{theorem}
\label{Sep8}
Пусть $\alpha$ --- одновременно сепарабельный и чисто несепарабельный элемент над $k$. Тогда $\alpha\in k$.
\end{theorem}

\begin{proof}
Предположим, что $\alpha$ --- чисто несепарабельный элемент над $k$. Пусть $l$ --- минимальное число такое, что $\alpha^{p^l}=a\in k$. Тогда $\sqrt[p]{a}\not\in k$. Согласно теореме \ref{Sov2} $x^{p^l}-a$ неразложим. Следовательно, $f(x)=x^{p^l}-a$ --- минимальный многочлен элемента $\alpha$. С другой стороны, $f'(x)=0$. Следовательно, $\alpha$ несепарабелен.
\end{proof}

\begin{theorem}
\label{Sep9}
Пусть $K$ --- конечное сепарабельное расширение поля $k$. Тогда существует элемент $\alpha\in K$ такой, что $K=k(\alpha)$.
\end{theorem}

\begin{proof}
Мы будем предполагать, что $k$ бесконечное поле (доказательство для конечных полей будет дано в следующем параграфе).
Предположим, что $K=k(\alpha,\beta)$. Пусть $n=[K:k]$, и $\sigma_1,\sigma_2,\ldots,\sigma_n$ --- различные вложения $K$ в $\bar{k}$ над $k$. Рассмотрим $$P(x)=\prod\limits_{i\neq j}(\sigma_i\alpha+(\sigma_i\beta)x-\sigma_j\alpha-(\sigma_j\beta)x).$$ Заметим, что $P(x)$ ненулевой многочлен. Тогда существует $c\in k$ такой, что $P(c)\neq 0$. Тогда все элементы $\sigma_i(\alpha+c\beta)$ различны, а следовательно $k(\alpha+c\beta)$ имеет над $k$ степень не меньше $n$. С другой стороны, $[K:k]=n$. Следовательно, $k(\alpha+c\beta)=K$.
\end{proof}

Если $K=k(\alpha)$, то элемент $\alpha$ называется \emph{примитивным элементом} поля $K$ над $k$.

\section{Конечные поля}

В этом параграфе мы рассмотрим конечные поля. Пусть $k$ --- поле из $q$ элементов. Очевидно, что $char(k)=p>0$. Следовательно, поле $k$ содержит $\ZZ_p$ в качестве подполя. Тогда $k$ является конечным расширением поля $\ZZ_p$, т.е. $[k:\ZZ_p]=n$. Таким образом, любой элемент $\alpha\in k$ имеет единственное представление в виде $$\alpha=a_1e_1+a_2e_2+\cdots+a_ne_n,$$ где $e_1,e_2,\ldots,e_n$ --- базис $k$ как векторного пространства над $\ZZ_p$, $a_1,a_2,\ldots,a_n\in\ZZ_p$. Отсюда, число элементов в поле $k$ равно $p^n$.

\begin{theorem}
\label{Kon1}
Пусть $k^*$ --- мультипликативная группа поля $k$, т.е. множество $k\setminus\{0\}$ с операцией умножение. Тогда $k^*$ --- циклическая группа порядка $p^n-1$.
\end{theorem}

\begin{proof}
Предположим, что $k^*$ не является циклической группой. Тогда существует $r<p^n-1$ такое, что $\alpha^r=1$ для любого $\alpha\in k^*$. Таким образом, все элементы $k^*$ являются корнями многочлена $x^r-1=0$, но этот многочлен имеет не более $r$ корней. Противоречие.
\end{proof}

\begin{remark}
Фактически мы доказали, что любая конечная мультипликативная группа в поле циклическая.
\end{remark}

\begin{remark}
Именно из этой теоремы следует теорема \ref{Sep9} для конечных полей. Действительно, если $K$ --- конечное расширение конечного поля $k$, то $K$ --- конечное поле. Тогда его мультипликативная группа $K^*$ --- циклическая. Следовательно, существует $\alpha\in K$ порождающий эту группу. Тогда $K=k(\alpha)$.
\end{remark}

Рассмотрим поле разложения многочлена $f(x)=x^{p^n}-x$ над полем $\ZZ_p$. Мы утверждаем, что это поле состоит из корней $f(x)$. Действительно, если $\alpha,\beta$ --- корни $f(x)$, то $$(\alpha+\beta)^{p^n}-(\alpha+\beta)=\alpha^{p^n}+\beta^{p^n}-\alpha-\beta=0,$$ $$(\alpha\beta)^{p^n}-\alpha\beta=\alpha\beta-\alpha\beta=0,$$ $$(\alpha^{-1})^{p^n}-\alpha^{-1}=(\alpha^{p^n})^{-1}-\alpha^{-1}=\alpha^{-1}-\alpha^{-1}=0,$$ $$(-\alpha)^{p^n}-(-\alpha)=-\alpha+\alpha=0.$$ Заметим, что $0$ и $1$ --- корни $f(x)$. Следовательно, поле разложение многочлена $f(x)=x^{p^n}-x$ состоит из его корней. С другой стороны, $f'(x)=-1$. Следовательно, все корни $f(x)$ различные. Таким образом, мы получили поле состоящее из $p^n$ элементов.

\chapter{Теория Галуа}

\section{Группа автоморфизмов поля}

Пусть $K$ --- поле, и $G$ --- группа автоморфизмов поля $K$. Обозначим через $K^G$ --- множество неподвижных элементов относительно группы $G$. Тогда $K^G$ мы будем называть \emph{неподвижным полем} группы $G$ (или \emph{полем инвариантов} группы $G$). Очевидно, что $K^G$ является полем.

Алгебраическое расширение $K$ поля $k$ называется \emph{расширением Галуа}, если оно нормально и сепарабельно. Мы будем считать, что $K$ вложено в алгебраическое замыкание $k$. Группа автоморфизмов поля $K$ над $k$ называется \emph{группой Галуа} поля $K$ над $k$ и обозначается $G(K/k)$.

\begin{theorem}
\label{Gal1}
Пусть $K$ --- расширение Галуа поля $k$, $G$ --- его группа Галуа. Тогда $k=K^G$. Если $E$ --- промежуточное поле, $k\subset E\subset K$, то $K$ --- расширение Галуа над $E$. Отображение множества промежуточных полей в множество подгрупп группы $G$ инъективно.
\end{theorem}

\begin{proof}
Пусть $\alpha\in K^G$ и $\sigma$ --- вложение $k(\alpha)$ в $\bar{K}$. Продолжим $\sigma$ до вложения $K$ в $\bar{K}$. Тогда $\sigma$ --- автоморфизм поля $K$, и, следовательно, элемент группы $G$. Поскольку $\sigma$ оставляет $\alpha$ неподвижным, то $[k(\alpha):k]_S=1$. Поскольку $\alpha$ сепарабелен, то $\alpha\in k$.

Пусть $E$ --- промежуточное поле. Тогда $K$ нормально и сепарабельно над $E$ (см. теоремы \ref{Nor8} и \ref{Sep7}). Следовательно, $K$ --- расширение Галуа поля $E$. Пусть $H=G(K/E)\subset G$. Тогда $K^H=E$. Пусть $E'$ --- другое промежуточное поле и $H'=G(K/E')$. Если $H=H'$, то $$E=K^H=K^{H'}=E'.$$ Следовательно, отображение $E\rightarrow G(K/E)$ --- инъективно.
\end{proof}

\begin{corollary}
\label{Gal2}
Пусть $K$ --- расширение Галуа поля $k$, $G$ --- его группа Галуа. Пусть $E_1$ и $E_2$ --- два промежуточных поля, $H_1,H_2$ --- группы Галуа поля $K$ над $E_1$ и $E_2$ соответственно. Тогда неподвижное поле наименьшей подгруппы, содержащей $H_1,H_2$, есть $E_1\cap E_2$.
\end{corollary}

\begin{corollary}
\label{Gal3}
Пусть $K$ --- расширение Галуа поля $k$, $G$ --- его группа Галуа. Пусть $E_1$ и $E_2$ --- два промежуточных поля, $H_1,H_2$ --- группы Галуа поля $K$ над $E_1$ и $E_2$ соответственно. Тогда $E_2\subset E_1$ в том и только в том случае, когда $H_1\subset H_2$.
\end{corollary}

Мы будем говорить, что подгруппа $H\subset G$ принадлежит промежуточному полю $E$, если $H=G(K/E)$.

\begin{lemma}
\label{GLem1}
Пусть $E$ --- алгебраическое сепарабельное расширение поля $k$. Предположим, что существует натуральное число $n$ такое, что всякий элемент $\alpha\in E$ имеет степень меньше $n$. Тогда $E$ --- конечное расширение поля $k$ и $[E:k]\leq n$.
\end{lemma}

\begin{proof}
Пусть $\alpha\in E$ --- элемент максимальной степени $m$, т.е. $m=[k(\alpha):k]$ максимальна. Заметим, что $m\leq n$. Предположим, что $k(\alpha)\neq E$. Тогда существует $\beta\in E$ такой, что $\beta\not\in k(\alpha)$. Тогда $$k\subset k(\alpha)\subset k(\alpha,\beta)$$ и $[k(\alpha,\beta),k]>m$. По теореме о примитивном элементе (см. теорема \ref{Sep9}) существует $\gamma\in k(\alpha,\beta)$ такой, что $k(\gamma)=k(\alpha,\beta)$. Тогда степень элемента $\gamma$ равна $[k(\gamma),k]>m$. Противоречие.
\end{proof}

\begin{theorem}
\label{Gal4}
Пусть $K$ --- поле и $G$ --- конечная группа автоморфизмов поля $K$, имеющая порядок $n$. Пусть $k=K^G$. Тогда $K$ --- конечное расширение Галуа поля $k$ и его группа Галуа есть $G$. Более того, $[K:k]=n$.
\end{theorem}

\begin{proof}
Пусть $\alpha\in K$, и пусть $\sigma_1, \sigma_2,\ldots,\sigma_m$ --- максимальное множество элементов из $G$ таких, что $\sigma_1\alpha, \sigma_2\alpha,\ldots,\sigma_m\alpha$ различны. Тогда группа $G$ действует на множестве $\{\sigma_1\alpha, \sigma_2\alpha,\ldots,\sigma_m\alpha\}$ (если $\tau\in G$, то $\tau$ отображает $\{\sigma_1\alpha, \sigma_2\alpha,\ldots,\sigma_m\alpha\}$ в $\{\tau\sigma_1\alpha, \tau\sigma_2\alpha,\ldots,\tau\sigma_m\alpha\}$). Рассмотрим $$f(x)=\prod\limits_{i=1}^m(x-\sigma_i\alpha).$$ Заметим, что $\alpha$ является корнем этого многочлена и любой элемент группы $G$ оставляет $f(x)$ на месте. Следовательно, коэффициенты $f(x)$ лежат в $k$. Таким образом, $K$ --- алгебраическое расширение поля $k$. Поскольку все корни многочлена $f(x)$ имеют кратность один, то $\alpha$ сепарабелен над $k$ (см. \ref{Sep3}). Таким образом, $K$ --- сепарабельное расширение поля $k$. Поскольку $f(x)$ разлагается на линейные множители, то минимальный многочлен любого элемента $\alpha\in K$ над $k$ разлагается на линейные множители. Таким образом, $K$ --- нормальное расширение поля $k$. Следовательно,  $K$ --- расширение Галуа поля $k$. Поскольку степень $f(x)$ меньше порядка группы, любой элемент $\alpha\in K$ имеет степень меньшую $n$. Отсюда, $[K:k]\leq n$. Согласно теореме \ref{Sep6} $n\leq[K:k]$. Следовательно, $[K:k]=n$ и $G$ --- группа Галуа расширения $K$ над $k$.
\end{proof}

\begin{corollary}
\label{Gal5}
Пусть $K$ --- конечное расширение Галуа поля $k$, $G$ --- его группа Галуа. Тогда любая подгруппа $H\subset G$ принадлежит некоторому полю $E$, такому, что $k\subset E\subset K$.
\end{corollary}

\begin{proof}
Пусть $E=K^H$. Согласно теореме \ref{Gal4} $K$ --- расширение Галуа поля $E$ и $H=G(K/E)$.
\end{proof}

\begin{theorem}
\label{Gal6}
Пусть $K$ --- расширение Галуа поля $k$, $G$ --- его группа Галуа. Пусть $E$ --- промежуточное поле, $k\subset E\subset K$, и $H=G(K/E)$. Расширение $E$ над $k$ нормально тогда и только тогда, когда $H$ --- нормальная подгруппа в $G$. Более того, $G(E/k)\cong G/H$.
\end{theorem}

\begin{proof}
Пусть $E$ --- нормальное расширение поля $k$ и $G'=G(E/k)$. Тогда отображение ограничения $\sigma\rightarrow\sigma|_F$ отображает $G$ в $G'$. Ядром этого отображения, по определению, является группа $H$. Следовательно, $H$ --- нормальная подгруппа. Пусть $\tau\in G'$. Тогда $\tau$ продолжается до вложения $K$ в $\bar{K}$, которое должно быть автоморфизмом поля $K$. Следовательно, отображение ограничения сюръективно. Отсюда, $G(E/k)\cong G/H$. Предположим, что $E$ не нормально над $k$. Тогда, согласно теореме \ref{Nor6}, существует вложение $\tau$ поля $E$ в $K$ над $k$, которое не является автоморфизмом, т.е. $\tau E\neq E$. Продолжим $\tau$ до автоморфизма поля $K$. Пусть $\sigma\in H$. Тогда $\tau\sigma\tau^{-1}$ --- элемент группы $G(K/(\tau F))$. Таким образом, группы Галуа $G(K/ F)$ и $G(K/(\tau F))$ сопряжены и, принадлежа разным полям, не могут совпадать.
\end{proof}


\begin{definition}
Расширение Галуа называется \emph{абелевым} (\emph{циклическим}), если группа Галуа абелева (циклическая).
\end{definition}

 \begin{claim}
\label{Gal7}
Пусть $K$ --- абелево (циклическое) расширение Галуа поля $k$, и $E$ --- промежуточное поле, $k\subset E\subset K$. Тогда $E$ --- абелево (циклическое) расширение Галуа поля $k$.
\end{claim}

\begin{proof}
Следует из теоремы \ref{Gal6}.
\end{proof}

Суммируя доказанные утверждения, мы получаем основную теорему теории Галуа.

\begin{theorem}
\label{Gal8}
Пусть $K$ --- конечное расширение Галуа поля $k$, $G$ --- его группа Галуа. Тогда между множеством подполей $E$ в $K$, содержащих $k$, и множеством подгрупп $H$ в $G$ существует биективное соответствие, задаваемое $E=K^H$. Поле $E$ будет расширением Галуа поля $k$ тогда и только тогда, когда $H$ --- нормальная подгруппа в $G$. Более того, $G(E/k)\cong G/H$.
\end{theorem}

Пусть $k$ --- поле, $f(x)\in k[x]$. Пусть $K$ --- поле разложения многочлена $f(x)$, и $G$ --- группа Галуа поля $K$ над $k$. Тогда $G$ называется \emph{группой Галуа многочлена} $f(x)$. Элементы из $G$ переставляют корни многочлена $f(x)$. Таким образом, мы имеем инъективный гомоморфизм группы $G$ в группу $S_n$.

\begin{example}
Пусть $k$ --- поле и $char(k)\neq 2$, $a$ не является квадратом в $k$. Тогда многочлен $f(x)=x^2-a$ неприводим. Поскольку $char(k)\neq 2$, то $f(x)$ сепарабелен. Его группа Галуа --- циклическая группа порядка два.
\end{example}

\begin{example}
Пусть $k$ --- поле и $char(k)\neq 2,3$. Пусть $f(x)$ --- неприводимый кубический многочлен, $G$ --- его группа Галуа. Если $\alpha$ --- корень многочлена $f(x)$. Тогда $[k(\alpha):k]=3$. Поскольку $G$ --- подгруппа $S_3$, то $G$ либо $\ZZ_3$, либо $S_3$. Пусть $\alpha_1,\alpha_2,\alpha_3$ --- различные корни $f(x)$. Рассмотрим $$\delta=(\alpha_1-\alpha_2)(\alpha_1-\alpha_3)(\alpha_2-\alpha_3),\quad \Delta=\delta^2.$$ Пусть $\sigma\in G$. Заметим, что $\sigma \delta=\pm\delta$, $\sigma\Delta=\Delta$. Следовательно, $\Delta\in k$. Заметим, что множество $\sigma$, которые оставляют $\delta$ на месте, это в точности четные перестановки. Таким образом, $G=S_3$ тогда и только тогда, когда $\delta\not\in k$, т.е. $\Delta$ не является квадратом.
\end{example}

Пусть $k$ --- поле. Элемент $\zeta\in k$ называется \emph{корнем из единицы} степени $n$, если $\zeta^n=1$.

\begin{remark}
Пусть $k$ --- поле характеристики $p$. Тогда уравнение $x^{p^m}=1$ имеет только один корень, а именно $1$. Следовательно, в поле характеристики $p$ нет корней $p^m$-й степени из $1$, кроме $1$.
\end{remark}

Пусть $n$ --- натуральное число, взаимно простое с характеристикой поля $k$. Тогда многочлен $x^n-1$ имеет $n$ различных корней. Действительно, его производная равна $nx^{n-1}$, и обращается в ноль только при $x=0$. Таким образом, $x^n-1$ не имеет кратных корней. Следовательно, в $\bar{k}$ существуют $n$ различных корней $n$-й степени из единицы. Они образуют циклическую группу. Образующие этой группы называются \emph{примитивными} или \emph{первообразными} корнями $n$-й степени из единицы.

\begin{lemma}[лемма Гаусса]
\label{Gaus}
Пусть $f(x)$ и $g(x)$ --- многочлены с целыми коэффициентами. Пусть $a$ --- наибольший общий делитель коэффициентов многочлена $f(x)$, $b$ --- наибольший общий делитель коэффициентов многочлена $g(x)$, $c$ --- наибольший общий делитель коэффициентов многочлена $f(x)g(x)$. Тогда $c=ba$.
\end{lemma}

\begin{proof}
Достаточно доказать, что если $a=b=1$, то $c=1$. Предположим, что $c$ делится на простое число $p$. Пусть $$f(x)=a_n x^n+ a_{n-1}x^{n-1}+\cdots+a_1 x+ a_0,$$ $$g(x)=b_m x^m+ b_{m-1}x^{m-1}+\cdots+b_1 x+ b_0.$$ Пусть $r$ --- наименьшее число такое, что $a_r$ не делится на $p$, $s$ --- наименьшее число такое, что $b_s$ не делится на $p$. Рассмотрим коэффициент при $x^{r+s}$ в $f(x)g(x)$. Он равен $$c_{r+s}=a_rb_s+a_{r+1}b_{s-1}+a_{r+2}b_{s-2}+\cdots+a_{r-1}b_{s+1}+a_{r-2}b_{s+2}+\cdots.$$ Заметим, что все слагаемые, кроме $a_rb_s$ делятся на $p$, а $a_rb_s$ не делится на $p$. Тогда $c_{r+s}$ также не делится на $p$.
\end{proof}

\begin{theorem}
\label{Edin1}
Пусть $\zeta$ --- примитивный корень $n$-й степени из единицы. Тогда $[\QQ(\zeta):\QQ]=\varphi(n)$, где $\varphi(n)$ --- функция Эйлера.
\end{theorem}

\begin{proof}
Пусть $f(x)$ --- минимальный многочлен элемента $\zeta$ над $\QQ$. Тогда $f(x)$ делит $x^n-1$, т.е. $x^n-1=f(x)g(x)$. Из леммы Гаусса следует, что $f(x)$ и $g(x)$ --- многочлены с целыми коэффициентами. Пусть $p$ --- простое число, не делящее $n$. Тогда $\zeta^p$ --- примитивный корень $n$-й степени из единицы. Более того, все примитивные корни $n$-й степени из единицы могут быть получены последовательным возведением $\zeta$ в простые степени с показателями, не делящими $n$. Докажем, что $\zeta^p$ --- корень многочлена $f(x)$. Предположим, что $\zeta^p$ не является корнем многочлена $f(x)$. Тогда $\zeta^p$ --- корень многочлена $g(x)$. Тогда $\zeta$ --- корень многочлена $g(x^p)$. Следовательно, $g(x^p)$ делится на $f(x)$, т.е. $g(x^p)=f(x)h(x)$. Заметим, что $h(x)$ имеет целые коэффициенты. Поскольку $a^p=a\quad(\mod p)$, то $g(x^p)=(g(x))^p \quad(\mod p)$. Отсюда, $$(g(x))^p=f(x)h(x)\quad(\mod p).$$ Тогда многочлены $\bar{f}$ и $\bar{g}$ над $\ZZ_p$, полученные редукцией по модулю $p$, не являются взаимно простыми. Следовательно, многочлен $x^n-1$ имеет кратные корни в расширении $\ZZ_p$. С другой стороны, его производная не равна нулю в поле характеристики $p$. Противоречие. Таким образом, $\zeta^p$ --- корень многочлена $f(x)$. Следовательно, все примитивные корни $n$-й степени из единицы являются корнями $f(x)$. Тогда степень $f(x)$ не меньше $\varphi(n)$, а, следовательно, равна $\varphi(n)$.
\end{proof}

\begin{theorem}
\label{Edin2}
Пусть $\zeta$ --- примитивный корень $n$-й степени из единицы. Тогда $G(\QQ(\zeta)/\QQ)=U(n)$, где $U(n)$ --- группа единиц кольца $\ZZ_n$.
\end{theorem}

\begin{remark}
Группа единиц кольца $\ZZ_n$ это в точности множество обратимых элементов в $\ZZ_n$. Поскольку элемент $k\in\ZZ_n$ обратим тогда и только тогда, когда $(k,n)=1$, то порядок группы $U(n)$ равен $\varphi(n)$.
\end{remark}

\begin{proof}
Пусть $\sigma\in G$ и $\zeta$ --- примитивный корень из единицы степени $n$. Тогда $\sigma(\zeta)$ определяет автоморфизм $\sigma$. Заметим, что $\sigma(\zeta)=\zeta^k$ --- примитивный корень из единицы степени $n$. Тогда $(k,n)=1$. Определим $\psi\colon G\rightarrow U(n)$, как $\psi(\sigma)=k$. Если $\sigma'(\zeta)=\zeta^{k'}$, то $$\sigma'(\sigma(\zeta))=\sigma'(\zeta^k)=(\zeta^{k'})^k=\zeta^{kk'}.$$ Таким образом, $\psi$ --- гомоморфизм. Если $\psi(\sigma)=1$, то $\sigma(\zeta)=\zeta$ и, следовательно, $\sigma$ --- тождественное отображение. Таким образом, $\psi$ --- инъективный гомоморфизм. Поскольку $|G(\QQ(\zeta)/\QQ)|=|U(n)|=\varphi(n)$, то $\psi$ --- изоморфизм.
\end{proof}

\begin{theorem}
\label{Edin3}
Пусть $k$ --- поле, содержащее примитивный корень $n$-й степени из единицы, и $n$ взаимно просто с характеристикой поля. Пусть $a\in k$. Пусть $\alpha$ --- корень многочлена $x^n-a$. Тогда $k(\alpha)$ --- циклическое расширение степени $d$ и $\alpha^d\in k$. В частности, если $x^n-a$ неприводим, то $G(k(\alpha)/k)$ --- циклическая группа порядка $n$.
\end{theorem}

\begin{proof}
Пусть $\zeta$ --- примитивный корень $n$-й степени из единицы. Заметим, что корни $x^n-a$ есть $\alpha\zeta^k$. Поскольку они все принадлежат $k(\alpha)$, то $k(\alpha)$ --- нормальное расширение $k$. Более того, $\sigma(\alpha)=\zeta^k\alpha$. Определим $\psi\colon G\rightarrow \ZZ_n$, как $\psi(\sigma)=k$. Пусть $\sigma'(\alpha)=\alpha\zeta^{k'}$. Поскольку все элементы группы Галуа оставляют поле $k$ на месте, а, следовательно и все корни из единицы, то $$\sigma'(\sigma(\alpha))=\sigma'(\alpha\zeta^k)=\alpha\zeta^{k'}\zeta^k=\alpha\zeta^{k+k'}.$$ Таким образом, $\psi$ --- гомоморфизм.
Если $\psi(\sigma)=0$, то $\sigma(\alpha)=\alpha$ и, следовательно, $\sigma$ --- тождественное отображение. Таким образом, $\psi$ --- инъективный гомоморфизм. Следовательно, $G(k(\alpha)/k)$ --- циклическая группа, порядок который делит $n$. Пусть $|G(k(\alpha)/k)|=d$, и $\sigma$ --- порождающий элемент группы $G(k(\alpha)/k)$. Пусть $\sigma(\alpha)=\zeta^k\alpha$. Тогда $k$ имеет порядок $d$ в $\ZZ_n$. Таким образом, $$\sigma(\alpha^d)=(\zeta^k\alpha)^d=\zeta^{kd}\alpha^d=\alpha^d.$$ Поскольку $\alpha^d$ --- неподвижный элемент, то $\alpha^d\in k$. Если $x^n-a$ неприводим, то $[k(\alpha):k]=n$. Следовательно, $|G(k(\alpha)/k)|=n$. Отсюда, $G(k(\alpha)/k)=\ZZ_n$.
\end{proof}

\begin{corollary}
\label{Edin4}
Пусть $k$ --- поле, содержащее примитивный корень $q$-й степени из единицы, где $q$ --- простое число, взаимно простое с характеристикой поля. Пусть $a\in k$. Тогда многочлен $x^q-a$ либо неприводим, либо раскладывается на линейные множители.
\end{corollary}

\section{Норма и след}

Пусть $K$ --- конечное сепарабельное расширение поля $k$, $[K:k]=n$. Пусть $\sigma_1,\sigma_2,\ldots,\sigma_n$ --- различные вложения $K$ в алгебраическое замыкание $\bar{k}$ поля $k$. Пусть $\alpha\in K$. Тогда определим \emph{норму} $\alpha$ формулой $$N_k^K(\alpha)=\prod\limits_{i=1}^n \sigma_i(\alpha).$$ Аналогично определим \emph{след} $\alpha$ формулой $$\Tr_k^K(\alpha)=\sum\limits_{i=1}^n \sigma_i(\alpha).$$

\begin{theorem}
\label{Norm1}
Норма является мультипликативным гомоморфизмом $K^*$ в $k^*$. След является аддитивным гомоморфизмом $K$ в $k$.
\end{theorem}

\begin{proof}
Заметим, что любой автоморфизм $\sigma$ оставляет норму и след на месте. Следовательно, $N_k^K(\alpha)\in k^*$, если $\alpha\neq 0$, и $\Tr_k^K(\alpha)\in k$. Очевидно, что $$N_k^K(\alpha_1\alpha_2)=\prod\limits_{i=1}^n \sigma_i(\alpha_1\alpha_2)=\prod\limits_{i=1}^n \sigma_i(\alpha_1)\sigma_i(\alpha_2)=N_k^K(\alpha_1)N_k^K(\alpha_2).$$ Аналогично, $$\Tr_k^K(\alpha_1+\alpha_2)=\sum\limits_{i=1}^n \sigma_i(\alpha_1+\alpha_2)=\sum\limits_{i=1}^n \sigma_i(\alpha_1)+\sigma_i(\alpha_2)=\Tr_k^K(\alpha_1)+\Tr_k^K(\alpha_2).$$
\end{proof}

\begin{theorem}
\label{Norm2}
Пусть $F$ --- конечное сепарабельное расширение поля $k$, $E$ --- конечное сепарабельное расширение поля $F$. Тогда $$N_k^E=N_k^F\circ N_F^E,\quad\Tr_k^E=\Tr_k^F\circ\Tr_F^E.$$
\end{theorem}

\begin{proof}
Пусть $\{\tau_i\}$ --- семейство вложений $F$  в $\bar{k}$ над $k$. Продолжим каждое $\tau_i$ до вложения $E$ в $\bar{k}$ (будем обозначать это продолжение также через $\tau_i$). Пусть $\{\sigma_i\}$ --- семейство вложений $E$  в $\bar{k}$ над $F$. Пусть $\sigma$ --- вложение $E$ в $\bar{k}$ над $k$. Поскольку ограничение $\sigma$ на $F$ совпадает с некоторым $\tau_j$, то $\tau_j^{-1}\sigma$ оставляет $F$ неподвижным. Таким образом, существует $\sigma_i$ такое, что $\tau_j^{-1}\sigma=\sigma_i$. Отсюда, $\sigma=\tau_j\sigma_i$. Следовательно, семейство $\{\tau_j\sigma_i\}$ задает все различные вложения $E$ в $\bar{k}$ над $k$. Отсюда, $$N_k^E(\alpha)=\prod\limits_{i,j} \tau_j\sigma_i(\alpha)=\prod\limits_{j} \tau_j\left(\prod\limits_{i}\sigma_i(\alpha)\right)=N_k^F(N_F^E(\alpha)),$$
$$\Tr_k^E(\alpha)=\sum\limits_{i,j} \tau_j\sigma_i(\alpha)=\sum\limits_{j} \tau_j\left(\sum\limits_{i}\sigma_i(\alpha)\right)=\Tr_k^F(\Tr_F^E(\alpha)).$$
\end{proof}

\begin{theorem}
\label{Norm3}
Пусть $K=k(\alpha)$ и $f(x)=x^n+a_{n-1}x^{n-1}+\cdots+a_1 x+a_0$ --- минимальный многочлен элемента $\alpha$. Тогда $$N_k^K(\alpha)=(-1)^n a_0,\quad \Tr_k^K(\alpha)=-a_{n-1}.$$
\end{theorem}

\begin{proof}
Пусть $$f(x)=(x-\alpha_1)(x-\alpha_2)\cdots(x-\alpha_n)$$ в $\bar{k}$. Тогда вложение $\sigma\colon k(\alpha)\rightarrow\bar{k}$ задается $\sigma(\alpha)=\alpha_i$. Таким образом, $$N_k^K(\alpha)=\prod\limits_{i=1}^n \sigma_i(\alpha)=\prod\limits_{i=1}^n\alpha_i=(-1)^n a_0.$$ Аналогично, $$\Tr_k^K(\alpha)=\sum\limits_{i=1}^n \sigma_i(\alpha)=\sum\limits_{i=1}^n\alpha_i=-a_{n-1}.$$
\end{proof}

Теперь нам понадобятся некоторые факты о характерах группы.

\begin{definition}
Пусть $G$ --- группа и $K$ --- поле. \emph{Характером} группы $G$ в $K$ называется гомоморфизм $\chi\colon G\rightarrow K^*$. Характеры $\chi_1,\chi_2,\ldots,\chi_n$ называются \emph{линейно независимыми} над $K$, если равенство $$a_1\chi_1+a_2\chi_2+\cdots+a_n\chi_n=0$$ выполнено тогда и только тогда, когда все $a_i=0$. Здесь все $a_i\in K$, равенство нулю $a_1\chi_1+a_2\chi_2+\cdots+a_n\chi_n$ понимается, как тождественное равенство, т.е. $$f(g)=a_1\chi_1(g)+a_2\chi_2(g)+\cdots+a_n\chi_n(g)=0,$$ для любого $g\in G$.
\end{definition}

\begin{theorem}
\label{Har1}
Пусть $\chi_1,\chi_2,\ldots,\chi_n$ --- различные характеры $G$ в $K$. Тогда они линейно независимы над $K$.
\end{theorem}

\begin{proof}
Докажем по индукции. Один характер, очевидно, линейно независим. Предположим, что мы доказали для $n-1$ характера. Предположим, что выполнено $$a_1\chi_1+a_2\chi_2+\cdots+a_n\chi_n=0$$ и все $a_i\neq 0$. Поскольку характеры $\chi_1$ и $\chi_2$ различны, то существует $y\in G$ такой, что $\chi_1(y)\neq\chi_2(y)$. Для всех $g\in G$ имеем $$a_1\chi_1(yg)+a_2\chi_2(yg)+\cdots+a_n\chi_n(yg)=0.$$ Отсюда, $$a_1\chi_1(y)\chi_1+a_2\chi_2(y)\chi_2+\cdots+a_n\chi_n(y)\chi_n=0.$$ Разделим это равенство на $\chi_1(y)$ и вычтем из $$a_1\chi_1+a_2\chi_2+\cdots+a_n\chi_n=0.$$ Получаем $$\left(a_2-a_2\frac{\chi_2(y)}{\chi_1(y)}\right)\chi_2+\cdots+\left(a_n-a_n\frac{\chi_n(y)}{\chi_1(y)}\right)\chi_n=0.$$ Первый коэффициент отличен от нуля. Таким образом мы получили, что характеры $\chi_1,\chi_2,\ldots,\chi_{n-1}$ линейно зависимы. Противоречие.
\end{proof}

\begin{theorem}[теорема Гильберта 90]
\label{NorGil}
Пусть $K$ --- циклическое расширение поля $k$ с группой Галуа $G$. Пусть $\sigma$ --- образующая этой группы, и $\beta\in K$. Норма $N_k^K(\beta)=1$ тогда и только тогда, когда существует $\alpha\neq 0$ в $K$, такой, что $\beta=\frac{\alpha}{\sigma \alpha}$.
\end{theorem}

\begin{proof}
Предположим, что такой элемент существует. Тогда $N(\beta)=\frac{N(\alpha)}{N(\sigma\alpha)}$. Поскольку норма --- это произведение по всем автоморфизмам из $G$, то применение $\sigma$ лишь переставляет эти автоморфизмы. Следовательно, $N(\sigma\alpha)=N(\alpha)$. Тогда $N(\beta)=1$.

Предположим, что $N(\beta)=1$. Согласно теореме \ref{Har1} отображение $$id+\beta\sigma+\beta\sigma(\beta)\sigma^2+\cdots+\beta\sigma(\beta)\cdots\sigma^{n-2}(\beta)\sigma^{n-1}$$ не равно тождественно нулю (здесь $id$ --- тождественное отображение). Тогда существует $\gamma\in K$ такое, что $$\alpha=\gamma+\beta\sigma(\gamma)+\beta\sigma(\beta)\sigma^2(\gamma)+\cdots+\beta\sigma(\beta)\cdots\sigma^{n-2}(\beta)\sigma^{n-1}(\gamma)$$ не равен нулю. Применим $\beta\sigma$ к $\alpha$. Получаем $$\beta\sigma(\alpha)=\beta\sigma(\gamma)+\beta\sigma(\beta)\sigma^2(\gamma)+\cdots+\beta\sigma(\beta)\cdots\sigma^{n-2}(\beta)\sigma^{n-1}(\beta)\sigma^{n}(\gamma).$$ Заметим, что $$\beta\sigma(\beta)\cdots\sigma^{n-2}(\beta)\sigma^{n-1}(\beta)=N(\beta)=1$$ и $\sigma^{n}(\gamma)=\gamma$. Таким образом, $\beta\sigma(\alpha)=\alpha$. Отсюда, $\beta=\frac{\alpha}{\sigma \alpha}$.
\end{proof}

\begin{theorem}
\label{Norm4}
Пусть $k$ --- поле, содержащее примитивный корень $n$-й степени из единицы, и $n$ взаимно просто с характеристикой поля. Пусть $K$ --- циклическое расширение степени $n$. Тогда существуют $\alpha\in K$, $a\in k$ такие, что $K=k(\alpha)$ и $\alpha$ --- корень уравнения $x^n-a=0$.
\end{theorem}

\begin{proof}
Пусть $\zeta$ --- примитивный корень из единицы и $K$ --- циклическое расширение поля $k$. Пусть $\sigma$ --- образующая группы $G$. Заметим, что $N(\zeta^{-1})=(\zeta^{-1})^n=1$. Согласно теореме Гильберта 90, существует $\alpha\in K$ такой, что $\sigma\alpha=\zeta\alpha$. Поскольку $\zeta\in k$, то $\sigma^k\alpha=\zeta^k\alpha$. Следовательно, $[k(\alpha):k]\geq n$. Поскольку $[K:k]\geq n$, то $K=k(\alpha)$. Более того, $$\sigma(\alpha^n)=(\sigma(\alpha))^n=(\zeta\alpha)^n=\alpha^n.$$ Отсюда, $\alpha^n\in k$.
\end{proof}

Теперь рассмотрим другой вариант теоремы Гильберта 90.

\begin{theorem}[теорема Гильберта 90]
\label{NorGil2}
Пусть $K$ --- циклическое расширение поля $k$ с группой Галуа $G$. Пусть $\sigma$ --- образующая этой группы, и $\beta\in K$. След $\Tr_k^K(\beta)=0$ тогда и только тогда, когда существует $\alpha\in K$, такой, что $\beta=\alpha-\sigma \alpha$.
\end{theorem}

\begin{proof}
Предположим, что такой элемент существует. Тогда $\Tr_k^K(\beta)=\Tr_k^K(\alpha-\sigma \alpha)$. Поскольку след --- это сумма по всем автоморфизмам из $G$, то применение $\sigma$ лишь переставляет эти автоморфизмы. Следовательно, $\Tr(\sigma\alpha)=\Tr(\alpha)$. Тогда $\Tr(\beta)=0$.

Предположим, что $\Tr(\beta)=0$. Заметим, что существует $\gamma\in K$ такое, что $\Tr(\gamma)\neq 0$. Положим
 $$\alpha=\frac{1}{\Tr(\gamma)}(\beta\sigma(\gamma)+(\beta+\sigma(\beta))\sigma^2(\gamma)+\cdots+(\beta+\sigma(\beta)+\cdots+\sigma^{n-2}(\beta))\sigma^{n-1}(\gamma)).$$ Тогда $\beta=\alpha-\sigma\alpha$.
\end{proof}

\section{Резольвента}

\begin{definition}
Пусть $k$ --- поле, содержащее корни $n$-й степени из единицы. Предположим, что $char(k)=0$. Пусть $K$ --- циклическое расширение поля $k$, и $\sigma$ --- порождает группу Галуа $G(K/k)$. Пусть $x\in K$, $\zeta$ --- корень $n$-й степени из единицы. Тогда выражение $$(\zeta,x)=x+\zeta\sigma(x)+\zeta^2\sigma^2(x)+\cdots+\zeta^{n-1}\sigma^{n-1}(x)$$ называется \emph{резольвентой Лагранжа}.
\end{definition}

\begin{claim}
\label{Res1}
\begin{enumerate}
\item $\sigma((\zeta,x))=\zeta^{-1}(\zeta,x)$;
\item $\sigma((1,x))=\Tr(x)\in k$;
\item $(\zeta,x)^n\in k$;
\item $(\zeta,x)(\zeta^{-1},x)\in k$.
\end{enumerate}
\end{claim}

\begin{proof}
Первые два свойства очевидны. Докажем (3). Получаем $$\sigma((\zeta,x)^n)=\sigma^n((\zeta,x))=(\zeta^{-1}(\zeta,x))^n=(\zeta,x)^n.$$ Следовательно, $(\zeta,x)^n$ --- неподвижный элемент, а значит $(\zeta,x)^n\in k$. Докажем (4). Получаем $$\sigma((\zeta,x)(\zeta^{-1},x))=\sigma((\zeta,x))\sigma((\zeta^{-1},x))=\zeta^{-1}(\zeta,x)\zeta(\zeta,x)=(\zeta,x).$$ Следовательно, $(\zeta,x)(\zeta^{-1},x)$ --- неподвижный элемент, а значит $(\zeta,x)(\zeta^{-1},x)\in k$.
\end{proof}

Рассмотрим кубическое уравнение $x^3+px+q=0$. Пусть $k=\QQ(p,q,j)$, где $j$ --- кубический корень из единицы. Заметим, что любое кубическое уравнение приводится к такому виду. Пусть $\alpha,\beta,\gamma$ --- корни этого уравнения. Пусть $d=(\alpha-\beta)(\beta-\gamma)(\gamma-\alpha)$. Заметим, что $k(\alpha,\beta,\gamma)=k(\alpha,d)$ и $d^2=-4p^3-27q^2$. Предположим, что группа Галуа этого уравнения есть $S_3$. Пусть $K$ --- поле разложение многочлена $x^3+px+q$, и $E$ --- неподвижное поле группы $A_3$. Тогда $E$ --- расширение степени два над $k$, т.е. $E=k(s)$. Пусть $\sigma$ --- порождающий элемент группы $A_3$. Мы можем считать, что $\sigma=(\alpha\beta\gamma)$, т.е. $\sigma$ переставляет по кругу корни уравнения $x^3+px+q=0$. Поскольку $K$ --- нормальное расширение поля $k$, то $K$ --- нормальное расширение поля $E$. Следовательно, $K$ --- расширение Галуа поля $E$. Заметим, что $G(K/E)=A_3$. Таким образом, $K$ --- циклическое расширение поля $E$. Согласно теореме \ref{Norm4} существует элемент $s'\in K$ такой, что $s'^3\in E$ и $K=E(s')$. Найдем $s$ и $s'$. Заметим, что $\sigma d=d$. Таким образом, $d\in E$. Поскольку транспозиция $(\alpha\beta)$ переводит $d$ в $-d$, то $d\not\in K$. Рассмотрим автоморфизм $\tau=id+j\sigma+j^2\sigma^2$. Согласно теореме \ref{Har1} $\tau$ ненулевой. Более того, $\tau(\alpha)\neq 0$. Действительно, если $\tau(\alpha)=0$, то $\tau(\beta)=\tau(\sigma(\alpha))=j^2\tau(\alpha)=0$. Аналогично, $\tau(\gamma)=0$. Следовательно, $q\tau(1)=\tau(q)=\tau(\alpha\beta\gamma)=0$ и $\tau=0$. Рассмотрим $(j,\alpha)=\alpha+j\beta+j^2\gamma=\tau(\alpha).$ Поскольку $\sigma((j,\alpha))=j^2(j,\alpha)\neq (j,\alpha)$, то $(j,\alpha)\not\in E$. Согласно \ref{Res1} $(j,\alpha)^3\in E$. Отсюда, $K=L((j,\alpha))$.

\section{Нормальный базис}

Пусть $A$ --- абелева группа, $k$ --- поле и $\lambda_1,\lambda_2,\cdots,\lambda_n\colon A\rightarrow k$ --- аддитивные гомоморфизмы. Будем говорить, что $\lambda_1,\lambda_2,\cdots,\lambda_n$ \emph{алгебраически зависимы}, если существует многочлен $f(x_1,x_2,\ldots,x_n)$ над $k$ такой, что $$f(\lambda_1(a),\lambda_2(a),\ldots,\lambda_n(a))=0$$ для всех $a\in A$.  Многочлен $f(x_1,x_2,\ldots,x_n)$ называется \emph{аддитивным}, если он индуцирует аддитивный гомоморфизм $k^n$ в $k$.

\begin{theorem}
\label{NormBas1}
$\lambda_1,\lambda_2,\cdots,\lambda_n\colon A\rightarrow k$ --- аддитивные гомоморфизмы абелевой группы $A$ в поле $k$. Если эти гомоморфизмы алгебраически зависимы, то существует аддитивный многочлен $f(x_1,x_2,\ldots,x_n)$ над $k$ такой, что $$f(\lambda_1(a),\lambda_2(a),\ldots,\lambda_n(a))=0$$ для всех $a\in A$.
\end{theorem}

\begin{proof}
Мы докажем эту теорему для случая бесконечного поля. Пусть $f(x_1,x_2,\ldots,x_n)$ --- многочлен наименьшей возможной степени такой, что $$f(\lambda_1(a),\lambda_2(a),\ldots,\lambda_n(a))=0$$ для всех $a\in A$. Пусть $X=(x_1,x_2,\ldots,x_n)$, $Y=(y_1,y_2,\ldots,y_n)$, $\Lambda=(\lambda_1,\lambda_2,\ldots,\lambda_n)$. Рассмотрим $g(X,Y)=f(X+Y)-f(X)-f(Y).$ Заметим, что $$g(\Lambda(a),\Lambda(b))=f(\Lambda(a+b))-f(\Lambda(a))-f(\Lambda(b))=0$$ для любых $a,b\in A$. Нам нужно доказать, что $g$ --- нулевой многочлен. Заметим, что степень $g(X,Y)$ по $X$ строго меньше степени $f(X)$. Аналогично по $Y$. Предположим, что $g$ не равен тождественно нулю. Рассмотрим два случая.
\begin{case}
Имеем $g(\xi,\Lambda(b))=0$ для всех $\xi\in k^n$, $b\in A$. По предположению, существует $\xi\in k^n$ такой, что $g(\xi,Y)$ не равен тождественно нулю. Положим $P(Y)=g(\xi,Y)$. Тогда $$P(\lambda_1(a),\lambda_2(a),\ldots,\lambda_n(a))=0$$ для всех $a\in A$. С другой стороны, степень $P$ меньше степени $f$. Противоречие.
\end{case}
\begin{case}
Существуют  $\xi\in k^n$, $b\in A$ такие, что $g(\xi,\Lambda(b))\neq 0$. Положим $P(X)=g(X,\Lambda(b))$. Тогда $P(X)$ --- ненулевой многочлен. С другой стороны, $P(\Lambda(a))=0$ для любого $a\in A$, и степень многочлена $P$ меньше степени $f$. Противоречие.
\end{case}
Таким образом, $g$ индуцирует нулевую функцию. Поскольку поле бесконечно, то $g$ --- нулевой многочлен.
\end{proof}

\begin{claim}
\label{NormBas2}
Пусть $f(x_1,x_2,\ldots,x_n)$ --- аддитивный многочлен. Тогда $$f(x_1,x_2,\ldots,x_n)=f_1(x_1)+f_2(x_2)+\cdots+f_n(x_n),$$ где $f_i(x)$ --- аддитивные многочлены от одной переменной.
\end{claim}

\begin{proof}
Пусть $f_i(x_i)=f(0,\ldots,0,x_i,0,\ldots,0)$. Тогда $f_i(x)$ --- аддитивные многочлены от одной переменной и $$f(x_1,x_2,\ldots,x_n)=f_1(x_1)+f_2(x_2)+\cdots+f_n(x_n).$$
\end{proof}

\begin{claim}
\label{NormBas3}
Пусть $f(x)$ --- аддитивный многочлен. Тогда $$f(x)=\sum\limits_{i=1}^m a_i x^{p^i},$$ где $p$ --- характеристика поля. Если $char(k)=0$, то $f(x)=ax$.
\end{claim}

\begin{proof}
Пусть $f(x)=\sum\limits_{i=1}^n a_i x^i$. Тогда $$g(x,y)=f(x+y)-f(x)-f(y)=\sum\limits_{i=1}^n a_i ((x+y)^i-x^i-y^i)=0.$$ Пусть $a_i\neq 0$. Поскольку $(x+y)^i-x^i-y^i$ содержит $ix^{i-1}y$, то $g(x,y)$ содержит $a_i ix^{i-1}y$. С другой стороны, $g(x,y)$ --- нулевой многочлен. Следовательно, $i$ делится на характеритику поля. Пусть $i=p^ms$. Тогда $$(x+y)^i-x^i-y^i=(x^{p^m}+y^{p^m})^s-(x^{p^m})^s-(y^{p^m})^s.$$ Отсюда, рассуждая аналогично, $s=1$.
\end{proof}

\begin{theorem}
\label{NormBas4}
Пусть $K$ --- бесконечное поле и $\sigma_1,\sigma_2,\ldots,\sigma_n$ --- различные элементы конечной группы автоморфизмов поля $K$. Тогда $\sigma_1,\sigma_2,\ldots,\sigma_n$ алгебраически независимы над $K$.
\end{theorem}

\begin{proof}
В случае характеристики ноль, теорема следует из \ref{Har1}, \ref{NormBas1}, \ref{NormBas2}, \ref{NormBas3}. Пусть характеристика $p>0$, и $\sigma_1,\sigma_2,\ldots,\sigma_n$ алгебраически зависимы. Согласно теореме \ref{NormBas1} существует аддитивный многочлен $f(x_1,x_2,\ldots,x_n)$ такой, что $f\neq 0$, но $$f(\sigma_1(a),\sigma_2(a),\ldots,\sigma_n(a))=0$$ для любого $a\in K$. В силу \ref{NormBas2} и \ref{NormBas3} мы можем записать $$\sum\limits_{i=1}^n\sum\limits_{j=1}^m a_{ij}(\sigma_i(a))^{p^j}=0$$  для любого $a\in K$. Согласно \ref{Har1} гомоморфизмы $x\rightarrow (\sigma_i(x))^{p^j}$ не могут быть различными. Таким образом, существуют $i_1,i_2,j_1,j_2$ такие, что $(\sigma_{i_1}(x))^{p^{j_1}}=(\sigma_{i_2}(x))^{p^{j_2}},$ при этом либо $i_1\neq i_2$, либо $j_1\neq j_2$. Пусть $j_1\leq j_2$. Заметим, что извлечение корня $p$-й степени в поле характеристики $p$ однозначно. Тогда $\sigma_{i_1}(x)=(\sigma_{i_2}(x))^{p^{j_2-j_1}}.$ Положим $\sigma=\sigma^{-1}_{i_2}\sigma_{i_1}$. Тогда $\sigma(x)=x^{p^{j_2-j_1}}$. Поскольку $\sigma$ --- элемент конечной группы, то существует $n$ такое, что $\sigma^n=id$. Тогда $x=x^{p^{n(j_2-j_1})}$ для всех $x\in K$. Поскольку $K$ --- бесконечное поле, то $j_2=j_1$. Отсюда, $\sigma_{i_1}(x)=\sigma_{i_2}(x)$ для всех $x\in K$. Противоречие.
\end{proof}


\begin{theorem}
\label{NormBas5}
Пусть $K$ --- конечное расширение Галуа поля $k$, $G$ --- его группа Галуа. Пусть $n=|G|$, $\sigma_1,\sigma_2,\ldots,\sigma_n$ --- элементы группы $G$. Тогда существует элемент $w\in K$ такой, что $\sigma_1 w,\sigma_2 w,\ldots,\sigma_n w$ --- базис $K$ над $k$.
\end{theorem}

\begin{proof}
Мы докажем эту теорему для случая бесконечного поля. Рассмотрим множество переменных $x_1=x_{\sigma_1},\ldots,x_n=x_{\sigma_n}$. Пусть $f(x_1,x_2,\ldots,x_n)=\det(t_{ij})$, где $t_{ij}=x_{\sigma^{-1}_i\sigma_j}$. Заметим, что $f(x_1,x_2,\ldots,x_n)$ не является тождественным нулем, что видно если подставить 1 вместо $x_e$ и 0 вместо остальных $x_i$. Согласно теореме \ref{NormBas4} существует $w\in K$ такое, что $\det(\sigma^{-1}_i\sigma_j(w))\neq 0$. Докажем, что $\sigma_1 w,\sigma_2 w,\ldots,\sigma_n w$ линейно независимы. Предположим, что существуют $a_1,a_2,\ldots a_n\in k$ такие, что $$a_1\sigma_1(w)+a_2\sigma_2(w)+\cdots+a_n\sigma_n(w)=0.$$ Применим $\sigma^{-1}_i$ к этому соотношению для каждого $i=1,2,\ldots,n$. Получим систему линейных уравнений относительно переменных $a_1,a_2,\ldots a_n$, $$\begin{cases}\sigma^{-1}_1\sigma_1(w)a_1+\sigma^{-1}_1\sigma_2(w)a_2+\cdots+\sigma^{-1}_1\sigma_n(w)a_n=0\\
\sigma^{-1}_2\sigma_1(w)a_1+\sigma^{-1}_2\sigma_2(w)a_2+\cdots+\sigma^{-1}_2\sigma_n(w)a_n=0\\
\cdots\quad\cdots\quad\cdots\\
\sigma^{-1}_n\sigma_1(w)a_1+\sigma^{-1}_n\sigma_2(w)a_2+\cdots+\sigma^{-1}_n\sigma_n(w)a_n=0.\end{cases}$$ Определитель этой системы не равен нулю. Следовательно, все $a_i=0$.
\end{proof}

\section{Радикальные расширения}


\begin{theorem}[теорема Артина--Шрейера]
\label{ArtShr}
Пусть $k$ --- поле характеристики $p$. Тогда если $K$ --- циклическое расширение $k$ степени $p$, то существует такой элемент $\alpha\in K$ такой, что $K=k(\alpha)$, и $\alpha$ --- корень многочлена $x^p-x-a$, $a\in k$.
Обратно, для любого $a\in k$ многочлен $f(x)=x^p-x-a$ либо имеет корень в $k$, и тогда все его корни лежат в $k$, либо $f(x)$ неприводим. В последнем случае, если $\alpha$ --- корень $f(x)$, то $k(\alpha)$ --- циклическое расширение $k$ степени $p$.
\end{theorem}

\begin{proof}
Пусть $K$ --- циклическое расширение $k$ степени $p$. Тогда $$\Tr(-1)=(-1)+(-1)+\cdots+(-1)=-p=0.$$ Согласно теореме \ref{NorGil2} существует $\alpha\in K$ такой, что $-1=\alpha-\sigma\alpha$. Отсюда, $\sigma\alpha=\alpha+1$. Тогда $\sigma^i\alpha=\alpha+i$ при $i=1,2,\ldots,p$. Таким образом, $\sigma^i$ --- различные вложения поля $k(\alpha)$. Отсюда, $[k(\alpha):k]\geq p$. Тогда $K=k(\alpha)$. Заметим, что $$\sigma(\alpha^p-\alpha)=(\sigma(\alpha))^-\sigma(\alpha)=(\alpha+1)^p-(\alpha+1)=\alpha^p-\alpha.$$ Следовательно, $\alpha^p-\alpha\in k$. Таким образом, $\alpha$ --- корень многочлена $x^p-x-a$, $a\in k$.

Обратно. Пусть $\alpha$ --- корень многочлена $x^p-x-a$, $a\in k$. Заметим, что $$(\alpha+i)^p-(\alpha+i)-a=\alpha^p+i-\alpha-i-a=\alpha^p-\alpha-a=0,$$ где $i=1,2,\ldots,p$. Следовательно, $\alpha+i$ --- корни многочлена  $x^p-x-a$. Таким образом, $x^p-x-a$ имеет $p$ различных корней, и если один корень лежит в $k$, то все его корни лежат в $k$. Предположим, что ни один корень не лежит в $k$. Докажем, что многочлен $x^p-x-a$ неприводим. Предположим, что $$x^p-x-a=f(x)g(x),$$ где $1\leq\deg f<p$. Поскольку $$x^p-x-a=(x-\alpha)(x-\alpha-1)(x-\alpha-2)\cdots(x-\alpha-(p-1)),$$ то $$f(x)=(x-\alpha-i_1)(x-\alpha-i_2)\cdots(x-\alpha-i_d),$$ где $d=\deg f$. Коэффициент при $x^{d-1}$ будет равен $-d\alpha+j$, где $j$ --- некоторое целое число, т.е. $j\in k$. Поскольку $d\neq 0$ в $k$, то $-d\alpha+j$ не принадлежит $k$. Противоречие. Таким образом, многочлен $x^p-x-a$ неприводим. Все его корни лежат в $k(\alpha)$. Тогда $k(\alpha)$ --- нормальное расширение поля $k$. Так как $x^p-x-a$ не имеет кратных корней, то $k(\alpha)$ --- расширение Галуа поля $k$. Имеется автоморфизм $\sigma$ поля $k(\alpha)$, такой, что $\sigma\alpha=\alpha+1$. Степени $\sigma^i$ автоморфизма $\sigma$ дают $\sigma^i(\alpha)=\alpha+i$, т.е.  $\sigma^i$ различны для $i=0,1,2,\ldots,p-1$. Следовательно, группа Галуа состоит из $\sigma^i$, а потому является циклической.
\end{proof}

\begin{definition}
Пусть $K$ --- расширение поля $k$. Будем говорить, что $K$ --- \emph{разрешимо в радикалах} (\emph{радикальное расширение}), если существует башня расширений $$k=K_0\subset K_1\subset\cdots\subset K_n=K$$ такая, что каждое расширение $K_i$ над $K_{i-1}$ принадлежит одному из следующих типов
\begin{enumerate}
\item получается присоединением корня многочлена $x^n-a$, где $a\in K_{i-1}$, $n$ взаимно просто с характеристикой;
\item получается присоединением корня многочлена $x^p-x-a$, где $a\in K_{i-1}$, $p$ --- характеристика поля.
\end{enumerate}
\end{definition}

Пусть $E=k(\alpha_1,\alpha_2,\ldots,\alpha_n)$, $K$ --- расширение поля $k$. Пусть $K$ и $E$ вложены в поле $L$. Поле $$F=KE=K(\alpha_1,\alpha_2,\ldots,\alpha_n)$$ мы будем называть \emph{подъемом} $K$ над $F$.

\begin{theorem}
\label{Komp1}
Пусть $K$ --- конечное расширение Галуа поля $k$, $F$ --- произвольное расширение поля $k$, причем $K$, $F$ --- подполя некоторого поля $L$. Тогда $KF$ --- расширение Галуа поля $F$. Пусть $H$ --- группа Галуа $KF$ над $F$, $G$ --- группа Галуа $K$ над $k$. Пусть $\sigma\in H$. Тогда ограничение $\sigma$ на $K$ задает вложение $H$ в группу $G$.
\end{theorem}

\begin{proof}
Докажем, что $KF$ --- нормальное расширение поля $F$. Пусть $\sigma$ --- вложение $KF$ в $\bar{L}$ над $F$. Тогда $\sigma$ тождественно на $F$, а, следовательно, на $k$. Поскольку $K$ --- нормальное расширение $k$, то $\sigma$ --- автоморфизм $K$. Таким образом, $\sigma$ отображает $KF$ в себя. Следовательно, $KF$ --- нормальное расширение поля $F$. Пусть $K=k(\alpha_1,\alpha_2,\ldots,\alpha_n)$. Тогда все $\alpha_i$ сепарабельны. Поскольку $KF=F(\alpha_1,\alpha_2,\ldots,\alpha_n)$, то $KF$ --- сепарабельное расширение поля $F$. Таким образом, $KF$ --- расширение Галуа поля $F$. Пусть $H$ --- группа Галуа $KF$ над $F$. Пусть $\sigma\in H$. Рассмотрим ограничение $\sigma$ на поле $K$. Если $\sigma$ тождественно на $K$, то $\sigma$ тождественно на $KF$ (так как всякий элемент из $KF$ может быть представлен в виде комбинации сумм, произведений и отношений элементов из $K$ и $F$). Следовательно, ограничение $\sigma$ на $K$ задает инъективный гомоморфизм $H$ в группу $G$.
\end{proof}

\begin{theorem}
\label{Komp2}
Пусть $K_1$, $K_2$ --- конечное расширения Галуа поля $k$, $G_1$, $G_2$ --- их группы Галуа соответственно. Предположим, что $K_1$, $K_2$ --- подполя некоторого алгебраически замкнутого поля $L$. Тогда $K_1K_2$ --- расширение Галуа над $k$. Пусть $G$ --- группа Галуа поля $K_1K_2$ над $k$. Тогда ограничение $\sigma\in G$ на $K_1$ и $K_2$ задает инъективный гомоморфизм групп $G\rightarrow G_1\times G_2$ посредством $\sigma\rightarrow(\sigma|_{K_1},\sigma|_{K_2})$.
\end{theorem}

\begin{proof}
Согласно теореме \ref{Komp1} $K_1K_2$ --- расширение Галуа поля $K_2$ (т.е. это расширение нормально и сепарабельно. Поскольку $K_2$ --- сепарабельное расширение поля $k$, то согласно теореме \ref{Sep7} $K_1K_2$ --- сепарабельное расширение поля $k$. Пусть $\sigma$ --- вложение поля $K_1K_2$ в поле $L$ над $k$. Тогда ограничения $\sigma$ на $K_1$ и $K_2$ оставляют эти поля на месте. Следовательно, $\sigma$ --- автоморфизм поля $K_1K_2$ над $k$. Таким образом, $K_1K_2$ --- нормально расширение поля $k$. Следовательно, $K_1K_2$ --- расширение Галуа поля $k$. Отображение $G\rightarrow G_1\times G_2$ посредством ограничений $\sigma\rightarrow(\sigma|_{K_1},\sigma|_{K_2})$ является гомоморфизмом групп. Если $\sigma$ тождественен на $K_1$  $K_2$, то он очевидно тождественен на $K_1K_2$. Таким образом, наше отображение инъективно.
\end{proof}


\begin{corollary}
\label{Komp3}
В условиях теоремы \ref{Komp2} предположим, что $K_1\cap K_2=k$. Тогда отображение $\sigma\rightarrow(\sigma|_{K_1},\sigma|_{K_2})$ задает изоморфизм $G\cong G_1\times G_2$.
\end{corollary}

\begin{proof}
Предположим, что $K_1\cap K_2=k$. Пусть $\sigma_1\in G_1$. Тогда $\sigma_1$ продолжается на $K_1K_2$ над $K_2$. Таким образом, мы получили, что подгруппа $G_1\times\{e\}$ лежит в образе нашего гомоморфизма. Аналогично, $\{e\}\times G_2$ лежит в образе нашего гомоморфизма. Следовательно, отображение $\sigma\rightarrow(\sigma|_{K_1},\sigma|_{K_2})$ задает изоморфизм $G\cong G_1\times G_2$.
\end{proof}

\begin{corollary}
\label{Komp4}
Пусть $K_1,K_2,\ldots,K_n$ --- конечные расширения Галуа поля $k$ с группами Галуа $G_1,G_2,\ldots,G_n$ соответственно. Предположим, что $$K_i\cap (K_1K_2\cdots K_{i-1}K_{i+1}\cdots K_n)=k$$ для любого $i$. Тогда группа Галуа $K_1K_2\cdots K_n$ над $k$ изоморфна $G_1\times G_2\times\cdots\times G_n$.
\end{corollary}

\begin{corollary}
\label{Komp5}
Пусть $K$ --- конечное расширения Галуа поля $k$ с группой Галуа $G$. Предположим, что $G=G_1\times G_2$. Пусть $K_1$ --- неподвижное поле группы $G_1\times \{e\}$, $K_2$ --- неподвижное поле группы $\{e\}\times G_2$. Тогда $K_1,K_2$ --- конечные расширения Галуа над $k$, $K_1\cap K_2=k$. Более того, $K=K_1K_2$.
\end{corollary}

\begin{proof}
Поскольку подгруппы $G_1\times \{e\}$ и $\{e\}\times G_2$ нормальны в $G$, то $K_1,K_2$ --- конечные расширения Галуа над $k$. Пусть $\alpha\in K_1\cap K_2$ и $\sigma\in G$. Тогда $\sigma=(\sigma_1,\sigma_2)$, $\sigma_1\in G_1$, $\sigma_2\in G_2$. Тогда $$\sigma(\alpha)=(\sigma_1,\sigma_2)(\alpha)=(\sigma_1,e)((e,\sigma_2)(\alpha))=\alpha.$$ Следовательно, $\alpha\in k$. Заметим, что $K_1K_2\subset K$. С другой стороны, согласно \ref{Komp2} и \ref{Komp3}, $K_1K_2$ --- конечное расширение Галуа над $k$ с группой Галуа $G\cong G_1\times G_2$. Следовательно, $K=K_1K_2$.
\end{proof}

\begin{corollary}
\label{Komp6}
Пусть $K$ --- конечное расширения Галуа поля $k$ с группой Галуа $G$. Предположим, что $G=G_1\times G_2\times\cdots\times G_n$. Пусть $K_i$ --- неподвижное поле группы $$G_1\times G_2\times\cdots G_{i-1}\times\{e\}\times G_{i+1}\times\cdots\times G_n.$$ Тогда $K_i$ --- конечное расширение Галуа над $k$, $$K_i\cap (K_1K_2\cdots K_{i-1}K_{i+1}\cdots K_n)=k$$ для любого $i$. Более того, $K=K_1K_2\cdots K_n$.
\end{corollary}

\begin{definition}
Пусть $K$ --- конечное сепарабельное расширение поля $k$. Пусть $E$ --- наименьшее расширение Галуа поля $k$, которое содержит $K$. Будем говорить, что расширение $K$ над $k$ \emph{разрешимо}, если группа Галуа $G(E/k)$ разрешима.
\end{definition}

\begin{theorem}
\label{Razr1}
Пусть $E$ --- разрешимое расширение поля $k$, $F$ --- любое расширение поля $k$. Причем $E$ и $F$ содержаться в некотором алгебраически замкнутом поле. Тогда $EF$ --- разрешимое расширение поля $F$.
\end{theorem}

\begin{proof}
Пусть $K$ --- разрешимое расширение Галуа поля $k$ и $E\subset K$. Тогда $KF$ --- разрешимое расширение Галуа поля $F$ и группа $G(KF/F)$ --- подгруппа группы $G(K/k)$ (см. \ref{Komp1}). Поскольку $G(K/k)$ разрешима, то $G(KF/F)$ разрешима. Так как $EF\subset KF$, то $EF$ --- разрешимое расширение поля $F$.
\end{proof}

\begin{theorem}
\label{Razr2}
Пусть $F$ --- расширение поля $k$, $E$ --- расширение поля $F$. Тогда $E$ --- разрешимое расширение поля $k$ тогда и только тогда, когда $F$ --- разрешимое расширение поля $k$ и $E$ --- разрешимое расширение поля $F$.
\end{theorem}

\begin{proof}
Пусть $F$ --- разрешимое расширение поля $k$ и $E$ --- разрешимое расширение поля $F$. Пусть $K$ --- конечное разрешимое расширение Галуа поля $k$, содержащее $F$. Согласно теореме \ref{Razr1} $EK$ --- разрешимо над $K$. Пусть $L$ --- разрешимое расширение Галуа поля $K$, содержащее $EK$. Пусть $\sigma$ --- вложение $L$ над $k$ в алгебраически замкнутое поле. Заметим, что $\sigma K=K$. Пусть $M$ --- минимальное поле, содержащее все $\sigma L$, т.е. $$M=\sigma_1 L\sigma_2 L\cdots\sigma_n L.$$ Заметим, что $M$ --- расширение Галуа поля $k$ (оно сепарабельно, поскольку $L$ сепарабельно над $k$, и нормально, поскольку любое вложение $M$ над $k$ переставляет $\sigma_i L$). В силу теоремы \ref{Komp2} группа Галуа поля $M$ над $K$ является подгруппой произведения $\prod\limits_{\sigma} G(\sigma L/K)$. Следовательно, она разрешима. Согласно теореме \ref{Gal8} имеет место сюръективный гомоморфизм $G(M/k)\rightarrow G(K/k)$. Отсюда, $G(M/k)$ содержит разрешимую нормальную подгруппу, факторгруппа по которой разрешима. Следовательно, $G(M/k)$ разрешима. Поскольку $E\subset M$, то $E$ --- разрешимое расширение поля $k$.
\end{proof}

\begin{theorem}
\label{RazrRad}
Пусть $E$ --- сепарабельное расширение поля $k$. Тогда $E$ --- разрешимо в радикалах тогда и только тогда, когда $E$ --- разрешимое расширение поля $k$.
\end{theorem}


\begin{proof}
Предположим, что $E$ --- разрешимо. Пусть $K$ --- разрешимое расширение Галуа поля $k$, содержащее $E$. Пусть $m$ --- произведение всех степеней простых чисел, не равных характеристике и делящих $[K:k]$. Положим $F=k(\zeta)$, где $\zeta$ --- примитивный корень $m$-й степени из единицы. Заметим, что $F$ --- абелево расширение поля $k$. Поднимем $K$ над $F$. Согласно \ref{Razr1} $KF$ разрешимо над $F$. Тогда существует башня полей между $k$ и $KF$ такая, что каждый ее этаж --- циклическое расширение. В силу теорем \ref{Edin3} и \ref{ArtShr} $KF$ разрешимо в радикалах над $k$. Следовательно, $E$ --- разрешимо в радикалах над $k$.

Обратно, предположим, что $E$ --- разрешимо в радикалах над $k$. Пусть $\sigma_1,\sigma_2,\ldots,\sigma_n$ --- вложения $E$ над $k$ в алгебраическое замыкание $\bar{E}$. Пусть $K$ --- наименьшее поле, содержащее все $\sigma_i E$. Тогда $K$ --- расширение Галуа разрешимое в радикалах. Пусть $m$ --- произведение всех степеней простых чисел, не равных характеристике и делящих $[K:k]$. Положим $F=k(\zeta)$, где $\zeta$ --- примитивный корень $m$-й степени из единицы. Заметим, что $KF$ разрешимо над $F$. Отсюда, $KF$ разрешимо над $k$.
\end{proof}


\section{Теория Куммера}

Пусть $K$ --- расширение Галуа поля $k$ и $m$ --- целое положительное число. Будем говорить, что это расширение \emph{показателя} $m$, если $\sigma^m=1$ для всех $\sigma\in G(K/k)$. Пусть $m$ взаимно просто с характеристикой поля. Мы будем обозначать через $a^{\frac{1}{m}}$ любой элемент $\alpha$, что $\alpha^m=a$. Пусть $B$ --- подгруппа $k^*$, содержащая $(k^*)^m$. Положим $K_B=k(B^{\frac{1}{m}})$ --- расширение, порожденное всеми $a^{\frac{1}{m}}$, где $a\in B$. Заметим, что $K_B$ --- расширение Галуа. Действительно, любое вложение $\sigma$ поля $K_B$ в $\bar{k}$ переводит корни любого многочлена $x^m-a$ в себя. Поскольку $K_B$ порождается $a^{\frac{1}{m}}$, то $\sigma$ автоморфизм. Пусть $G$ --- группа Галуа расширения $K_B$ над $k$. Пусть $\sigma\in G$, $\alpha=a^{\frac{1}{m}}$. Тогда $\sigma\alpha=\zeta\alpha$, где $\zeta$ --- корень $m$-й степени из единицы. Отображение $\sigma\rightarrow\zeta$ является гомоморфизмом $G$ в $\ZZ_m$. Заметим, что $\zeta=\frac{\sigma\alpha}{\alpha}$. Пусть $\alpha'=a^{\frac{1}{m}}$ --- другой корень $m$-й степени из $a$. Тогда  $\alpha'=\zeta'\alpha$. Отсюда, $$\frac{\sigma\alpha'}{\alpha'}=\frac{\sigma(\zeta'\alpha')}{\zeta'\alpha}=\frac{\sigma\alpha}{\alpha}=\zeta.$$ Таким образом, $\frac{\sigma\alpha}{\alpha}$ не зависит от выбора $\alpha$. Соответствие $(\sigma,a)=\frac{\sigma\alpha}{\alpha}$ задает отображение $G\times B\rightarrow\ZZ_m$. Очевидно, что $(\sigma,a)=1$, если $a\in (k^*)^m$. Пусть $a,b\in B$ и $\alpha^m=a$, $\beta^m=b$. Тогда $(\alpha\beta)^m=ab$. Следовательно, $$(\sigma,ab)=\frac{\sigma(\alpha\beta)}{\alpha\beta}=\frac{\sigma(\alpha)\sigma(\beta)}{\alpha\beta}=\frac{\sigma(\alpha)}{\alpha}\frac{\sigma(\beta)}{\beta}=(\sigma,a)(\sigma,b).$$

\begin{theorem}
\label{Kum1}
Пусть $k$ --- поле, $m$ --- целое положительное число взаимно простое с характеристикой, причем примитивный корень $m$-й степени из единицы лежит в $k$. Пусть $B$ --- подгруппа в $k^*$, содержащая $(k^*)^m$, $K_B=k(B^{\frac{1}{m}})$. Тогда $K_B$ --- абелево расширение Галуа показателя $m$. Пусть $G$ --- его группа Галуа. Имеет место отображение $G\times B\rightarrow\ZZ_m$, задаваемое соответствием $(\sigma,a)=\frac{\sigma\alpha}{\alpha}$. Ядро слева равно $1$, ядро справа есть $(k^*)^m$.
\end{theorem}

\begin{proof}
Пусть $\sigma\in G$ и $(\sigma,a)=1$ для любого $a\in B$. Тогда $\sigma\alpha=\alpha$ для любого $\alpha$ такого, что $\alpha^n=a$. Следовательно, $\sigma$ действует тождественно на $K_B$. Таким образом, $\sigma=e$. Пусть $a\in B$ и $(\sigma,a)=1$ для любого $\sigma\in G$. Рассмотрим подполе $k(a^{\frac{1}{m}})$. Если $a^{\frac{1}{m}}$ не лежит в $k$, то существует автоморфизм $\sigma$ поля $k(a^{\frac{1}{m}})$ над $k$, который не является тождественным. Автоморфизм $\sigma$ можно продолжить на $K_B$, т.е. до элемента группы Галуа. С другой стороны, $(\sigma,a)\neq 1$. Противоречие.
\end{proof}

\begin{corollary}
\label{Kum2}
Расширение $K_B$ над $k$ конечно тогда и только тогда, когда индекс $(B:(k^*)^m)$ конечен, и $[K_B:k]=(B:(k^*)^m)$.
\end{corollary}

\begin{theorem}
\label{Kum3}
Отображение $B\rightarrow K_B$ задает биективное соответствие между множеством всех подгрупп $B$, содержащих $(k^*)^m$, и множеством абелевых расширений над $k$ показателя $m$.
\end{theorem}

\begin{proof}
Пусть $B_1,B_2$ --- подгруппы в $k^*$, содержащие $(k^*)^m$. Если $B_1\subset B_2$, то $k(B_1^{\frac{1}{m}})\subset k(B_2^{\frac{1}{m}})$. Обратно, предположим, что $k(B_1^{\frac{1}{m}})\subset k(B_2^{\frac{1}{m}})$. Нам нужно доказать, что $B_1\subset B_2$. Пусть $b\in B_1$. Тогда $k(b^{\frac{1}{m}})\subset k(B_2^{\frac{1}{m}})$, причем $k(b^{\frac{1}{m}})$ содержится в некотором конечном подрасширении $k(B_2^{\frac{1}{m}})$. Таким образом, мы можем считать, что $B_2/k^*$ --- конечно порожденная, а, следовательно, конечная, группа. Пусть $B_3$ --- группа, порожденная $B_2$ и $k$. Тогда $k(B_3^{\frac{1}{m}})=k(B_2^{\frac{1}{m}})$. С другой стороны, согласно \ref{Kum2}, $$(B_2:(k^*)^m)=[K_{B_2}:k]=[K_{B_3}:k]=(B_3:(k^*)^m).$$ Следовательно, $B_2=B_3$. Отсюда, $b\in B_2$. Тогда $B_1\subset B_2$. Отсюда следует инъективность отображения $B\rightarrow K_B$.

Пусть $K$ --- абелево расширение поля $k$ показателя $m$. Предположим, что $K$ конечно. Тогда группа Галуа $G(K/k)$ раскладывается в прямое произведение циклических групп, порядка не выше $m$. Мы можем применить \ref{Komp6}. Таким образом, $K=K_1K_2\cdots K_n$, где $K_i$ --- циклические расширения. Согласно теореме \ref{Norm4}, $K_i$ может быть получено присоединением корня $m$-й степени из элемента $b_i$. Следовательно, $K$ может быть получено присоединением корней $m$-й степени из элементов $\{b_i\}$. Здесь мы уже можем не предполагать конечность расширения $K$ и числа элементов $\{b_i\}$. Пусть $B$ --- подгруппа в $k^*$, порожденная всеми $b$ и $(k^*)^m$. Тогда $k(B^{\frac{1}{m}})=K$.
\end{proof}

\section{Целые расширения Галуа}

В этом параграфе слово "кольцо" будет обозначать коммутативное кольцо с единицей.

\begin{definition}
Пусть $A$ --- кольцо и $M$ --- $A$-модуль. Будем говорить, что $M$ \emph{точный}, если из равентсва $aM=0$ следует, что $a=0$.
\end{definition}


\begin{theorem}
\label{Zel1}
Пусть $A$ --- подкольцо кольца $B$ и $\alpha\in B$. Следующие условия эквивалентны:
\begin{enumerate}
\item $\alpha$ есть корень многочлена $x^n+a_{n-1}x^{n-1}+\cdots+a_1x+a_0$.
\item Подкольцо $A[\alpha]$ --- конечно порожденный $A$-модуль.
\item Существует точный модуль над $A[\alpha]$, являющийся конечно порожденным $A$-модулем.
\end{enumerate}
\end{theorem}

\begin{proof}
Предположим, что выполнено первое условие. Пусть $f(x)\in A[x]$ --- многочлен со старшим коэффициентом единица, для которого $f(\alpha)=0$. Если $g(x)\in A[x]$, то $g(x)=f(x)q(x)+r(x)$, где $\deg r<\deg f=n$. Тогда $f(\alpha)=r(\alpha)$. Следовательно, $1,\alpha,\alpha^2,\ldots,\alpha^{n-1}$ являются образующими $A[\alpha]$. Уравнение $f(x)=0$ называется \emph{целым уравнением} для $\alpha$ над $A$.

Предположим, что выполнено второе условие. Тогда в качестве точного модуля можно взять само $A[\alpha]$.

Предположим, что выполнено третье условие. Пусть $M$ --- точный модуль над $A[\alpha]$, конечнопорожденный над $A$. Пусть $w_1,w_2,\ldots,w_n$ --- его пораждающие. Тогда существуют элементы $a_{ij}\in A$ такие, что $$\begin{cases}\alpha w_1=a_{11}w_1+a_{12}w_2+\cdots+a_{1n}w_n\\
\alpha w_2=a_{21}w_1+a_{22}w_2+\cdots+a_{2n}w_n\\
\cdots\quad\cdots\quad\cdots\\
\alpha w_n=a_{n1}w_1+a_{n2}w_2+\cdots+a_{nn}w_n.\end{cases}$$
Перенесем $\alpha w_1,\alpha w_2,\ldots,\alpha w_n$ вправо. Получаем систему, которая должна иметь ненулевое решение. Тогда $$d=\begin{vmatrix}a_{11}-\alpha & a_{12} & \cdots & a_{1n} \\
a_{21} & a_{22}-\alpha & \cdots & a_{2n}\\
\cdots & \cdots & \cdots & \cdots\\
a_{n1} & a_{n2} & \cdots & a_{nn}-\alpha
\end{vmatrix}$$ аннулирует $M$, т.е. $dM=0$. Поскольку $M$ --- точный мотуль, то $d=0$. Тогда $$f(x)=\begin{vmatrix}a_{11}-x & a_{12} & \cdots & a_{1n} \\
a_{21} & a_{22}-x & \cdots & a_{2n}\\
\cdots & \cdots & \cdots & \cdots\\
a_{n1} & a_{n2} & \cdots & a_{nn}-x
\end{vmatrix}$$ задает (с точностью до знака) целое уравнение для $\alpha$.
\end{proof}

Элемент $\alpha$, удовлетворяющий этим условиям называется \emph{целым} над $A$.

\begin{claim}
\label{Zel2}
Пусть $A$ --- целостное кольцо, $K$ --- его поле частных, и $\alpha$ --- алгебраический элемент над $K$. Тогда существует $d\in A$ такой, что $d\alpha$ --- целый элемент над $A$.
\end{claim}

\begin{proof}
Пусть $f(x)$ --- минимальный многочлен элемента $\alpha$. Умножая $f(x)$ на наименьшее общее кратное знаменателей его коэффициентов, получаем $$a_n\alpha^n+a_{n-1}\alpha^{n-1}+\cdots+a_1\alpha+a_0=0,$$ где $a_i\in A$, $a_n\neq 0$. Умножим это уравнение на $a_n^{n-1}$, получим $$(a_n\alpha)^n+a_{n-1}(a_n\alpha)^{n-1}+\cdots+a_n^{n-2}a_1(a_n\alpha)+a_n^{n-1}a_0=0.$$ Таким образом, $a_n\alpha$ --- целый элемент над $A$.
\end{proof}

Пусть кольцо $A$ содержится в кольце $B$. Мы говорим, что $B$ --- \emph{целое кольцо} над $A$ (\emph{целое расширение} кольца $A$), если любой элемент из $B$ является целым над $A$.

\begin{claim}
\label{Zel3}
Пусть $B$ --- целое расширение кольца $A$, конечнопорожденное, как $A$-алгебра. Тогда $B$ --- конечнопорожденный $A$-модуль.
\end{claim}

\begin{proof}
Докажем по индукции. Пусть $$A\subset A[\alpha_1]\subset A[\alpha_1,\alpha_2]\subset\cdots\subset A[\alpha_1,\alpha_2,\ldots,\alpha_n]=B,$$ где каждый $\alpha_i$ --- целый элемент над $A$, а следовательно и над $A[\alpha_1,\alpha_2,\ldots,\alpha_{i-1}]$. Исходя из определения, $A[\alpha_1,\alpha_2,\ldots,\alpha_{i-1}][\alpha_i]$ --- конечно порожденный $A[\alpha_1,\alpha_2,\ldots,\alpha_{i-1}]$-модуль. Отсюда, $B$ --- конечнопорожденный $A$-модуль.
\end{proof}

\begin{theorem}
\label{Zel4}
Пусть $B$ --- целое расширение кольца $A$, $C$ --- целое расширение кольца $B$. Тогда $C$ --- целое расширение кольца $A$. Обратно, если $C$ --- целое расширение кольца $A$, то $B$ --- целое расширение кольца $A$, и $C$ --- целое расширение кольца $B$.
\end{theorem}

\begin{proof}
Если $C$ --- целое кольцо над $A$, то ясно, что $B$ --- целое кольцо над $A$, и $C$ --- целое кольцо над $B$. Предположим, что $B$ --- целое расширение кольца $A$, $C$ --- целое расширение кольца $B$. Пусть $\alpha\in C$. Тогда $$\alpha^n+b_{n-1}\alpha^{n-1}+\cdots+b_1\alpha+b_0=0,$$ где $b_i\in B$. Положим $B_1=A[b_0,b_1,\cdots,b_{n-1}]$. Тогда, согласно утверждению \ref{Zel3}, $B_1$ --- конечнопорожденный $A$-модуль. Следовательно, $B_1[\alpha]$ --- конечнопорожденный $A$-модуль. С другой стороны, поскольку $A[\alpha]\subset B_1[\alpha]$, то $B_1[\alpha]$ --- точный $A[\alpha]$-модуль. Отсюда, $C$ --- целое расширение кольца $A$.
\end{proof}

\begin{theorem}
\label{Zel5}
Пусть $A$ --- подкольцо кольца $B$. Тогда элементы $B$, целые над $A$, образуют подкольцо в $B$.
\end{theorem}

\begin{proof}
Если $\alpha\in B$ --- целый над $A$, то $A[\alpha]$ --- целое расширение кольца $A$. Действительно, для любого $\alpha'\in A[\alpha]$, $A[\alpha]$ является точным $A[\alpha']$-модулем. С другой стороны, $A[\alpha]$ --- конечно порожденный $A$-модуль. Пусть $\alpha,\beta\in B$ --- целые элементы над $A$. Рассмотрим башню $A\subset A[\alpha]\subset A[\alpha,\beta]$. Каждый этаж этой башни является целым расширением. Тогда, по теореме \ref{Zel4}, $A[\alpha,\beta]$ --- целое расширение $A$. Таким образом, $\alpha\pm\beta$, $\alpha\beta$ --- целые элементы над $A$.
\end{proof}

\begin{theorem}
\label{Zel6}
Пусть $A$ --- целостное кольцо, $k$ --- его поле частных, $E$ --- конечное расширение над $k$ и $\alpha\in E$ --- целый элемент над $A$. Тогда коэффициенты минимального многочлена $\alpha$ являются целыми над $A$. В частности целыми будут норма и след.
\end{theorem}

\begin{proof}
Для всякого вложения $\sigma$ поля $E$ над $k$.
\end{proof}

Пусть $A\subset B$. Множество элементов из $B$, целых над $A$, называется \emph{целым замыканием} кольца $A$ в $B$. Будем говорить, что целостное кольцо $A$ \emph{целозамкнуто}, если целое замыкание $A$ в своем поле частных совпадает с $A$.







\begin{thebibliography}{lll}
\bibitem{V}
Ван-дер-Варден Б.Л. \emph{Современная алгебра}.
\bibitem{ZS}
Зарисский О., Самюэль П. \emph{Комутативная алгебра}.
\bibitem{KM}
Каргополов М.И., Мерзляков Ю.И. \emph{Основы теории групп}.
\bibitem{Ko}
Кострикин А.И. \emph{Основы алгебры}.
\bibitem{Ku}
Курош А.Г., \emph{Курс высшей алгебры}.
\bibitem{L}
Ленг С. \emph{Алгебра}.
\bibitem{P}
Постников М. М. \emph{Теория Галуа}.





\end{thebibliography}


\end{document}
\bibitem{KSB} 