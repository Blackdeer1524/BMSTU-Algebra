\documentclass[12pt, titlepage, oneside]{amsbook}
\makeatletter \@addtoreset{equation}{chapter}
\@addtoreset{figure}{chapter} \@addtoreset{theorem}{chapter}
\makeatother \sloppy \pagestyle{plain} \setcounter{tocdepth}{1}

\renewcommand{\theequation}{\thesection.\arabic{equation}}

\usepackage[russian]{babel}

\usepackage{pstcol}
\usepackage{pstricks, pst-node}
\usepackage[matrix, arrow,curve]{xy}
\usepackage{xypic,amscd}
\usepackage{amsmath}
\usepackage{amssymb}
\usepackage{latexsym}
\usepackage{graphicx}
\usepackage{wasysym}
\binoppenalty=10000 \relpenalty=10000
\parskip = 3pt
\parindent = 0.4cm

\newcommand{\Div}{\operatorname{div}}
\newcommand{\ddef}{\operatorname{def}}
\newcommand{\rot}{\operatorname{rot}}
\newcommand{\grad}{\operatorname{grad}}
\newcommand{\Pic}{\operatorname{Pic}}
\newcommand{\Aut}{\operatorname{Aut}}
\newcommand{\PSL}{\operatorname{PSL}}
\newcommand{\GL}{\operatorname{GL}}
\newcommand{\Cr}{\operatorname{Cr}}
\newcommand{\St}{\operatorname{St}}
\newcommand{\Orb}{\operatorname{Orb}}
\newcommand{\Bs}{\operatorname{Bs}}
\newcommand{\mld}{\operatorname{mld}}
\newcommand{\cov}{\operatorname{cov}}
\newcommand{\cor}{\operatorname{cor}}
\newcommand{\sign}{\operatorname{sign}}
\newcommand{\mult}{\operatorname{mult}}
\newcommand{\Sing}{\operatorname{Sing}}
\newcommand{\Diff}{\operatorname{Diff}}
\newcommand{\Supp}{\operatorname{Supp}}
\newcommand{\Ext}{\operatorname{Ext}}
\newcommand{\Exc}{\operatorname{Exc}}
\newcommand{\Cl}{\operatorname{Cl}}
\newcommand{\discrep}{\operatorname{discrep}}
\newcommand{\discr}{\operatorname{discr}}
\newcommand{\di}{\operatorname{di}}
\newcommand{\Int}{\operatorname{Int}}
%\newcommand{\sign}{\operatorname{sign}}
\renewcommand\gcd{\operatorname{\text{{\rm НОД}}}\ }
%\newcommand{\GL}{\operatorname{GL}}


\newcommand{\muu}{{\boldsymbol{\mu}}}
\newcommand{\OOO}{{\mathcal O}}
\newcommand{\TTT}{{\mathcal T}}
\newcommand{\BBBB}{{\mathcal B}}
\newcommand{\LL}{{\mathcal L}}
\newcommand{\EE}{{\mathcal E}}
\newcommand{\PP}{\mathbf{P}}
\newcommand{\MM}{\mathbf{M}}
\newcommand{\NN}{\mathbb{N}}
\newcommand{\ZZ}{\mathbb{Z}}
\newcommand{\FF}{\mathbb{F}}
\newcommand{\RR}{\mathbb{R}}
\newcommand{\CC}{\mathbb{C}}
\newcommand{\QQ}{\mathbb{Q}}
\newcommand{\AAA}{\mathfrak{A}}
\newcommand{\DDD}{\mathfrak{D}}
\newcommand{\SSS}{\mathfrak{S}}
\newcommand{\MMM}{\mathfrak{M}}
\newcommand{\RRR}{\mathfrak{R}}
\newcommand{\BBB}{\mathfrak{B}}
\newcommand{\aaa}{\mathfrak{a}}
\newcommand{\ppp}{\mathfrak{p}}
\newcommand{\mmm}{\mathfrak{m}}
\newcommand{\DD}{\mathbb{D}}
\newcommand{\DDDD}{\mathbf{D}}

\newtheorem{theorem}{Теорема}[chapter]
\newtheorem{proposition}[theorem]{Предложение}
\newtheorem{lemma}[theorem]{Лемма}
\newtheorem{corollary}[theorem]{Следствие}
\newtheorem{claim}[theorem]{Утверждение}

\theoremstyle{definition}
\newtheorem{example}[theorem]{Пример}
\newtheorem{definition}[theorem]{Определение}
\newtheorem{notation}[theorem]{Обозначения}
\newtheorem{construction}[theorem]{Конструкция}
\newtheorem{pusto}[theorem]{}
\newtheorem{remark}[theorem]{Замечание}
\newtheorem{Exercise}[theorem]{Упражнение}
\newtheorem{zero}[theorem]{}
\newtheorem{case}[theorem]{}

\theoremstyle{remark}


\date{}

\begin{document}

\begin{titlepage}
	\begin{center}
		\large{\textbf{Лекции по курсу "Функциональный анализ"}} \quad \\
		\quad
		\\ \quad
		\\ \quad
		\large{\textbf{Белоусов Григорий Николаевич}} \quad \\ \quad
		
	\end{center}
\end{titlepage}

\tableofcontents

\chapter{Топологические и метрические пространства}

\section{Неравенства Юнга, Гельдера и Минковского}

Вначале рассмотрим несколько неравенств, которые понадобятся нам на протяжении курса.

\begin{theorem}[неравенство Юнга]
	\label{NerUng}
	Пусть числа $p>1$ и $q>1$ удовлетворяют условию $\frac{1}{p}+\frac{1}{q}=1$, $a,b\geq 0$. Тогда $$ab\leq\frac{a^p}{p}+\frac{b^q}{q}.$$
\end{theorem}

\begin{proof}
	Мы можем считать, что $a,b>0$. Рассмотрим функцию $$f(x)=x^{\frac{1}{p}}-\frac{x}{p},\quad x\geq 0.$$ Заметим, что $$f'(x)=\frac{1}{p}x^{\frac{1}{p}-1}-\frac{1}{p}=\frac{1}{p}\left(x^{-\frac{1}{q}}-1\right)$$ положительна при $0<x<1$, и отрицательна при $x>1$. Тогда в точке $x=1$ функция $f(x)$ принимает наибольшее значение. Следовательно, $$x^{\frac{1}{p}}-\frac{x}{p}\leq f(1)=1-\frac{1}{p}=\frac{1}{q}.$$ Пусть $x=\frac{a^p}{b^q}$. Тогда $$\frac{a}{b^{\frac{q}{p}}}-\frac{1}{p}\frac{a^p}{b^q}\leq\frac{1}{q}.$$ Отсюда, $$ab^{q-\frac{q}{p}}-\frac{a^p}{p}\leq\frac{b^q}{q}.$$ Получаем $$ab^{q(1-\frac{1}{p})}\leq\frac{a^p}{p}+\frac{b^q}{q}.$$ Здесь $q(1-\frac{1}{p})=q\frac{1}{q}=1$.
\end{proof}

\begin{theorem}[неравенство Гельдера]
	\label{NerGed}
	Пусть числа $p>1$ и $q>1$ удовлетворяют условию $\frac{1}{p}+\frac{1}{q}=1$. Тогда $$\sum\limits_{i=1}^n |a_i b_i|\leq\left(\sum\limits_{i=1}^n |a_i|^p\right)^{\frac{1}{p}}\left(\sum\limits_{i=1}^n |b_i|^q\right)^{\frac{1}{q}}.$$
\end{theorem}

\begin{proof}
	Заметим, что это неравенство однородно, т.е. оно выполнено для векторов $a$ и $b$ тогда и только тогда, когда оно выполнено для векторов $\lambda a$ и $\mu b$. Таким образом, мы можем считать, что $$\sum\limits_{i=1}^n |a_i|^p=\sum\limits_{i=1}^n |b_i|^q=1.$$ Согласно неравенству Юнга (см. \ref{NerUng}), $$|a_i b_i|\leq\frac{|a_i|^p}{p}+\frac{|b_i|^q}{q}.$$ Суммируя это неравенство по всем $i$, получаем $$\sum\limits_{i=1}^n |a_i b_i|\leq 1.$$ Что и требовалось доказать.
\end{proof}

Заметим, что при $p=q=2$ неравенство Гельдера переходит в неравенство Коши--Буняковского.

\begin{theorem}[неравенство Минковского]
	\label{NerMin}
	Пусть $p\geq 1$. Тогда $$\left(\sum\limits_{i=1}^n |a_i+b_i|^p\right)^{\frac{1}{p}}\leq\left(\sum\limits_{i=1}^n |a_i|^p\right)^{\frac{1}{p}}+\left(\sum\limits_{i=1}^n |b_i|^p\right)^{\frac{1}{p}}.$$
\end{theorem}

\begin{proof}
	При $p=1$ неравенство очевидно. Заметим, что $$(|a_i|+|b_i|)^p=(|a_i|+|b_i|)^{p-1}|a_i|+(|a_i|+|b_i|)^{p-1}|b_i|.$$ Суммируя это неравенство по всем $i$, получаем $$\sum\limits_{i=1}^n(|a_i|+|b_i|)^p=\sum\limits_{i=1}^n(|a_i|+|b_i|)^{p-1}|a_i|+\sum\limits_{i=1}^n(|a_i|+|b_i|)^{p-1}|b_i|.$$ Применим к каждому слагаемому в правой части неравенство Гельдера (см. \ref{NerGed}). Получим
	\begin{gather*}\sum\limits_{i=1}^n(|a_i|+|b_i|)^p\leq\left(\sum\limits_{i=1}^n((|a_i|+|b_i|)^{p-1})^q\right)^{\frac{1}{q}}\left(\sum\limits_{i=1}^n |a_i|^p\right)^{\frac{1}{p}}+\\
		+\left(\sum\limits_{i=1}^n((|a_i|+|b_i|)^{p-1})^q\right)^{\frac{1}{q}}\left(\sum\limits_{i=1}^n |b_i|^p\right)^{\frac{1}{p}},\end{gather*} где $q$ удовлетворяет условию $\frac{1}{p}+\frac{1}{q}=1$. Заметим, что $(p-1)q=p$. Тогда $$\sum\limits_{i=1}^n(|a_i|+|b_i|)^p\leq \left(\sum\limits_{i=1}^n(|a_i|+|b_i|)^{p}\right)^{\frac{1}{q}}\left(\left(\sum\limits_{i=1}^n |a_i|^p\right)^{\frac{1}{p}}+\left(\sum\limits_{i=1}^n |b_i|^p\right)^{\frac{1}{p}}\right).$$ Поделим обе части равенства на $$\left(\sum\limits_{i=1}^n(|a_i|+|b_i|)^{p}\right)^{\frac{1}{q}}.$$ Поскольку $1-\frac{1}{q}=\frac{1}{p}$, получаем $$\left(\sum\limits_{i=1}^n(|a_i|+|b_i|)^p\right)^{\frac{1}{p}}\leq \left(\sum\limits_{i=1}^n |a_i|^p\right)^{\frac{1}{p}}+\left(\sum\limits_{i=1}^n |b_i|^p\right)^{\frac{1}{p}}.$$ Откуда следует неравенство Минковского.
\end{proof}


\section{Топологические пространства}

\begin{definition}
	Пусть на множестве $X$ задана система подмножеств $\TTT$ такая, что
	\begin{itemize}
		\item если $U_i\in\TTT$ $\forall i$, то $\bigcap\limits_{i=1}^n U_i\in\TTT$, т.е. пересечение конечного числа подмножеств из $\TTT$ есть элемент $\TTT$.
		\item если $U_\alpha\in\TTT$ $\forall \alpha$, то $\bigcup\limits_{\alpha} U_\alpha\in\TTT$, т.е. объединение любого числа (конечного или бесконечного) подмножеств из $\TTT$ есть элемент $\TTT$.
		\item $X\in\TTT$, $\emptyset\in\TTT$.
	\end{itemize}
	Тогда пара $(X,T)$ называется \emph{топологическим пространством}. Система $\TTT$ называется \emph{топологией} на $X$. Элементы системы $\TTT$ называются \emph{открытыми множествами}. Элементы множества $X$ называются \emph{точками}. Открытое множество, содержащее точку $x$ мы будем называть \emph{окрестностью} точки $x$.
\end{definition}

\begin{definition}
	Семейство $\BBB$ называется \emph{базой} топологии, если $\BBB$ состоит из элементов $\TTT$ и любое множество $U\in\TTT$ представимо в виде объединения элементов семейства $\BBB$.
\end{definition}

\begin{definition}
	Множество $V\subset X$ называется \emph{замкнутым}, если его дополнение открыто, т.е. $X\setminus V\in\TTT$.
\end{definition}

\begin{remark}
	Заметим, что множества $X$ и $\emptyset$ одновременно открыты и замкнуты.
\end{remark}

Непосредственно из определения видно, что объединение конечного числа замкнутых множеств снова замкнуто. Пересечение любого числа замкнутых множеств также замкнуто.

\begin{example}
	Пусть $X$ состоит из двух элементов, $X=\{a,b\}$. Тогда $\TTT=\{\emptyset,a,X\}$ образует топологию на $X$, в которой точка $a$ открыта, а точка $b$ замкнута. Это топологическое пространство называется \emph{связное двоеточие}.
\end{example}

\begin{example}
	Пусть $X=\RR$ и $\BBB=\{(a,b)\}$ --- множество интервалов. Тогда $\BBB$ является базой топологии на $\RR$. Эта топология называется \emph{стандартной} топологией на прямой. Теперь в качестве $\TTT$ мы можем взять множества $\{\emptyset,\RR,\RR\setminus\{x_1,\ldots,x_n\}\}$, т.е. открытыми множествами являются пустое множество, вся прямая и прямая без конечного числа точек. Эта топология называется \emph{топологией Зарисского}.
\end{example}

\begin{remark}
	Предыдущий пример показывает, что на одном и том же множестве можно задать несколько топологий.
\end{remark}

\begin{theorem}
	\label{Top1}
	Семейство $\BBB$ тогта и только тогда является базой топологии $\TTT$, когда для любого множества $U\in\TTT$ и любой точки $x\in U$ существует $V\in\BBB$ такая, что $x\in V\subset U$.
\end{theorem}

\begin{proof}
	Пусть $U\in\TTT$ --- открытое множество. Тогда для каждой точки $\alpha\in U$ существует $V_\alpha\in\BBB$ такая, что $V_\alpha\subset U$. Тогда $U=\bigcup\limits_{\alpha} V_\alpha$. Обратно, пусть $\BBB$ --- база топологии $\TTT$. Тогда любое множество $U\in\TTT$ представимо $U=\bigcup\limits_{\alpha} V_\alpha$, $V_\alpha\in\BBB$. Пусть $x\in U=\bigcup\limits_{\alpha} V_\alpha$. Тогда существует $V_\alpha$ такая, что $x\in V_\alpha$ и $V_\alpha\subset U$.
\end{proof}

Заметим, что в процессе доказательства, мы доказали следующую лемму.
\begin{lemma}
	\label{Top2}
	Множество $U$ открыто тогда и только тогда, когда для любой точки $x\in U$ существует окрестность $V$ такая, что $V\subset U$.
\end{lemma}

Введем еще несколько определений.

\begin{definition}
	Будем говорить, что топологическое пространство $X$ удовлетворяет \emph{первой аксиоме отделимости} (обладает свойством $T_1$), если для любых двух точек $x,y\in X$ существует открытые множества $U$ и $V$ такие, что $x\in U, y\not\in U$ и $x\not\in V, y\in V$.
\end{definition}

\begin{definition}
	Будем говорить, что топологическое пространство $X$ удовлетворяет \emph{второй аксиоме отделимости} (обладает свойством $T_2$), если для любых двух точек $x,y\in X$ существует открытые множества $U$ и $V$ такие, что $x\in U, y\in V$ и $U\cap V=\emptyset$. $T_2$-пространства называются \emph{Хаусдорфовами}.
\end{definition}

Очевидно, что $T_2$-пространства являются и $T_1$-пространствами.

\begin{definition}
	Будем говорить, что топологическое пространство $X$ удовлетворяет \emph{второй аксиоме счетности}, если существует счетная база.
\end{definition}

\begin{definition}
	Будем говорить, что топологическое пространство $X$ удовлетворяет \emph{первой аксиоме счетности}, если для любой точки $x\in X$ существует счетная система множеств $\Sigma_x\subset\TTT$, содержащих $x$. Более того, для любого множества $V\in\TTT$, содержащего $x$, существует множество $U_i\in\Sigma_x$ такое, что $U_i\subset V$.
\end{definition}

Из теоремы \ref{Top1} видно, что пространства, удовлетворяющие второй аксиоме счетности, удовлетворяют и первой аксиоме счетности.

\begin{definition}
	Пусть $f\colon X\rightarrow Y$ --- отображение двух топологических пространств. Будем говорить, что $f$ \emph{непрерывно} в точке $x_0\in X$, если для любой окрестности $V\subset Y$ точки $f(x_0)$ существует окрестность $U$ точки $x_0$ такая, что $f(U)\subset V$. Отображение $f$ называется \emph{непрерывным}, если оно непрерывно в каждой точки пространства $X$. Непрерывное отображение $f\colon X\rightarrow Y$ называется \emph{гомеоморфизмом}, если $f$ взаимно однозначно отображает $X$ в $Y$ и отображение $f^{-1}\colon Y\rightarrow X$ непрерывно.
\end{definition}

\begin{theorem}
	\label{Top3}
	Отображение $f\colon X\rightarrow Y$ непрерывно тогда и только тогда, когда прообраз любого открытого множества является открытым множеством.
\end{theorem}

\begin{proof}
	Пусть $f\colon X\rightarrow Y$  --- непрерывное отображение, $V\subset Y$ --- открытое множество. Пусть $x\in f^{-1}(V)$. Заметим, что $V$ --- окрестность $f(x)$. Тогда существует окрестность $U$ точки $x$ такая, что $f(U)\subset V$, т.е. $U\subset f^{-1}(V)$. Согласно лемме \ref{Top2} $f^{-1}(V)$ открыто.
	Обратно, пусть $V\subset Y$ --- окрестность точки $f(x_0)$. Тогда $f^{-1}(V)$ --- окрестность точки $x_0$ и $f(f^{-1}(V))=V\subset V$. Следовательно, $f$ непрерывно.
\end{proof}

\section{Метрические пространства}

\begin{definition}
	Пусть на множестве $X$ задана функция $\rho(x,y)$ такая что
	\begin{itemize}
		\item $\rho(x,y)=0$ тогда, и только тогда когда $x=y$;
		\item $\rho(x,y)=\rho(y,x)$;
		\item $\rho(x,y)\leq \rho(x,z)+\rho(z,y)$ (неравенство треугольника).
	\end{itemize}
	Тогда пара $(X,\rho)$ называется \emph{метрическим пространством}. Функция $\rho(x,y)$ называется \emph{метрикой}.
\end{definition}

\begin{claim}
	\label{Met1}
	$\rho(x,y)\geq 0$.
\end{claim}

\begin{proof}
	Действительно, $$0=\rho(x,x)\leq \rho(x,y)+\rho(y,x)=2\rho(x,y).$$
\end{proof}

\begin{definition}
	\emph{Шаровой окрестностью} точки $x\in X$ радиуса $\varepsilon$ называется множество $U_{\varepsilon}(x)=\{ y\mid \rho(x,y)<\varepsilon\}$.
\end{definition}

\begin{definition}
	Точка $x$ называется \emph{внутренней точкой} множества $Y$, если существует $\varepsilon$ такое, что $U_{\varepsilon}(x)\subset Y$. Множество внутренних точек называется \emph{внутренностью} и обозначается $\Int Y$. Точка $x$ называется \emph{граничной точкой} множества $Y$, если для любого $\varepsilon>0$ шаровая окрестность $U_{\varepsilon}(x)$ содержит точки $Y$, но $U_{\varepsilon}(x)\not\subset Y$. Множество всех граничных точек называется \emph{границей} и обозначается $\partial Y$. \emph{Замыканием} множества $Y$ называется $\bar{Y}=Y\cup\partial Y$.
\end{definition}

\begin{definition}
	Множество $Y$ называется \emph{замкнутым}, если $\partial Y\subset Y$, и \emph{открытым}, если $\partial Y\cap Y=\emptyset$. Множества $\emptyset$ и $X$ мы будем считать и замкнутыми, и открытыми.
\end{definition}

Очевидно, что если $Y$ --- открытое множество, то $X\setminus Y$ --- замкнутое множество, и наоборот.
Сейчас мы докажем несколько утверждений, которые покажут, что система открытых множеств задает топологию.

\begin{definition}
	\emph{Расстоянием} между двумя множествами $A,B\subset X$ называется $\rho(A,B)\inf\limits_{x\in A,y\in B}\rho(x,y)$. \emph{Точкой прикосновения} множества $A\subset X$ называется всякая точка $x$ для которой $\rho(x,A)=0$.
\end{definition}

Заметим, что точка прикосновения множества $Y$ это точка в любой окрестности которой содержатся точки множества $Y$, т.е. это либо внутренняя точка, либо точка границы. Очевидно, что замыкание множества $Y$ можно описать, как множество всех точек прикосновения.

\begin{claim}
	\label{Met2}
	Пересечение двух открытых множеств --- открытое множество.
\end{claim}

\begin{proof}
	Пусть $U_1$ и $U_2$ --- два открытых множества. Пусть $x\in U_1\cap U_2$. Тогда $x\in U_1$ и $x\in U_2$. Следовательно, существует $\varepsilon_1$ такое что $U_{\varepsilon_1}(x)\subset U_1$, и существует $\varepsilon_2$ такое, что $U_{\varepsilon_2}(x)\subset U_2$. Положим $\varepsilon=\min(\varepsilon_1,\varepsilon_2)$. Тогда $U_{\varepsilon}(x)\subset U_1$ и $U_{\varepsilon}(x)\subset U_2$. Следовательно, $U_{\varepsilon}(x)\subset U_1\cap U_2$. Отсюда, $x$ --- внутренняя точка.
\end{proof}

Заметим, что из утверждения \ref{Met2} следует, что пересечение любого конечного числа открытых множеств --- открытое множество.

\begin{claim}
	\label{Met3}
	Объединение любого числа открытых множеств --- открытое множество.
\end{claim}

\begin{proof}
	Пусть $x\in\bigcup\limits_{\alpha} U_\alpha$. Тогда существует $U_\alpha$ такое, что $x\in U_\alpha$. Следовательно, существует $U_{\varepsilon}(x)$ такое, что $U_{\varepsilon}(x)\subset U_\alpha$. Тогда $U_{\varepsilon}(x)\subset \bigcup\limits_{\alpha} U_\alpha$.
\end{proof}

Из утверждений \ref{Met2} и \ref{Met3} следует, что открытые множества задают топологию.

\begin{claim}
	\label{Met4}
	Множество шаровых окрестностей образуют базу этой топологии.
\end{claim}

\begin{proof}
	%Пусть $V$ --- открытое множество. Тогда для каждой точке $\alpha\in V$ существует шаровая окрестность $U_\alpha\subset V$. Тогда $V=\bigcup\limits_{\alpha} U_\alpha$.\
	Следует из теоремы \ref{Top1}.
\end{proof}

\begin{example}
	\label{Ex1}
	Пусть $X=\RR$. Тогда мы можем задать метрику $\rho(x,y)=|x-y|$. Аналогично мы можем ввести метрику на $\RR^n$. Пусть $x=(x_1,x_2,\ldots,x_n)$, $y=(y_1,y_2,\ldots,y_n)$. Тогда $$\rho(x,y)=\sqrt{(x_1-y_1)^2+(x_2-y_2)^2+\cdots+(x_n-y_n)^2}.$$ Более того, если $p\geq 1$, то мы можем определить метрику $$\rho(x,y)=\left(\sum\limits_{i=1}^n|x_i-y_i|^p\right)^{\frac{1}{p}}.$$ Из неравенства Минковского (см. \ref{NerMin}) следует неравенство треугольника для этой метрике. Еще одну метрику на $\RR^n$ можно задать, как $$\rho(x,y)=\max\limits_i |x_i-y_i|.$$
\end{example}

\begin{definition}
	Две метрики называются \emph{эквивалентными}, если они задают одну и ту же топологию.
\end{definition}

\begin{remark}
	Все метрики в примере \ref{Ex1} эквивалентны.
\end{remark}

\begin{claim}
	\label{Met5}
	Все метрические пространства являются Хаусдорфовыми.
\end{claim}

\begin{proof}
	Пусть $x,y\in X$ и $\rho(x,y)=a$. Возьмем две шаровые окрестности $U_{\frac{a}{2}}(x)$ и $U_{\frac{a}{2}}(y)$. Докажем, что они не пересекаются. Предположим, что существует точка $z\in U_{\frac{a}{2}}(x)$ и $z\in U_{\frac{a}{2}}(y)$. Тогда $$\rho(x,y)\leq\rho(z,x)+\rho(z,y)<\frac{a}{2}+\frac{a}{2}=a.$$ Противоречие.
\end{proof}

\begin{claim}
	\label{Met6}
	Все метрические пространства удовлетворяют первой аксиоме счетности.
\end{claim}

\begin{proof}
	Как известно, множество рациональных чисел счетно. Возьмем множество шаровых окрестностей рационального радиуса с центром в точке $x$. Пусть $x\in V$. Тогда существует шаровая окрестность $U_{\varepsilon}(x)$ такая, что $U_{\varepsilon}(x)\subset V$. Пусть $\tilde{\varepsilon}$ --- рациональное число, меньшее $\varepsilon$. Тогда $U_{\tilde{\varepsilon}}(x)\subset U_{\varepsilon}(x)\subset V$.
\end{proof}

Теперь рассмотрим операцию замыкание. Имеют место следующие свойства.
\begin{enumerate}
	\item $\bar{Y}$ --- замкнутое множество.
	\item $\overline{Y_1\cup Y_2}=\bar{Y}_1\cup\bar{Y}_2$
	\item $\overline{Y_1\cap Y_2}=\bar{Y}_1\cap\bar{Y}_2$
\end{enumerate}

\begin{definition}
	Множество $Y\subset X$ называется \emph{всюду плотным} в $X$, если $\bar{Y}=X$. Множество $Y\subset X$ называется \emph{нигде не плотным}, если $\Int\bar{Y}=\emptyset$.
\end{definition}

\begin{definition}
	Пространство $X$ называется \emph{сепарабельным}, если в нем существует счетное всюду плотное множество $Y$.
\end{definition}

\begin{example}
	Пусть $X=\RR$. Поскольку $\QQ$ всюду плотно в $\RR$ и счетно, то $\RR$ сепарабельно. Аналогично сепарабельными являются все пространства $\RR^n$.
\end{example}

\begin{theorem}
	\label{Met7}
	Каждое пространство $X$, удовлетворяющее второй аксиоме счетности, сепарабельно.
\end{theorem}

\begin{proof}
	Пусть $\BBB=\{U_i\}$ счетная база пространства $X$. Выберем в каждом $U_i$ точку $x_i$. Обозначим через $M$ --- множество этих точек. Пусть $x$ --- произвольная точка и $V$ --- ее окрестность. Согласно теореме \ref{Top1} существует множество $U_i\subset V$, т.е. существует точка $x_i\in V$. Тогда $x\in\bar{M}$. Следовательно, $\bar{M}=X$.
\end{proof}

\begin{theorem}
	\label{Met8}
	Пусть $f\colon X\rightarrow Y$ --- отображение двух метрических пространств. Отображение $f$ непрерывно в точке $x_0$ тогда и только тогда, когда для любого $\varepsilon>0$ существует $\delta(\varepsilon)$ такое, что для любого $x$ с условием $\rho(x,x_0)<\delta$, выполнено неравенство $\rho_1(f(x),f(x_0))<\varepsilon$ (здесь $\rho_1$ означает метрику в $Y$).
\end{theorem}

\begin{proof}
	Пусть $f\colon X\rightarrow Y$ непрерывно в точке $x_0$. Рассмотрим шаровую окрестность $U_{\varepsilon}(f(x_0))$. Согласно определению, существует окрестность $V$ точки $x_0$ такая, что $f(V)\subset U$. Поскольку $V$ открыто и содержит $x_0$, то существует шаровая окрестность $V_{\delta}(x_0)\subset V$. Следовательно, $f(V_{\delta}(x_0))\subset U_{\varepsilon}(f(x_0))$. Обратно, пусть $U$ --- окрестность точки $f(x_0)$. Тогда существует шаровая окрестность $U_{\varepsilon}(f(x_0))\subset U$. По предположению, существует $\delta(\varepsilon)$ такое, что для любого $x$ с условием $\rho(x,x_0)<\delta$, выполнено неравенство $\rho_1(f(x),f(x_0))<\varepsilon$. Рассмотрим шаровую окрестность $V_{\delta}(x_0)$. Тогда $f(V_{\delta}(x_0))\subset U_{\varepsilon}(f(x_0))\subset U$. Следовательно, $f$ непрерывно в точке $x_0$.
\end{proof}

\section{Полные метрические пространства}

Пусть дана последовательность $\{x_n\}$ в метрическом пространстве $X$. Будем говорить, что эта последовательность \emph{сходится} к точке $x$, если для любого $\varepsilon>0$ существует $N$ такое, что для любых $n>N$ выполнено неравенство $\rho(x_n,x)<\varepsilon$. Или по другому $$\lim\limits_{n\rightarrow\infty}\rho(x_n,x)=0.$$

\begin{theorem}
	\label{Pol1}
	Точка $x\in Y$ тогда и только тогда является точкой прикосновения этого множества, когда существует последовательность $\{x_n\}$ точек из $Y$ сходящаяся к $x$.
\end{theorem}

\begin{proof}
	Пусть $x$ --- точка прикосновения. Тогда в шаровой окрестности $U_{\frac{1}{n}}(x)$ содержится хотя бы одна точка $x_n\in Y$. Эти точки образуют последовательность, сходящуюся к $x$. Обратно, пусть существует последовательность $\{x_n\}$ точек из $Y$ сходящаяся к $x$. Тогда в любой окрестности $U_{\varepsilon}(x)$ содержатся все точки последовательности $\{x_n\}$, начиная с какого-то $n$.
\end{proof}

\begin{definition}
	Последовательность $\{x_n\}\subset X$ называется \emph{фундаментальной}, если для любого $\varepsilon>0$ существует $N(\varepsilon)$ такое, что для любых $n,m>N$ выполнено неравенство $\rho(x_n,x_m)<\varepsilon$.
\end{definition}

Любая сходящаяся последовательность фундаментальна. Действительно, выберем такое $N$, что для любых $n,m>N$ выполнялись неравенства $\rho(x_n,x)<\frac{\varepsilon}{2}$, $\rho(x_m,x)<\frac{\varepsilon}{2}$. Тогда, из неравенства треугольника, $$\rho(x_n,x_m)\leq\rho(x_n,x)+\rho(x,x_m)<\frac{\varepsilon}{2}+\frac{\varepsilon}{2}=\varepsilon.$$

\begin{definition}
	Если в пространстве $X$ любая фундаментальная последовательность сходится, то это пространство называется \emph{полным}.
\end{definition}

\begin{example}
	Пространство $\RR$ с обычной метрикой является полным. Пространство $\QQ$ не является полным. Пространство $\RR^n$ с обычной метрикой также является полным.
\end{example}

\begin{theorem}
	\label{Pol2}
	Метрическое пространство $X$ полно тогда и только тогда, когда в нем всякая последовательность вложенных друг в друга замкнутых шаров, радиусы которых стремятся к нулю, имела непустое пересечение.
\end{theorem}

\begin{proof}
	Предположим, что $X$ полно. Пусть $B_1, B_2,\ldots$ --- последовательность вложенных друг в друга замкнутых шаров, и пусть $x_1,x_2,\ldots$ --- их центры, $r_1,r_2,\ldots$ --- их радиусы ($r_n\rightarrow 0$). Заметим, что $\rho(x_n,x_m)<\max(r_n,r_m)$. Следовательно, последовательность $x_1,x_2,\ldots$ фундаментальна. Пусть $x$ --- ее предел. С другой стороны, шар $B_m$ содержит все точки последовательности $\{x_n\}$, за исключением, быть может, точек $x_1,x_2,\ldots,x_{m-1}$. Тогда $x$ --- точка прикосновения шара $B_m$ (см. теорема \ref{Pol1}). Поскольку $B_m$ --- замкнутый шар, то $x\in B_m$. Следовательно, $x\in\bigcap\limits_n B_n$.
	
	Предположим, что всякая последовательность вложенных друг в друга замкнутых шаров, радиусы которых стремятся к нулю, имеет непустое пересечение. Пусть $\{x_n\}$ --- фундаментальная последовательность. Тогда существует элемент $x_{n_1}$ такой, что $\rho(x_{n_1}, x_n)<\frac{1}{2}$ для всех $n>n_1$. Пусть $B_1$ --- замкнутый шар, радиуса $1$ с центром в $x_{n_1}$. Тогда $B_1$ содержит все элементы последовательности $\{x_n\}$ начиная с $x_{n_1}$. Выберем элемент $x_{n_2}$ такой, что $\rho(x_{n_2}, x_n)<\frac{1}{2^2}$ для всех $n>n_2$ и $n_2>n_1$. Пусть $B_2$ --- замкнутый шар, радиуса $\frac{1}{2}$ с центром в $x_{n_2}$. Продолжая этот процесс, мы получим последовательность вложенных друг в друга замкнутых шаров $B_n$. Их радиусы равны $\frac{1}{2^{n-1}}$, т.е. стремятся к нулю. Тогда существует точка $x\in\bigcap\limits_n B_n$. Очевидно, что $x$ является пределом последовательности $\{x_n\}$.
\end{proof}

\begin{theorem}[теорема Бэра]
	\label{Pol3}
	Полное метрическое пространство $X$ не может быть представлено в виде объединения счетного числа нигде не плотных множеств.
\end{theorem}

\begin{proof}
	Предположим противное, т.е. пусть $$X=\bigcup\limits_{n=1}^{\infty}M_n,$$ где $M_n$ нигде не плотно. Пусть $B_1$ --- замкнутый шар, такой что $B_1\cap M_1=\emptyset$, и его радиуса меньше $1$. Поскольку $B_1\cap M_2$ нигде не плотно, то существует замкнутый шар $B_2$  такой что $B_2\subset B_1$, $B_2\cap M_2=\emptyset$ и радиус $B_2$ меньше $\frac{1}{2}$. Аналогично, $B_2\cap M_3$ нигде не плотно. Тогда существует замкнутый шар $B_3$  такой что $B_3\subset B_2$, $B_3\cap M_3=\emptyset$ и радиус $B_3$ меньше $\frac{1}{2^2}$. Продолжая этот процесс, мы получим последовательность вложенных друг в друга замкнутых шаров $B_n$, радиусы которых стремятся к нулю. По теореме \ref{Pol2} существует точка $x\in\bigcap\limits_n B_n$. С другой стороны, $x\not\in\bigcup\limits_{n=1}^{\infty}M_n$.
\end{proof}

\begin{definition}
	Точка $x\in X$ называется \emph{изолированной}, если существует $\varepsilon>0$ такой, что $U_{\varepsilon}(x)$ состоит из одной точки $x$. Пространство, каждая точка которого является изолированной, называется \emph{дискретным}.
\end{definition}

\begin{claim}
	\label{Pol4}
	Всякое полное метрическое пространство без изолированных точек несчетно.
\end{claim}

\begin{definition}
	Пусть $X$ --- метрическое пространство. Полное метрическое пространство $X^*$ называется \emph{пополнением} пространства $X$, если $X\subset X^*$ и $\bar{X}=X^*$.
\end{definition}

\begin{theorem}
	\label{Pol5}
	Каждое метрическое пространство $X$ имеет единственное (с точности до изометрии) пополнение.
\end{theorem}

\begin{proof}
	Пусть $X$ --- метрическое пространство. Рассмотрим множество фундаментальных последовательностей $\tilde{X}$. Назовем две фундаментальные последовательности $\{x_n\}$ и $\{y_n\}$ эквивалентными, если $\lim\limits_{n\rightarrow\infty}\rho(x_n,y_n)=0$. Это отношение рефлексивно, симметрично и транзитивно. Рефлексивность и симметричность очевидна. Транзитивность следует из неравенства треугольника $$\rho(x_n,z_n)\leq \rho(x_n,y_n)+\rho(y_n,z_n).$$ Таким образом, $\tilde{X}$ распадается на классы эквивалентности. Определим пространство $X^*$. Его точками будут классы эквивалентных между собой последовательностей. Определим метрику на этом пространстве следующим образом. Пусть $x^*$, $y^*$ --- два класса эквивалентности. Пусть $\{x_n\}\in x^*$ и $\{y_n\}\in y^*$. Тогда $$\rho(x^*,y^*)=\lim\limits_{n\rightarrow\infty}\rho(x_n,y_n).$$ Докажем, что этот предел существует и не зависит от выбора $\{x_n\}\in x^*$ и $\{y_n\}\in y^*$. Заметим, что $$|\rho(x_n,y_n)-\rho(x_m,y_m)|=|\rho(x_n,y_n)-\rho(x_n,y_m)+\rho(x_n,y_m)-\rho(x_m,y_m)|\leq$$ $$\leq|\rho(x_n,y_n)-\rho(x_n,y_m)|+|\rho(x_n,y_m)-\rho(x_m,y_m)|\leq\rho(y_n,y_m)+\rho(x_n,x_m).$$ Здесь мы использовали $$\rho(x_n,y_n)\leq\rho(y_n,y_m)+\rho(x_n,y_m),$$ $$\rho(x_n,y_m)\leq\rho(x_n,x_m)+\rho(x_m,y_m).$$ Поскольку $\{x_n\}$ и $\{y_n\}$ --- фундаментальные последовательности, то для любого $\varepsilon>0$ существует $N\in\NN$ такое, что для любых $n,m>N$ выполнено $\rho(y_n,y_m)<\frac{\varepsilon}{2}$ и $\rho(x_n,x_m)<\frac{\varepsilon}{2}$. Тогда $$|\rho(x_n,y_n)-\rho(x_m,y_m)|<\frac{\varepsilon}{2}+\frac{\varepsilon}{2}=\varepsilon.$$ Таким образом, согласно критерию Коши последовательность $\rho(x_n,y_n)$ сходится. Докажем единственность. Пусть $\{x_n\},\{x'_n\}\in x^*$ и $\{y_n\},\{y'_n\}\in y^*$. Тогда $$|\rho(x_n,y_n)-\rho(x'_n,y'_n)|=|\rho(x_n,y_n)-\rho(x_n,y'_n)+\rho(x_n,y'_n)-\rho(x'_n,y'_n)|\leq$$ $$|\rho(x_n,y_n)-\rho(x_n,y'_n)|+|\rho(x_n,y'_n)-\rho(x'_n,y'_n)|\leq\rho(y_n,y'_n)+\rho(x_n,x'_n).$$ Поскольку $$\lim\limits_{n\rightarrow\infty}\rho(x_n,x'_n)=0,\quad\lim\limits_{n\rightarrow\infty}\rho(y_n,y'_n)=0,$$ то $$\lim\limits_{n\rightarrow\infty}\rho(x_n,y_n)=\lim\limits_{n\rightarrow\infty}\rho(x'_n,y'_n).$$ Теперь проверим аксиомы метрического пространства. То, что $\rho(x^*,y^*)=0$ тогда и только тогда, когда $x^*=y^*$ следует из определения классов эквивалентности. Симметричность ($\rho(x^*,y^*)=\rho(y^*,x^*)$) очевидна. Осталось доказать неравенство треугольника. Пусть $x^*,y^*,z^*\in X^*$, и пусть $\{x_n\}\in x^*$, $\{y_n\}\in y^*$, $\{z_n\}\in z^*$. Поскольку $X$ --- метрическое пространство, то $$\rho(x_n,z_n)\leq\rho(x_n,y_n)+\rho(y_n,z_n).$$ Переходим к пределу при $n\rightarrow\infty$, получаем $$\lim\limits_{n\rightarrow\infty}\rho(x_n,z_n)\leq\lim\limits_{n\rightarrow\infty}\rho(x_n,y_n)+\lim\limits_{n\rightarrow\infty}\rho(y_n,z_n).$$ Отсюда, $$\rho(x^*,z^*)\leq\rho(x^*,y^*)+\rho(y^*,z^*).$$ Таким образом, $X^*$ --- метрическое пространство. Теперь отобразим $X$ в пространство $X^*$. Сопоставим $x\in X$ класс фундаментальных последовательностей, сходящихся к $x$. Заметим, что хотя бы одна такая последовательность есть всегда, например постоянная последовательность $x,x,x,\ldots$. Теперь докажем, что $X$ всюду плотно в $X^*$. Пусть $x^*\in X^*$ и $\{x_n\}\in x^*$. Пусть $\varepsilon>0$ --- произвольное число. Так как последовательность $\{x_n\}$ фундаментальная, то существует $N\in\NN$ такое, что $\rho(x_n,x_m)<\varepsilon$ для любых $n,m>N$. С другой стороны, $$\rho(x_n,x^*)=\lim\limits_{m\rightarrow\infty}\rho(x_n,x_m)\leq\varepsilon.$$ Таким образом, произвольная окрестность точки $x^*\in X^*$ содержит точку $X$. Теперь докажем, что $X^*$ полно. Заметим, что любая последовательность $\{x_n\}$, составленная из точек $X$ сходится. Пусть теперь $\{x^*_n\}$ --- фундаментальная последовательность из $X^*$. Поскольку $X$ всюду плотно в $X^*$, то существуют точки $x_n\in X$ удовлетворяющие условию $\rho(x_n,x^*_n)<\frac{1}{n}$. Заметим, что $$\rho(x_n,x_m)\leq\rho(x_n,x^*_n)+\rho(x^*_n,x^*_m)+\rho(x^*_m,x_m)<\frac{1}{n}+\frac{1}{m}+\rho(x^*_n,x^*_m).$$ Следовательно, последовательность $\{x_n\}$ фундаментальна. Тогда она сходится к $x^*$. С другой стороны, $$\rho(x^*,x^*_n)\leq\rho(x^*,x_n)+\rho(x_n,x^*_n)<\frac{1}{n}+\rho(x_n,x^*_n).$$ Таким образом, $\{x^*_n\}$ также сходится к $x^*$.
	
	Теперь докажем единственность. Пусть есть два пополнения $X^*$ и $X^{**}$ пространства $X$. Определим отображение $f\colon X^*\rightarrow X^{**}$ следующим образом, $f(x)=x$, если $x\in X$. Пусть $x^*\in X^*$. Тогда существует последовательность $\{x_n\}$, составленная из точек $X$ такая, что $\lim\limits_{n\rightarrow\infty} x_n=x^*$. Точки $\{x_n\}$ входят в $X^{**}$ и последовательность $\{x_n\}$ фундаментальна в $X^{**}$. Пусть $\lim\limits_{n\rightarrow\infty} x_n=x^{**}$ в пространстве $X^{**}$. Положим $f(x^*)=x^{**}.$ Проверим, что $f$ является изометрией. Пусть $\rho_1$ и $\rho_2$ --- метрики в $X^*$ и $X^{**}$ соответственно. Пусть $$\lim\limits_{n\rightarrow\infty} x_n=x^*,\quad \lim\limits_{n\rightarrow\infty} y_n=y^*\text{ в } X^*,$$ $$\lim\limits_{n\rightarrow\infty} x_n=x^{**},\quad \lim\limits_{n\rightarrow\infty} y_n=y^{**}\text{ в } X^{**}.$$ Тогда $$\rho_1(x^*,y^*)=\lim\limits_{n\rightarrow\infty}\rho_1(x_n,y_n)=\lim\limits_{n\rightarrow\infty}\rho(x_n,y_n),$$ $$\rho_2(x^{**},y^{**})=\lim\limits_{n\rightarrow\infty}\rho_2(x_n,y_n)=\lim\limits_{n\rightarrow\infty}\rho(x_n,y_n).$$ Таким образом, $\rho_1(x^*,y^*)=\rho_2(x^{**},y^{**})$.
\end{proof}

\begin{example}
	Пусть $X=\QQ$. Рассмотрим стандартную метрику $\rho(x,y)=|x-y|$. Тогда пополнение $X^*=\RR$.
\end{example}

\begin{example}
	Пусть $C[a;b]$ --- пространство непрерывных функций на $[a;b]$. Рассмотрим метрику $\rho(f,g)=\max\limits_{x\in[a;b]}|f(x)-g(x)|$. Тогда пространство $C[a;b]$ --- полное пространство. Действительно, пусть $\{y_n(x)\}$ --- фундаментальная последовательность. Это означает что для любого $\varepsilon>0$ существует $N$ такое, что для любых $n,m>N$ и любого $x\in[a;b]$ выполнено $|y_n(x)-y_m(x)|<\varepsilon$. Тогда $y_n(x)\rightrightarrows y(x)$. Устремив $m$ к бесконечности в неравенстве $|y_n(x)-y_m(x)|<\varepsilon$, получаем $|y_n(x)-y(x)|\leq\varepsilon$, т.е. $\rho(y_n,y)\leq\varepsilon$. Следовательно, $\{y_n(x)\}$ сходится к $y(x)$ в смысле метрики $\rho$.
\end{example}

\section{Принцип сжимающих отображений}

\begin{definition}
	Пусть $X$ --- метрическое пространство. Отображение $A$ называется \emph{сжимающим}, если существует число $\alpha<1$ такое, что $\rho(Ax,Ay)\leq\alpha\rho(x,y)$.
\end{definition}

\begin{claim}
	\label{Sj1}
	Любое сжимающее отображение непрерывно.
\end{claim}

\begin{proof}
	Действительно, если $\rho(x,y)<\varepsilon$, то $\rho(Ax,Ay)<\varepsilon$.
\end{proof}

\begin{theorem}[принцип сжимающих отображений]
	\label{Sj2}
	Любое сжимающее отображение полного метрического пространства $X$ имеет одну и только одну неподвижную точку.
\end{theorem}

\begin{proof}
	Пусть $x_0\in X$ --- произвольная точка. Положим $$x_1=Ax_0,\quad x_2=Ax_1=A^2x_0,\ldots,\quad x_n=Ax_{n-1}=A^n x_0,\ldots.$$ Докажем, что $\{x_n\}$ --- фундаментальная последовательность. Пусть $m>n$ --- натуральные числа. Тогда $$\rho(x_n,x_m)=\rho(A^n x_0,A^n x_{m-n})\leq \alpha^n\rho(x_0,x_{m-n})\leq$$ $$\leq\alpha^n(\rho(x_0,x_1)+\rho(x_1,x_2)+\cdots+\rho(x_{m-n-1},x_{m-n}))\leq$$ $$\leq\alpha^n(\rho(x_0,x_1)+\rho(Ax_0,Ax_1)+\cdots+\rho(A^{m-n-1}x_0,A^{m-n-1}x_1))\leq$$ $$\leq\alpha^n\rho(x_0,x_1)(1+\alpha+\alpha^2+\cdots+\alpha^{m-n-1})\leq\frac{\alpha^n\rho(x_0,x_1)}{1-\alpha}.$$ Поскольку $\alpha<1$, то $\frac{\alpha^n\rho(x_0,x_1)}{1-\alpha}$ стремится к нулю при $n\rightarrow\infty$. Следовательно, $\{x_n\}$ --- фундаментальная последовательность. Пусть $x$ --- ее предел. Тогда, в силу непрерывности отображения $A$ (см. \ref{Sj1}), получаем $$Ax=A\lim\limits_{n\rightarrow\infty} x_n=\lim\limits_{n\rightarrow\infty}A x_n=\lim\limits_{n\rightarrow\infty}x_{n+1}=x.$$
	
	Докажем единственность точки $x$. Пусть есть две неподвижные точки $x$ и $y$. Тогда $$\rho(x,y)=\rho(Ax,Ay)\leq\alpha\rho(x,y).$$ Поскольку $\alpha<1$, то $\rho(x,y)=0$. Следовательно, $x=y$.
\end{proof}

Рассмотрим некоторые приложения этой теоремы.

Пусть $f(x)$ определена на $[a;b]$ и удовлетворяет условию Липшица, существует $R$ такое, что для любых $x_1,x_2\in[a;b]$ выполнено $|f(x_1)-f(x_2)|\leq K|x_1-x_2|$. Предположим, что $K<1$ и $f(x)$ отображает $[a;b]$ в себя. Тогда $f(x)$ --- сжимающее отображение. Согласно теореме \ref{Sj2}, последовательность $x_0\in [a;b]$, $x_1=f(x_0)$, $x_2=f(x_1)$,... сходится к единственному корню $x=f(x)$. В частности, если $f(x)$ имеет производную на $[a;b]$, и $|f'(x)|\leq K<1$, то из теоремы Лагранжа следует, что $|f(x_1)-f(x_2)|\leq K|x_1-x_2|$, т.е. $f(x)$ --- сжимающее отображение.

Рассмотрим отображение $A$ $n$-мерного пространства в себя, задаваемое системой линейных уравнений $$\begin{cases} y_1=a_{11}x_1+a_{12}x_2+\cdots+a_{1n}x_n+b_1 \\
		y_2=a_{21}x_1+a_{22}x_2+\cdots+a_{2n}x_n+b_2 \\
		\cdots\quad\cdots\quad\cdots                 \\
		y_n=a_{n1}x_1+a_{n2}x_2+\cdots+a_{nn}x_n+b_n\end{cases}$$
Если $A$ сжимающее отображение, то мы можем применить этот метод к решению уравнения $x=Ax$. Осталось выяснить, когда это отображение сжимающее. Ответ зависит от выбора метрики. Пусть на $\RR^n$ задана метрика $\rho(x,y)=\max\limits_i |x_i-y_i|$, где $x=(x_1,x_2,\ldots,x_n)$, $y=(y_1,y_2,\ldots,y_n)$. Пусть $x'=(x'_1,x'_2,\ldots,x'_n)$ $x''=(x''_1,x''_2,\ldots,x''_n)$, $y'=Ax'$, $y''=Ax''$. Тогда $$\rho(y',y'')=\max\limits_i |y'_i-y''_i|=\max\limits_i \left|\left(\sum\limits_{j=1}^n a_{ij}x'_j+b_i\right)-\left(\sum\limits_{j=1}^n a_{ij}x''_j+b_i\right)\right|=$$ $$=\max\limits_i \left|\sum\limits_{j=1}^n a_{ij}(x'_j-x''_j)\right|\leq\max\limits_i\sum\limits_{j=1}^n|a_{ij}||x'_j-x''_j|\leq$$ $$\leq\max\limits_i\sum\limits_{j=1}^n|a_{ij}|\max\limits_j |x'_j-x''_j|=\left(\max\limits_i\sum\limits_{j=1}^n|a_{ij}|\right)\rho(x',x'').$$ Отсюда, если $\sum\limits_{j=1}^n|a_{ij}|<1$ для любого $i$, то $A$ --- сжимающее отображение. Рассмотрим метрику $\rho(x,y)=\sum\limits_{i=1}^n |x_i-y_i|$. Тогда $$\rho(y',y'')=\sum\limits_{i=1}^n  |y'_i-y''_i|=\sum\limits_{i=1}^n  \left|\left(\sum\limits_{j=1}^n a_{ij}x'_j+b_i\right)-\left(\sum\limits_{j=1}^n a_{ij}x''_j+b_i\right)\right|=$$
$$=\sum\limits_{i=1}^n \left|\sum\limits_{j=1}^n a_{ij}(x'_j-x''_j)\right|\leq\sum\limits_{i=1}^n \sum\limits_{j=1}^n |a_{ij}| |x'_j-x''_j|=\sum\limits_{j=1}^n\left(\sum\limits_{i=1}^n |a_{ij}|\right)|x'_j-x''_j|\leq$$
$$\leq\sum\limits_{j=1}^n\left(\max\limits_j\sum\limits_{i=1}^n |a_{ij}|\right)|x'_j-x''_j|=\left(\max\limits_j\sum\limits_{i=1}^n |a_{ij}|\right)\sum\limits_{j=1}^n|x'_j-x''_j|=$$ $$=\left(\max\limits_j\sum\limits_{i=1}^n |a_{ij}|\right)\rho(x',x'').$$ Следовательно, $A$ --- сжимающее отображение, если $\sum\limits_{i=1}^n|a_{ij}|<1$ для любого $j$.
Теперь рассмотрим стандартную метрику $$\rho(x,y)=\sqrt{\sum\limits_{i=1}^n(x_i-y_i)^2}.$$ Тогда $$\rho^2(y',y'')=\sum\limits_{i=1}^n\left(\left(\sum\limits_{j=1}^n a_{ij}x'_j+b_i\right)-\left(\sum\limits_{j=1}^n a_{ij}x''_j+b_i\right)\right)^2=$$ $$=\sum\limits_{i=1}^n\left(\sum\limits_{j=1}^n a_{ij}(x'_j-x''_j)\right)^2\leq$$ $$\text{ используя неравенство Коши-Буняковского }$$ $$\leq\sum\limits_{i=1}^n\left(\sum\limits_{j=1}^n a^2_{ij}\right)\left(\sum\limits_{j=1}^n (x'_j-x''_j)^2\right)=\left(\sum\limits_{i=1}^n\sum\limits_{j=1}^n a^2_{ij}\right)\rho(x',x'').$$ Отсюда, получаем условие $\sum\limits_{i=1}^n\sum\limits_{j=1}^n a^2_{ij}<1$.

Рассмотрим еще один важный пример. Пусть дано дифференциальное уравнение $y'=f(x,y)$ с начальным условием $y(x_0)=y_0$. Предположим, что $f(x,y)$ --- определена и непрерывна в некоторой области $G$, содержащей точку $(x_0,y_0)$. Более того, $f(x,y)$ удовлетворяет условию Липшица по $y$, т.е. существует $K$ такое, что $$|f(x,y_1)-f(x,y_2)|\leq K|y_1-y_2|$$ для любых $(x,y_1), (x,y_2)\in G$. Поскольку $f(x,y)$ --- непрерывно, то $|f(x,y)|<M$ в области $G'\subset G$. Теперь рассмотрим прямоугольную область $G''=\{[x_0-d,x_0+d]\times[y_0-Md,y_0+Md]\}\subset G'$. Более того, $dK<1$. Пусть $C^*$ --- пространство непрерывных функций на $[x_0-d,x_0+d]$ таких, что $|\varphi(x)-y_0|\leq Md$. Зададим метрику на $C^*$, $$\rho(\varphi_1,\varphi_2)=\max\limits_{x\in[x_0-d;x_0+d]}|\varphi_1(x)-\varphi_2(x)|.$$ Заметим, что пространство $C^*$ полно (оно является замкнутым подмножеством полного пространства).
Рассмотрим отображение $A$, определяемое $$\psi(x)=A(\varphi)=y_0+\int\limits_{x_0}^x f(t,\varphi(t))dt.$$ Заметим, что $A$ переводит $C^*$ в себя. Действительно, пусть $\varphi\in C^*$. Тогда $$|\psi(x)-y_0|=\left|\int\limits_{x_0}^x f(t,\varphi(t))dt\right|\leq M d.$$ Пусть $\varphi_1 \varphi_2\in C^*$ и $\psi_1=A(\varphi_1), \psi_2=A(\varphi_2)$. Тогда $$|\psi_1(x)-\psi_2(x)|=\left|\int\limits_{x_0}^x (f(t,\varphi_1(t))-f(t,\varphi_2(t)))dt\right|\leq$$ $$\leq\int\limits_{x_0}^x |f(t,\varphi_1(t))-f(t,\varphi_2(t))|dt\leq\int\limits_{x_0}^x K|\varphi_1(t)-\varphi_2(t)|dt\leq$$ $$\leq Kd\max\limits_{x\in[x_0-d;x_0+d]}|\varphi_1(x)-\varphi_2(x)|=Kd\rho(\varphi_1,\varphi_2).$$ Поскольку $Kd<1$, то $A$ сжимающее отображение. Следовательно, существует единственная функция $\varphi(x)$ такая, что $$\varphi(x)=y_0+\int\limits_{x_0}^x f(t,\varphi(t)dt.$$ Тогда $\varphi(x)$ является решением дифференциального уравнения $y'=f(x,y)$ с начальным условием $y(x_0)=y_0$.

\begin{remark}
	Предыдущий пример является доказательством теоремы Коши о существовании и единственности решения задачи Коши для обыкновенных дифференциальных уравнений.
\end{remark}

\section{Компактные пространства}

Одной из основных теорем в анализе является теорема Гейне--Бореля: из любого покрытия отрезка $[a;b]$ открытыми множествами можно выбрать конечное подпокрытие.

Теперь мы примем утверждение теоремы за определение компактных пространств.

\begin{definition}
	Топологическое пространство $X$ называется \emph{компактным}, если любое покрытие $X$ открытыми множествами содержит конечное подпокрытие. Компактное хаусдорфово пространство называется \emph{компактом}.
\end{definition}

\begin{definition}
	Пусть $\{A_{\alpha}\}$ --- некоторая система подмножеств в $X$. Будем называть $\{A_{\alpha}\}$ \emph{центрированной}, если любое конечное пересечение $\bigcap\limits_{i=1}^n A_i$ не пусто.
\end{definition}

\begin{theorem}
	\label{KomPr1}
	Топологическое пространство $X$ компактно тогда и только тогда, когда каждая центрированная система его замкнутых подмножеств имеет непустое пересечение.
\end{theorem}

\begin{proof}
	Пусть $\{A_{\alpha}\}$ --- центрированная система замкнутых подмножеств $X$ и пусть $X$ компактно. Множества $B_{\alpha}=X\setminus A_{\alpha}$ --- открыты. Поскольку для любого конечного набора $A_i$ $\bigcap\limits_{i=1}^n A_i\neq \emptyset$, то $B_i$ не покрывают все пространство. Тогда $B_{\alpha}$ не покрывают все пространство. Следовательно, $\bigcap A_{\alpha}\neq \emptyset$. Обратно, пусть $\{B_{\alpha}\}$ --- открытое покрытие пространства $X$. Пусть $A_{\alpha}=X\setminus B_{\alpha}$. Поскольку $B_{\alpha}$ --- покрытие, то $\bigcap A_{\alpha}= \emptyset$. Следовательно, система $\{A_{\alpha}\}$ не может быть центрированной. Тогда существуют множества $A_1,A_2,\ldots A_n$ такие, что $\bigcap\limits_{i=1}^n A_i= \emptyset$. Отсюда, $\bigcup\limits_{i=1}^n B_i=X.$
\end{proof}

\begin{corollary}
	\label{KomPr2}
	Замкнутое подмножество компактного пространства компактно.
\end{corollary}

\begin{proof}
	Пусть $Y\subset X$ --- замкнутое подмножество. Пусть $\{A_{\alpha}\}$ --- центрированная система замкнутых подмножеств в $Y$. Тогда $\{A_{\alpha}\}$ --- центрированная система замкнутых подмножеств и в $X$. Следовательно, согласно теореме \ref{KomPr1}, $\bigcap A_{\alpha}\neq \emptyset$. Отсюда, снова по теореме \ref{KomPr1}, $Y$ компактно.
\end{proof}

\begin{corollary}
	\label{KomPr3}
	Замкнутое подмножество компакта есть компакт.
\end{corollary}

\begin{proof}
	Очевидно, что подпространство хаусдорфова пространства есть хаусдорфово пространство. Теперь из следствия \ref{KomPr2} следует наше утверждение.
\end{proof}

\begin{theorem}
	\label{KomPr4}
	Компакт замкнут в любом содержащем его метрическом пространстве.
\end{theorem}

\begin{proof}
	Пусть $K$ --- компакт в метрическом пространстве $X$, и пусть $y\not\in K$. Тогда для любого $x\in K$ существуют окрестности $U_x$ и $V_x$ точек $x$ и $y$ соответственно такие, что $U_x\cap V_x=\emptyset$. Заметим, что $U_x$ покрывают компакт $K$. Тогда можно выбрать конечное подпокрытие $U_{x_1},U_{x_2},\ldots, U_{x_n}$. Пусть $V=V_{x_1}\cap V_{x_2}\cap\ldots\cap V_{x_n}$. Тогда $V$ не пересекается с $U_{x_1}\cup U_{x_2}\cup\ldots\cup U_{x_n}$, а следовательно и с $K$. Таким образом $y$ не может быть точкой прикосновения. Следовательно, $K$ замкнуто.
\end{proof}

\begin{definition}
	Пространство $X$ называется \emph{счетно компактным}, если в любом счетном покрытия $X$ есть конечное подпокрытие.
\end{definition}

Очевидно, что компактные пространства счетно компактны. Рассмотрим связь компактных и метрических пространств.

\begin{theorem}
	\label{KomPr5}
	Если $X$ --- счетно компактное метрическое пространство, то любое бесконечное подмножество имеет предельную точку.
\end{theorem}

\begin{proof}
	Предположим, что в $X$ есть бесконечное множество не имеющее ни одной предельной точки, то в $X$ существует счетное множество $$A=\{x_1,x_2,\ldots\}$$ не имеющее ни одной предельной точки. Рассмотрим $$A_n=\{x_{n},x_{n+1},\ldots\}.$$ Заметим, что $A_n$ --- замкнутые множества. Положим $B_n=X\setminus A_n$. Поскольку $\bigcap\limits_{i=1}^{\infty} A_i=\emptyset$, то $B_n$ --- счетное покрытие пространства $X$. С другой стороны, мы видим, что из этого покрытия нельзя выбрать конечное подпокрытие. Противоречие.
\end{proof}

\begin{remark}
	Заметим, что здесь мы фактически повторили рассуждения, использовавшиеся при доказательстве теоремы \ref{KomPr1}.
\end{remark}

\begin{corollary}
	\label{KomPr5-1}
	Если $X$ --- счетно компактное метрическое пространство, то оно полно.
\end{corollary}

\begin{definition}
	Пусть $X$ --- метрическое пространство, и пусть $M\subset X$ --- его подмножество. Множество $A$ называется \emph{$\varepsilon$-сетью} для $M$, если для любого $x\in M$ существует $a\in A$ такой, что $\rho(x,a)<\varepsilon$. Множество $M$ называется \emph{вполне ограниченным}, если для любого $\varepsilon>0$ существует конечная $\varepsilon$-сеть.
\end{definition}

\begin{remark}
	Заметим, что вполне ограниченные множества ограничены. Обратное неверно.
\end{remark}


\begin{theorem}
	\label{KomPr6}
	Если метрическое пространство $X$ счетно компактно, то оно вполне ограниченно.
\end{theorem}

\begin{proof}
	Пусть $X$ не вполне ограниченно. Тогда для некоторого $\varepsilon_0>0$ в $X$ не существует конечной $\varepsilon_0$-сети. Пусть $x_1\in X$ --- произвольная точка. Поскольку $\{x_1\}$ не является $\varepsilon_0$-сетью, то существует $x_2\in X$ такой, что $\rho(x_1,x_2)>\varepsilon_0$. Поскольку $\{x_1,x_2\}$ не является $\varepsilon_0$-сетью, то существует $x_3\in X$ такой, что $\rho(x_1,x_3)>\varepsilon_0$, $\rho(x_2,x_3)>\varepsilon_0$ и т.д. Таким образом, мы получили последовательность $$\{x_1,x_2,\ldots,x_n,\ldots\}$$ такую, что $\rho(x_n,x_m)>\varepsilon_0$ для любых $n\neq m$. Тогда эта последовательность не имеет ни одной предельной точки. Противоречие.
\end{proof}

\begin{lemma}
	\label{LemKomPr}
	Пусть $X$ --- топологическое пространство со счетной базой, т.е. $X$ удовлетворяет второй аксиоме счетности. Тогда из всякого открытого покрытия можно выбрать конечное или счетное подпокрытие.
\end{lemma}

\begin{proof}
	Пусть $\{U_{\alpha}\}$ --- открытое покрытие. Тогда для любого $x\in X$ существует $U_{\alpha}$ такой, что $x\in U_{\alpha}$. Пусть $\{B_n\}$ --- счетная база. Тогда существует $B_n(x)$ такой, что $x\in B_n(x)\subset U_{\alpha}$. Выбрав для каждого $B_n(x)$ одно из $U_{\alpha}$, мы получим конечное или счетное подпокрытие.
\end{proof}

\begin{theorem}
	\label{MetS}
	Пусть $X$ --- вполне ограниченное метрическое пространство. Тогда $X$ --- топологическое пространство со счетной базой, т.е. $X$ удовлетворяет второй аксиоме счетности.
\end{theorem}

\begin{proof}
	Для каждого $n\in\NN$ построим конечную $\frac{1}{n}$-сеть $\{x_{n1},x_{n2},\ldots,x_{nm_n}\}$. Для каждого $x_{ni}$ возьмем шар $B(x_{ni},\frac{1}{n})$ радиуса $\frac{1}{n}$ с центром в $x_{ni}$. Их объединение и будет счетной базой.
\end{proof}

\begin{corollary}
	\label{KomPr7}
	Всякое счетно компактное метрическое пространство компактно.
\end{corollary}

\begin{proof}
	Пусть $X$ --- счетно компактное пространство. Тогда, по теореме \ref{KomPr6}, оно вполне ограничено. По теореме \ref{MetS}, $X$ удовлетворяет второй аксиоме счетности. Согласно лемме \ref{LemKomPr} из любого покрытия $X$ можно выбрать счетное подпокрытие, а следовательно и конечное подпокрытие.
\end{proof}

Рассмотрим еще одну "компактность".

\begin{definition}
	Пусть $X$ --- метрическое пространство. Мы говорим, что $X$ секвенциально компактно, если из любой последовательности его точек можно выбрать сходящуюся подпоследовательность.
\end{definition}

\begin{remark}
	Выше (см. \ref{KomPr5}) мы доказали, что счетно компактное пространство секвенциально компактно. На самом деле верно и обратное утверждение.
\end{remark}

\begin{theorem}
	\label{KomPr8}
	Пусть $X$ --- секвенциально компактное пространство. Тогда любая непрерывная функция $f\colon X\rightarrow\RR$ ограничена и достигает своего максимума и минимума.
\end{theorem}

\begin{proof}
	Предположим, что $f(x)$ не ограничена. Тогда существует неограниченная возрастающая (или убывающая) последовательность $f(x_n)$. Поскольку $X$ секвенциально компактно, то из последовательности $\{x_n\}$ можно выбрать сходящуюся подпоследовательность $\{x_{n_k}\}$. Пусть $x_0$ --- ее предел. Тогда, в силу непрерывности $f(x)$, $f(x_{n_k})$ сходится к $f(x_0)$. Следовательно, $f(x)$ ограничена. Пусть $A=\sup f(x)$. Тогда существует последовательность $f(x_n)$ сходящуюся к $A$. Выберем в $\{x_n\}$ сходящуюся подпоследовательность$\{x_{n_k}\}$. Пусть $x_0$ --- ее предел. Тогда, в силу непрерывности $f(x)$, $f(x_{n_k})$ сходится к $f(x_0)$, т.е. $f(x_0)=A$. Аналогично доказывается для минимума.
\end{proof}

\begin{theorem}
	\label{KomPr9}
	Метрическое пространство $X$ компактно тогда и только тогда, когда оно секвенциально компактно.
\end{theorem}

\begin{proof}
	В одну сторону мы уже доказали. Докажем, в другую. Пусть $X$ секвенциально компактно. Предположим, что существует открытое покрытие $\{U_{\alpha}\}$ из которого нельзя выбрать конечное подпокрытие. Рассмотрим функцию $$f(x)=\sup\{r\in\RR\mid\exists U_{\alpha},  B_r(x)\subset U_{\alpha}\},$$ где $B_r(x)$ --- шар радиуса $r$ с центром в $x$. Докажем непрерывность $f(x)$. Более того, мы докажем 1-липшевость этой функции, т.е. для любых $x,y$ выполнено $|f(x)-f(y)|\leq\rho(x,y)$. Предположим противное, т.е. существуют $x,y\in X$ такие, что $|f(x)-f(y)|>\rho(x,y)$. Можно считать, что $f(x)>f(y)$. Тогда $f(x)>f(y)+\rho(x,y)$. Выберем $a,b\in\RR$ так, что $b>f(y)$, $a<f(x)$ и $a>b+\rho(x,y)$. Тогда $B_b(y)\subset B_a(x)$ и существует $U_{\alpha}$  такое, что $B_a(x)\subset U_{\alpha}$. Отсюда, $B_b(x)\subset U_{\alpha}$. Но тогда $f(y)\geq b$. Противоречие. Следовательно, функция $f(x)$ непрерывна. Тогда, по теореме \ref{KomPr8}, $f(x)$ достигает минимума. Пусть $m=\min f(x)$. Положим $r=\frac{m}{2}$. Пусть $x_1\in X$. Тогда существует $U_1\in\{U_{\alpha}\}$ такое, что $x_1\in U_1$ и $B_r(x_1)\subset U_1$. Выберем $x_2\in(X\setminus U_1)$. Тогда существует $U_2\in\{U_{\alpha}\}$ такое, что $x_2\in U_2$ и $B_r(x_2)\subset U_2$. Если мы выбрали $x_1,x_2,\ldots,x_n$ и $U_1,U_2,\ldots U_n$, выберем $x_{n+1}\in X\setminus\bigcup\limits_{i=1}^n U_i$. Тогда существует $U_{n+1}\in\{U_{\alpha}\}$ такое, что $x_{n+1}\in U_{n+1}$ и $B_r(x_{n+1})\subset U_{n+1}$. Таким образом, мы получили последовательность $\{x_n\}$ в которой $\rho(x_n,x_m)\geq r$, но такая последовательность не содержит сходящейся подпоследовательности.
\end{proof}

\begin{theorem}
	\label{KomPr10}
	Метрическое пространство $X$ компактно тогда и только тогда, когда оно полно и вполне ограниченно.
\end{theorem}

\begin{proof}
	Мы уже доказали, что если $X$ компактно, то оно полно и вполне ограниченно. Предположим, что $X$ полно и вполне ограниченно. Мы докажем, что $X$ секвенциально компактно. Тогда из теоремы \ref{KomPr9} будет следовать, что $X$ компактно. Пусть $\{x_n\}$ последовательность точек из $X$. Рассмотрим $1$-сеть и множество замкнутых шаров, радиуса $1$ с центрами в точках сети. Поскольку эти шары покрывают все пространство и их конечное число, то существует шар $B_1$ содержащий бесконечное множество точек последовательности $\{x_n\}$. Обозначим это множество $A_1$. Выберем одну из точек $x_{n_1}\in A_1$. Далее возьмем $\frac{1}{2}$-сеть. Рассмотрим множество замкнутых шаров, радиуса $\frac{1}{2}$ с центрами в точках сети. Поскольку эти шары покрывают все пространство и их конечное число, то существует шар $B_2$ содержащий бесконечное множество точек $A_1$. Обозначим это множество $A_2$. Выберем одну из точек $x_{n_2}\in A_2$. Далее возьмем $\frac{1}{4}$-сеть. Выберем $B_3$, содержащий бесконечное множество $A_3$ точек $A_2$ и $x_{n_3}\in A_3$ и т.д. Таким образом, мы получили последовательность $\{x_{n_i}\}$. Эта последовательность является фундаментальной, поскольку $\rho(x_n,x_m)\leq\frac{1}{2^{\min(n,m)}}$. Следовательно, у этой последовательности существует предел.
\end{proof}

\begin{definition}
	Множество $M$, лежащее в некотором метрическом пространстве $X$ называется \emph{относительно компактным}, если его замыкание компактно.
\end{definition}

\begin{corollary}
	\label{KomPr10-1}
	Пусть $X$ --- полное метрическое пространство. Для того, чтобы множество $M\subset X$ было относительно компактно необходимо и достаточно, чтобы оно было вполне ограничено.
\end{corollary}


\begin{definition}
	Пусть $f\colon X\rightarrow\RR$ --- функция на метрическом пространстве. Мы говорим, что $f$ \emph{равномерно непрерывна}, если для любого $\varepsilon>0$ существует $\delta$ такое, что для любых $x_1,x_2\in X$ с условием $\rho(x_1,x_2)<\delta$, выполнено неравенство $|f(x_1)-f(x_2)|<\varepsilon$.
\end{definition}

\begin{theorem}
	\label{KomPr11}
	Непрерывная функция на компактном метрическое пространство $X$ равномерно непрерывна.
\end{theorem}

\begin{proof}
	Пусть $f\colon X\rightarrow\RR$ --- непрерывная функция. Предположим, что $f(x)$ не равномерно непрерывна. Тогда существует $\varepsilon>0$ такое, что при любом $n$ существуют $x_n,x'_n\in X$ с условием $\rho(x_n,x'_n)<\frac{1}{n}$, для которых $|f(x_n)-f(x'_n)|>\varepsilon$. Из последовательности $\{x_n\}$ можно выбрать сходящуюся подпоследовательность $\{x_{n_k}\}$. Пусть $x$ --- ее предел. Тогда последовательность $\{x'_{n_k}\}$ также сходится к $x$. С другой стороны, поскольку $|f(x_{n_k})-f(x'_{n_k})|>\varepsilon$, то $$|f(x)-f(x_{n_k})|>\frac{\varepsilon}{2},\quad |f(x)-f(x'_{n_k})|>\frac{\varepsilon}{2},$$ что противоречит непрерывности $f(x)$.
\end{proof}


Рассмотрим метрическое пространство $C[a;b]$.

\begin{definition}
	Семейство функций $\Phi$, определенных на $[a;b]$ называется \emph{равномерно ограниченным}, если существует $K\in\RR$, что для любого $\varphi\in\Phi$ и любого $x\in[a;b]$ выполнено $|\varphi(x)|<K$. Семейство функций $\Phi$, определенных на $[a;b]$ называется \emph{равномерно непрерывным}, если для любого $\varepsilon>0$ существует $\delta$ такое, что для всех $x_1,x_2\in[a;b]$ с условием $|x_1-x_2|<\delta$, и любой $\varphi\in\Phi$ выполнено неравенство $|\varphi(x_1)-\varphi(x_2)|<\varepsilon$.
\end{definition}

\begin{theorem}[теорема Арцела]
	\label{Arz}
	Для того чтобы семейство непрерывных функций $\Phi$, определенных на $[a;b]$, было относительно компактно в $C[a;b]$, необходимо и достаточно, чтобы это семейство было равномерно ограничено и равномерно непрерывно.
\end{theorem}

\begin{proof}
	Необходимость. Пусть семейство непрерывных функций $\Phi$ компактно. Мы можем считать, что $\Phi$ замкнуто. Тогда, согласно следствию \ref{KomPr10-1}, для любого $\varepsilon$ в семействе $\Phi$ существует конечная $\frac{\varepsilon}{3}$-сеть $\varphi_1,\varphi_2,\ldots,\varphi_n$. Заметим, что все функции $\varphi_i$ ограничены, т.е. существуют $K_i$ такие, что $|\varphi(x)|<K_i$. Положим $K=\max K_i+\frac{\varepsilon}{3}$. Заметим, что для любой $\varphi\in\Phi$ существует $\varphi_i$ такая, что $$\rho(\varphi,\varphi_i)=\max\limits_{x\in[a;b]}|\varphi(x)-\varphi_i(x)|<\frac{\varepsilon}{3}.$$ Отсюда, для любого $x\in[a;b]$ $$|\varphi(x)|<|\varphi_i(x)|+\frac{\varepsilon}{3}<K_i+\frac{\varepsilon}{3}\leq K.$$ Таким образом, $\Phi$ равномерно ограничено. Поскольку все функции $\varphi_i$ непрерывны, то они равномерно непрерывны. Тогда существуют $\delta_i$ такие, что для любых $x_1,x_2\in[a;b]$ с условием $|x_1-x_2|<\delta_i$ выполнено неравенство $|\varphi(x_1)-\varphi(x_2)|<\frac{\varepsilon}{3}$. Положим $\delta=\min\delta_i$. Пусть $\varphi\in\Phi$ --- любая функция. Тогда существует $\varphi_i$ такая, что $$\rho(\varphi,\varphi_i)=\max\limits_{x\in[a;b]}|\varphi(x)-\varphi_i(x)|<\frac{\varepsilon}{3}.$$ Отсюда, для любых $x_1,x_2\in[a;b]$ с условием $|x_1-x_2|<\delta$ имеем $$|\varphi(x_1)-\varphi(x_2)|=|\varphi(x_1)-\varphi_i(x_1)+\varphi_i(x_1)-\varphi_i(x_2)+\varphi_i(x_2)-\varphi(x_2)|\leq$$ $$|\varphi(x_1)-\varphi_i(x_1)|+|\varphi_i(x_1)-\varphi_i(x_2)|+|\varphi_i(x_2)-\varphi(x_2)|<\frac{\varepsilon}{3}+\frac{\varepsilon}{3}+\frac{\varepsilon}{3}=\varepsilon.$$ Таким образом, $\Phi$ равномерно непрерывно.
	
	Достаточность. Пусть $\Phi$ равномерно ограничено и равномерно непрерывно. Следовательно существует $K\in \RR$ такое, что $|\varphi(x)|<K$ для любых $x\in[a;b]$, $\varphi\in\Phi$. Пусть $\varepsilon>0$ --- произвольное число. Тогда существует $\delta$ такое, что для любых $\varphi\in\Phi$ и $x_1,x_2\in[a;b]$ с условием $|x_1-x_2|<\delta$ выполнено неравенство $|\varphi(x_1)-\varphi(x_2)|<\varepsilon$. Разобьём отрезок $[a;b]$ точками $a=x_0<x_1<\cdots<x_k=b$ на интервалы длины меньше $\delta$. Отрезок $[-K;K]$ мы разобьем точками $-K=y_0<y_1<\cdots<y_m=K$ на интервалы длины меньше $\varepsilon$. Рассмотрим множество ломаных $\psi_n$, проходящих через точки $$(x_0,y_{j_0}),(x_1,y_{j_1}),(x_2,y_{j_2}),\ldots,(x_k,y_{j_k}),$$ где $j_0,j_1,\ldots, j_k$ --- произвольные целые числа от $0$ до $m$. Заметим, что таких ломаных конечное число. Пусть $\varphi\in\Phi$ --- произвольная функция, и $x\in[a;b]$ --- произвольная точка на отрезке. Выберем ломаную $\psi$ так, что $|\varphi(x_i)-\psi(x_i)|<\varepsilon$ для любого $x_i$. Выберем ближайшую к $x$ слева точку $x_i$. Тогда $$|\varphi(x)-\psi(x)|=|\varphi(x)-\varphi(x_i)+\varphi(x_i)-\psi(x_i)+\psi(x_i)-\psi(x)|\leq$$ $$\leq|\varphi(x)-\varphi(x_i)|+|\varphi(x_i)-\psi(x_i)|+|\psi(x_i)-\psi(x)|\leq$$ $$\varepsilon+\varepsilon+|\psi(x_i)-\psi(x)|=2\varepsilon+|\psi(x_i)-\psi(x_{i+1})|.$$ Здесь мы использовали линейность $\psi$ на отрезке $[x_i;x_{i+1}]$. Заметим, что $$|\psi(x_i)-\psi(x_{i+1})|=|\psi(x_i)-\varphi(x_i)+\varphi(x_i)-\varphi(x_{i+1})+\varphi(x_{i+1})-\psi(x_{i+1})|\leq$$ $$\leq|\psi(x_i)-\varphi(x_i)|+|\varphi(x_i)-\varphi(x_{i+1})|+|\varphi(x_{i+1})-\psi(x_{i+1})|<\varepsilon+\varepsilon+\varepsilon=3\varepsilon.$$ Таким образом, $$|\varphi(x)-\psi(x)|<5\varepsilon.$$ Мы получили $5\varepsilon$-сеть. В силу произвольности $\varepsilon$, $\Phi$ вполне ограниченное множество, а следовательно, $\Phi$ относительно компактно в $C[a;b]$.
\end{proof}


\chapter{Нормированные пространства}

\section{Линейные пространства}

\begin{definition}
	Пусть на множестве $L$ заданы операции сложения и умножения на число, удовлетворяющие следующим свойствам.
	\begin{enumerate}
		\item $(x+y)+z=x+(y+z)$ $\forall x,y,z\in L$;
		\item $x+y=y+x$ $\forall x,y\in L$;
		\item существует элемент $0\in L$ такой, что $x+0=x$ $\forall x\in L$;
		\item для любого $x\in L$ существует элемент $(-x)\in L$ такой, что $x+(-x)=0$;
		\item для любых чисел $\alpha$, $\beta$ и любого $x\in L$ выполнено $(\alpha+\beta)x=\alpha x+\beta x$;
		\item для любого числа $\alpha$ и любых $x,y\in L$ выполнено $\alpha(x+y)=\alpha x+\alpha y$;
		\item для любых чисел $\alpha$, $\beta$ и любого $x\in L$ выполнено $\alpha(\beta x)=(\alpha\beta) x$;
		\item $1\cdot x=x$ $\forall x\in L$.
	\end{enumerate}
	Тогда $L$ называется \emph{линейным пространством}.
\end{definition}


\begin{remark}
	Здесь $\alpha$ и $\beta$ принадлежат некоторому полю. Мы будем рассматривать пространства над вещественными и комплексными числами.
\end{remark}

\begin{definition}
	Элементы $x_1,x_2,\ldots, x_n\in L$ называются \emph{линейно зависимыми}, если существуют такие $a_1,a_2,\ldots, a_n$, не все равные нулю, что $$a_1x_1+a_2x_2+\cdots+a_nx_n=0.$$
\end{definition}

\begin{definition}
	Если в пространстве $L$ существуют $n$ линейно независивых элементов, а любые $n+1$ элементов линейно зависимы, то говорят, что $L$ имеет \emph{размерность} $n$. Если в $L$ можно найти систему из произвольного конечного числа векторов, то говорят, что $L$ \emph{бесконечномерно}. \emph{Базисом} $n$-мерного пространства $L$ называется любая система из $n$ линейно независимых элементов.
\end{definition}

\begin{definition}
	Непустое подмножество $L'$ линейного пространства $L$ называется \emph{подпространством}, если $L'$ --- является пространством, т.е. если $x,y\in L'$, то $\alpha x+\beta y\in L'$ для любых $\alpha,\beta$.
\end{definition}

Заметим, что линейное пространство является модулем над полем. Тогда, если $L'$ --- подпространство $L$, мы можем рассмотреть фактормодуль $L/L'$, который также будет линейным пространством. Пространство $L/L'$ называется \emph{факторпространством}. Размерность факторпространства $L/L'$ называется \emph{коразмерностью} подпространства $L'$ в пространстве $L$.

\begin{definition}
	Пусть $L$ --- линейное пространство над полем $K$. Отображение $f\colon V\rightarrow K$ называется \emph{линейным функционалом}, если
	\begin{enumerate}
		\item $f(x+y)=f(x)+f(y)$ $\forall x,y\in L$;
		\item $f(\alpha x)=\alpha f(x)$ $\forall x\in L,\alpha\in K$.
	\end{enumerate}
\end{definition}

\begin{definition}
	Пусть на линейном пространстве $L$ задан линейный функционал $f$. Множество $$L_f=\{x\mid x\in L, f(x)=0\}$$ называется \emph{подпространством нулей} (или \emph{ядром}) функционала $f$.
\end{definition}

Легко проверяется, что $L_f$ --- подпространство. Действительно, если $x,y\in L_f$, $\alpha\in K$, то $f(x+y)=f(x)+f(y)=0$, $f(\alpha x)=\alpha f(x)=0$.

\begin{claim}
	\label{LP1}
	Пусть $f$ --- линейный функционал, отличный от тождественного нуля. Тогда подпространство $L_f$ имеет коразмерность один.
\end{claim}

\begin{proof}
	Пусть $x_0\in L$ и $f(x_0)\neq 0$. Умножив при необходимости на $\frac{1}{f(x_0)}$, мы можем считать, что $f(x_0)=1$. Тогда $x-f(x) x_0\in L_f$. Действительно, $$f(x-f(x) x_0)=f(x)-f(x)f(x_0)=f(x)-f(x)=0.$$ Таким образом, для любого элемента $x\in L$ имеем представление $x=ax_0+y$, где $y\in L_f$. Это представление единственно. Действительно, предположим, что есть два представления $$x=a x_0+y,\quad x=a' x_0+y',\quad y,y'\in L_f.$$ Тогда $(a-a')x_0+(y-y')=0$. Если $a=a'$, то $y=y'$. Предположим, что $a\neq a'$. Тогда $$x_0=\frac{1}{a-a'}(y'-y)\in L_f.$$ Противоречие. Пусть $x,y\in L$.  Заметим, что $x$ и $y$ лежат в одном смежном классе тогда и только тогда, когда $f(x)=f(y)$. Действительно, если $x$ и $y$ лежат в одном смежном классе тогда и только тогда, когда $x-y\in L_f$. С другой стороны, $f(x-y)=f(x)-f(y)$. Поскольку всякий смежный класс определяется своим представителем, то в качестве такого можно взять $a x_0$. Отсюда видно, что пространство $L/L_f$ одномерно.
\end{proof}

Пусть на пространстве $L$ заданы два линейных функционала $f,g$. Мы можем определить их сумму, как $$(f+g)(x)=f(x)+g(x).$$ Не трудно заметить, что $f+g$ также линейный функционал. Мы можем определить умножение функционала $f$ на число $\alpha$ как $(\alpha f)(x)=\alpha f(x)$. Очевидно, что эти операции удовлетворяют аксиомам линейного пространства.

\begin{definition}
	Множество всех функционалов на данном линейном пространстве $L$ обозначается $L^*$ и называется \emph{сопряженным линейным пространством}.
\end{definition}

\begin{remark}
	Элементы пространства $L^*$ часто называют \emph{ковекторами}.
\end{remark}

Рассмотрим сначала конечномерный случай. Пусть $e_1,e_2,\ldots, e_n$ --- базис пространство $L$. Числа $$a_1=f(e_1),\quad a_2=f(e_2),\ldots\ldots,\quad a_n=f(e_n)$$ называются \emph{коэффициентами} функционала $f$ в базисе $e_1,e_2,\ldots, e_n$. Для любого вектора $$x=x_1 e_1+ x_2 e_2+\cdots+x_n e_n,$$ в силу линейности $f$, имеем $$f(x)=x_1 a_1+ x_2 a_2+\cdots+x_n a_n.$$ Таким образом, всякий линейный функционал однозначно определяется своими коэффициентами в базисе $e_1,e_2,\ldots, e_n$. Пусть $f_1,f_2,\ldots, f_n$ --- множество функционалов таких, что $$f_i(e_j)=\begin{cases}0,\quad i\neq j \\
		1,\quad i=j.\end{cases}$$

\begin{claim}
	\label{LP2}
	Функционалы $f_1,f_2,\ldots, f_n$ образуют базис в пространстве $L^*$. Более того, коэффициенты функционала являются его координатами.
\end{claim}

\begin{proof}
	Пусть задан линейный функционал $f$, и $a_1,a_2,\ldots, a_n$ --- его коэффициенты. Пусть $x=x_1 e_1+ x_2 e_2+\cdots+x_n e_n$. Тогда $$(a_1 f_1+ a_2 f_2+\cdots+ a_n f_n)(x)=(a_1 f_1+\cdots+ a_n f_n)(x_1 e_1+ \cdots+x_n e_n)=$$ $$=x_1(a_1 f_1+\cdots+ a_n f_n)(e_1)+\cdots+x_n(a_1 f_1+\cdots+ a_n f_n)(e_n)=$$ $$=x_1 a_1+ x_2 a_2+\cdots+x_n a_n.$$ Осталось проверить линейную независимость $f_1,f_2,\ldots, f_n$. Предположим, что существуют $\lambda_1,\lambda_2,\ldots, \lambda_n$ не все равные нулю, что $$\lambda_1f_1+\lambda_2f_2+\cdots+\lambda_n f_n=0.$$ Пусть $\lambda_i\neq 0$. Тогда $$0=(\lambda_1f_1+\lambda_2f_2+\cdots+\lambda_n f_n)(e_i)=\lambda_if_i(e_i)=\lambda_i.$$ Противоречие.
\end{proof}

\begin{corollary}
	\label{LP3}
	Для конечномерных пространств выполнено $\dim L=\dim L^*$.
\end{corollary}

Рассмотрим пространство $(L^*)^*$ сопряженное к сопряженному. Пусть $x\in L$. Тогда $x$ можно рассматривать, как линейный функционал на $L^*$. Действительно, $x(f)=f(x)$. Линейность очевидна. Тогда существует естественное отображение $L\rightarrow (L^*)^*$, которое является изоморфизмом в конечномерном случае.

\begin{definition}
	Пусть $x,y\in L$. Множество точек вида $\alpha x+(1-\alpha)y$, где $0\leq \alpha\leq 1$ называется \emph{(замкнутым) отрезком}. Множество $M$ называется \emph{выпуклым}, если для любых $x,y\in M$ отрезок, соединяющий их, лежит в $M$.
\end{definition}

Само $L$ очевидно является выпуклым множеством.

\begin{claim}
	\label{Vyp1}
	Пересечение любого числа выпуклых множеств есть выпуклое множество.
\end{claim}

\begin{proof}
	Пусть $M=\bigcap\limits_{\alpha} M_{\alpha}$ и все $M_{\alpha}$ --- выпуклые множества. Если $x,y\in M$, то $x,y\in M_{\alpha}$, для любого $\alpha$. Тогда отрезок $xy$ лежит во всех $M_{\alpha}$. Следовательно, отрезок $xy$ лежит в $M$.
\end{proof}

\begin{definition}
	Пусть $A$ --- множество в $L$. Наименьшее выпуклое множество, содержащие $A$, называется \emph{выпуклой оболочкой} множества $A$. Очевидно, что выпуклой оболочкой является пересечение всех выпуклых множеств, содержащих $A$.
\end{definition}

\begin{example}
	Рассмотрим один важный пример выпуклой оболочки. Пусть $x_1,x_2,\ldots,x_{n+1}$ --- точки пространства $L$. Будем говорить, что эти точки находятся в общем положении, если они не содержаться ни в каком $(n-1)$-мерном подпространстве. Выпуклая оболочка точек $x_1,x_2,\ldots,x_{n+1}$, находящихся в общем положении, называется $n$-мерным \emph{симплексом}.
\end{example}

\begin{definition}
	Неотрицательный функционал $p$, определенный на вещественном линейном пространстве $L$, называется \emph{выпуклым}, если
	\begin{enumerate}
		\item $p(x+y)\leq p(x)+p(y)$ $\forall x,y\in L$;
		\item $p(\alpha x)=\alpha p(x)$ $\forall x\in L$, $\alpha\geq 0$.
	\end{enumerate}
	Если выполнено $p(\alpha x)=|\alpha| p(x)$, то $p(x)$ называется \emph{полунормой}.
\end{definition}

\begin{claim}
	\label{Vyp2}
	Пусть $p(x)$ --- выпуклый функционал на линейном пространстве $L$ и $k$ --- положительное число. Тогда множество $E=\{x\mid p(x)\leq k\}$ выпукло.
\end{claim}

\begin{proof}
	Пусть $x,y\in E$. Тогда $$p(\alpha x+(1-\alpha)y)\leq p(\alpha x)+p((1-\alpha)y)=\alpha p(x)+(1-\alpha)p(y)\leq$$ $$\leq\alpha k+(1-\alpha)k=k.$$ Здесь $0\leq\alpha\leq 1$. Таким образом, $E$ выпукло.
\end{proof}

\begin{theorem}[теорема Хана--Банаха]
	\label{HanBan1}
	Пусть $p(x)$ --- конечный выпуклый функционал, определенный на вещественном пространстве $L$. Пусть $L_0$ --- подпространство в $L$, и $f_0(x)$ --- линейный функционал на $L_0$, удовлетворяющий условию $f_0(x)\leq p(x)$ для любого $x\in L_0$. Тогда существует линейный функционал $f(x)$ на пространстве $L$ такой, что $f(x)=f_0(x)$ для любого $x\in L_0$, и $f(x)\leq p(x)$ для любого $x\in L$.
\end{theorem}

\begin{proof}
	Пусть $y\in L$ и $y\not\in L_0$. Пусть $L'$ --- подпространство, порожденное $L_0$ и элементом $y$. Рассмотрим продолжение $f'(x)$ линейного функционала $f_0(x)$, определяемое $f'(y)=a$. Поскольку любой элемент $L'$ имеет вид $\alpha y+x$, где $x\in L_0$. Тогда $f'(\alpha y+x)=\alpha a+f_0(x)$. Теперь выберем $a$ так, чтобы $$f'(\alpha y+x)=\alpha a+f_0(x)\leq p(\alpha y+x).$$ Если $\alpha>0$, то, деля на $\alpha$, получаем $$f_0\left(\frac{x}{\alpha}\right)+a\leq p\left(\frac{x}{\alpha}+y\right).$$ Отсюда, $$a\leq p\left(\frac{x}{\alpha}+y\right)-f_0\left(\frac{x}{\alpha}\right).$$ Если $\alpha<0$, то, деля на $-\alpha$, получаем $$-f_0\left(\frac{x}{\alpha}\right)-a\leq p\left(-\frac{x}{\alpha}-y\right).$$ Отсюда, $$a\geq -p\left(-\frac{x}{\alpha}-y\right)-f_0\left(\frac{x}{\alpha}\right).$$ Докажем, что $$-f_0(y_1)+p(y_1+y)\geq -f_0(y_2)-p(-y_2-y)$$ для любых $y_1,y_2\in L_0$. Действительно, $$f_0(y_1)-f_0(y_2)=f_0(y_1-y_2)\leq p(y_1-y_2)=p((y_1+y)+(-y_2-y))\leq$$ $$\leq p(y_1+y)+p(-y_2-y).$$ Пусть $$a'=\inf\limits_{y_1}(-f_0(y_1)+p(y_1+y)),\quad a''=\sup\limits_{y_2}(-f_0(y_2)-p(-y_2-y)).$$ Тогда $a'\geq a''$. Выберем $a$ так, чтобы $a'\geq a\geq a''$. Тогда функционал $f'$, определяемый на $L'$ формулой $f'(\alpha y+x)=\alpha a+f_0(x)$ удовлетворяет условию $f'(x)\leq p(x)$ для любого $x\in L'$. Если в $L$ можно выбрать счетную систему $y_1,y_2,\ldots,y_n,\ldots$ порождающую все $L$, то искомый функционал строится по индукции. В общем случае нужно применить лемму Цорна. Совокупность $B$ продолжений $f(x)$ функционала $f_0(x)$, удовлетворяющих условию $f_0(x)\leq p(x)$, является частично упорядоченным множеством, и каждое его линейно упорядоченное подмножество $B_0$ обладает верхней гранью. Этой верхней гранью является функционал, определенный на объединении областей определения функционалов $f\in B_0$. По лемме Цорма во всем $B$ существует максимальный элемент $f$. Этот максимальный элемент и определяет искомый функционал.
\end{proof}

Рассмотрим теперь комплексный случай.

\begin{definition}
	Неотрицательный вещественный функционал $p$, определенный на комплексном линейном пространстве $L$, называется \emph{выпуклым}, если
	\begin{enumerate}
		\item $p(x+y)\leq p(x)+p(y)$ $\forall x,y\in L$;
		\item $p(\alpha x)=|\alpha| p(x)$ $\forall x\in L$, $\alpha\in\CC$.
	\end{enumerate}
\end{definition}

\begin{theorem}[теорема Хана--Банаха]
	\label{HanBan2}
	Пусть $p(x)$ --- конечный выпуклый функционал, определенный на комплексном пространстве $L$. Пусть $L_0$ --- подпространство в $L$, и $f_0(x)$ --- линейный функционал на $L_0$, удовлетворяющий условию $|f_0(x)|\leq p(x)$ для любого $x\in L_0$. Тогда существует линейный функционал $f(x)$ на пространстве $L$ такой, что $f(x)=f_0(x)$ для любого $x\in L_0$, и $|f(x)|\leq p(x)$ для любого $x\in L$.
\end{theorem}

\begin{proof}
	Пусть $L_R$ и $L_{0R}$ --- пространства $L$ и $L_0$, рассматриваемые как вещественные линейные пространства. Положим $f_{0R}= Re f_0(x)$. Поскольку $|f_0(x)|\leq p(x)$, то $f_{0R}(x)\leq p(x)$ для всех $x\in L_0$. Согласно теореме \ref{HanBan1}, существует  действительный линейный функционал $f_R$ на $L_R$, удовлетворяющий условиям $$f_R(x)\leq p(x),\forall x\in L_R,\quad f_R(x)=f_{0R}(x),\forall x\in L_{0R}.$$ Очевидно, что $$-f_R(x)=f_R(-x)\leq p(-x)=p(x).$$ Следовательно, $|f_R(x)|\leq p(x)$. Положим $f(x)=f_R(x)-if_R(ix)$. Заметим, что $f(x)=f_0(x)$ для любого $x\in L_0$, и $Re f(x)=f_R(x)$. Осталось показать, что $|f(x)|\leq p(x)$.  Предположим, что существует $x_0\in L$ такое, что $|f(x_0)|>p(x_0)$. Тогда $f(x_0)=re^{i\varphi}$. Положим $y_0=e^{-i\varphi}x_0$. Тогда $$f_R(y_0)=Re f(y_0)=Re(e^{-i\varphi}f(x_0))=Re(e^{-i\varphi}re^{i\varphi})=r>p(x_0).$$ Заметим, что $$p(y_0)=p(e^{i\varphi}x_0)=|e^{i\varphi}|p(x_0)=p(x_0).$$ Тогда $f_R(y_0)>p(y_0)$. Противоречие.
\end{proof}

\begin{definition}
	\emph{Ядром} множества $E\subset L$ называется совокупность точек $x\in E$ таких, что для любого $y\in E$ существует $\varepsilon$ такое, что $x+\alpha y\in E$ для любых $|\alpha|<\varepsilon$.
\end{definition}

\begin{definition}
	Пусть $E$ --- выпуклое множество, ядро которого содержит $0$. Пусть $$p_E(x)=\inf\{r\mid\frac{x}{r}\in E\}.$$ Функционал $p_E(x)$ называется \emph{функционалом Минковского}.
\end{definition}

\begin{theorem}
	\label{Vyp3}
	Функционал минковского $p_E(x)$ является конечным и выпуклым.
\end{theorem}

\begin{proof}
	Конечность следует из того, что $0$ принадлежит ядру. Докажем выпуклость. Очевидно, что $p_E(x)>0$. Пусть $\alpha>0$. Тогда $$p_E(\alpha x)=\inf\{r\mid\frac{\alpha x}{r}\in E\}=\alpha\inf\{r\mid\frac{x}{r}\in E\}=\alpha p_E(x).$$ Пусть $x,y\in L$ и $\varepsilon>0$ --- произвольное число. Выберем $r_1,r_2$ так, что $$p_E(x_1)<r_1<p_E(x_1)+\varepsilon,\quad p_E(x_2)<r_2<p_E(x_2)+\varepsilon.$$ Положим $r=r_1+r_2$. Тогда $$\frac{x_1+x_2}{r}=\frac{r_1}{r}\frac{x_1}{r_1}+\frac{r_2}{r}\frac{x_2}{r_2}$$ принадлежит отрезку с концами $\frac{x_1}{r_1}$, $\frac{x_2}{r_2}$. В силу выпуклости $E$, $\frac{x_1+x_2}{r}\in E$. Тогда $$p_E(x_1+x_2)\leq r=r_1+r_2<p_E(x_1)+p_E(x_2)+2\varepsilon.$$ В силу произвольности $\varepsilon$, получаем $p_E(x_1+x_2)\leq p_E(x_1)+p_E(x_2)$.
\end{proof}

\section{Нормированные пространства}

\begin{definition}
	Пусть $L$ --- линейное пространство. Конечный вещественный функционал $p(x)$ называется \emph{нормой}, если он удовлетворяет следующим условиям.
	\begin{enumerate}
		\item $p(x)\geq 0$, причем $p(x)=0$ только при $x=0$;
		\item $p(x+y)\leq p(x)+p(y)$, $\forall x,y\in L$;
		\item $p(\alpha x)=|\alpha| p(x)$, $\forall x\in L$, $\forall \alpha\in K$.
	\end{enumerate}
	Линейное пространство, в котором задана норма, называется \emph{нормированным пространством}. Норму элемента $x\in L$ мы будем обозначать $\|x\|$.
\end{definition}

Заметим, что на любом нормированном пространстве, мы можем задать метрику $$\rho(x,y)=\|x-y\|.$$  Непосредственно проверяются аксиомы метрики.

\begin{definition}
	Полное нормированное пространство называется \emph{банаховым пространством}.
\end{definition}

\begin{theorem}
	\label{Nor1}
	Нормированное пространство $L$ полно тогда и только тогда, когда из сходимости числового ряда $\sum\limits_{n=1}^{\infty}\|x_n\|$ следует сходимость ряда $\sum\limits_{n=1}^{\infty}x_n$.
\end{theorem}

\begin{proof}
	Пусть $L$ --- полно и $\sum\limits_{n=1}^{\infty}\|x_n\|$ сходится. Рассмотрим $s_n=\sum\limits_{k=1}^{n}x_k$. Тогда для любого $\varepsilon$ существует номер $N$ такой, что длялюбого $n>N$ и любого $m$ выполнено $$\|s_{n+m}-s_n\|\leq\sum\limits_{k=n+1}^{n+m}\|x_k\|<\varepsilon.$$ Следовательно, последовательность $\{s_n\}$ фундаментальна. Обратно. Пусть $\{x_n\}$ --- фундаментальная последовательность. Рассмотрим подпоследовательность $x_{n_m}$ такую, что $\|x_{n_{m+1}}-x_{n_m}\|<\frac{1}{2^m}$. Поскольку ряд $\sum\limits_{m=1}^{\infty}\|x_{n_{m+1}}-x_{n_m}\|$ сходится, то ряд $\sum\limits_{m=1}^{\infty}(x_{n_{m+1}}-x_{n_m})$ сходится. Следовательно, сходится последовательность $x_{n_{1}}-x_{n_m}$ при $m$ стремящимся к бесконечности. Тогда сходится последовательность $\{x_{n_m}\}$, а, следовательно, и последовательность $\{x_n\}$.
\end{proof}

\begin{definition}
	Две нормы $p_1$ и $p_2$ на линейном пространстве $L$ называются \emph{эквивалентными}, если существуют положительные числа $c_1$ и $c_2$ такие, что $$c_1p_1(x)\leq p_2(x)\leq c_2p_1(x).$$
\end{definition}

\begin{remark}
	Очевидно, что эта эквивалентность рефлексивна, симметрична и транзитивна.
\end{remark}

\begin{theorem}
	\label{Nor2}
	На любом конечномерном пространстве все нормы эквивалентны.
\end{theorem}

\begin{proof}
	Пусть $L$ --- конечномерное пространство и $e_1,e_2,\ldots, e_n$ --- его базис. Пусть $$x=x_1e_1+x_2e_2+\cdots+x_ne_n.$$ Введем норму $$p(x)=|x_1|+|x_2|+\cdots+|x_n|.$$ Пусть $q(x)$ --- другая норма. Положим $c=\max\limits_i q(e_i)$. Тогда $$q(x)\leq |x_1|q(e_1)+|x_2|q(e_2)+\cdots+|x_n|q(e_n)\leq c(|x_1|+|x_2|+\cdots+|x_n|)=cp(x).$$ В частности, $q(x-y)\leq cp(x-y)$. Таким образом, функция $q$ непрерывна относительно метрики $p$. Заметим, что $q$ достигает максимума и минимума на единичной сфере относительно $p$, т.е. на множестве $S=\{x\mid p(x)=1\}$. Пусть $m=\min\limits_{x\in S}q(x)$. Заметим, что $m>0$. Тогда $$q(x)=q\left(p(x)\frac{x}{p(x)}\right)=p(x)q\left(\frac{x}{p(x)}\right)\geq mp(x).$$ Здесь мы воспользовались тем, что $p(\frac{x}{p(x)})=1$. Таким образом, $mp(x)\leq q(x)\leq cp(x)$, т.е. любая метрика эквивалентна $p(x)$.
\end{proof}

\begin{corollary}
	\label{Nor3}
	В конечномерном нормированном пространстве замкнутые шары компактны.
\end{corollary}

\begin{corollary}
	\label{Nor4}
	Любое конечномерное нормированное вещественное или комплексное пространство полно.
\end{corollary}

\begin{corollary}
	\label{Nor5}
	Всякое конечномерное линейное подпространство нормированного пространства замкнуто.
\end{corollary}

\begin{remark}
	Утверждение \ref{Nor5} неверно в бесконечномерном случае. Например в пространстве $C[a;b]$ множество многочленов образуют подпространство, которое не замкнуто.
\end{remark}


\begin{theorem}[лемма о почти перпендикуляре]
	\label{Nor6}
	Пусть $L_0$ --- замкнутое подпространство в нормированном пространстве $L$ и $L_0\neq L$. Тогда для любого $\varepsilon>0$ существует $x_{\varepsilon}\in L$ такой, что $\|x_{\varepsilon}\|=1$ и $\|x_{\varepsilon}-y\|> 1-\varepsilon$ для любого $y\in L_0$.
\end{theorem}

\begin{proof}
	Пусть $z\in L$, $z\not\in L_0$. Положим, $\delta=\inf\limits_{y\in L_0}\|z-y\|.$ Поскольку $L_0$ --- замкнуто, то $\delta>0$. Выберем $\varepsilon_0$ так, что $\frac{\delta}{\delta+\varepsilon_0}>1-\varepsilon$. Выберем $y_0\in L_0$ так, что $\|z-y_0\|<\delta+\varepsilon_0$. Положим $x_{\varepsilon}=\frac{z-y_0}{\|z-y_0\|}$. Тогда для любого $y\in L_0$ выполнено $$\|x_{\varepsilon}-y\|=\frac{1}{\|z-y_0\|}\|z-y_0-\|z-y_0\|y\|\geq\frac{\delta}{\|z-y_0\|}\geq\frac{\delta}{\delta+\varepsilon_0}>1-\varepsilon.$$
\end{proof}

\begin{theorem}[теорема Рисса]
	\label{Nor7}
	Нормированное пространство $L$ конечномерно тогда и только тогда, когда любое ограниченное множество в $L$ относительно компактно.
\end{theorem}

\begin{proof}
	В одну сторону мы уже доказали (см. \ref{Nor3} \ref{KomPr3}). Предположим, что $L$ бесконечномерное. Пусть $x_1\in L$ и $\|x_1\|=1$. Положим $L_1$ --- линейная оболочка $x_1$. Поскольку $L$ бесконечномерно, то $L_1\neq L$. Согласно теореме \ref{Nor6} существует $x_2$ такой, что $\|x_2\|=1$ и для любого $x\in L_1$ выполнено $\|x_2-x\|>\frac{1}{2}$. В частности $\|x_2-x_1\|>\frac{1}{2}$. Положим $L_2$ --- линейная оболочка $x_1,x_2$. Снова $L_2\neq L$. Согласно теореме \ref{Nor6} существует $x_3$ такой, что $\|x_3\|=1$ и для любого $x\in L_2$ выполнено $\|x_3-x\|>\frac{1}{2}$. В частности $\|x_3-x_1\|>\frac{1}{2}$, $\|x_3-x_2\|>\frac{1}{2}$. Продолжая этот процесс, мы получаем последовательность $x_1,x_2,\ldots, x_n,\ldots$ такую, что $\|x_n-x_m\|>\frac{1}{2}$. Из этой подпоследовательности нельзя выбрать сходящуюся подпоследовательность. Следовательно, шар радиуса $1$ не компактен.
\end{proof}

Рассмотрим теперь сопряженное пространство $L$. Пусть $L^*$ --- множество непрерывных линейных функционалов на $L$. Зададим на нем норму $$\|f\|=\sup\limits_{x\neq 0}\frac{|f(x)|}{\|x\|}.$$ Эта норма удовлетворяет всем требованиям. Действительно, $\|f\|>0$ для любого ненулевого функционала, $\|\alpha f\|=|\alpha| \|f\|$. Проверим неравенство треугольника, $$\|f_1+f_2\|=\sup\limits_{x\neq 0}\frac{|f_1(x)+f_2(x)|}{\|x\|}\leq=\sup\limits_{x\neq 0}\frac{|f_1(x)|}{\|x\|}+\sup\limits_{x\neq 0}\frac{|f_2(x)|}{\|x\|}=\|f_1\|+\|f_2\|.$$ Топология в $L^*$, определяемая этой нормой, называется \emph{сильной топологией} в $L^*$.

\begin{theorem}
	\label{Nor8}
	Сопряженное пространство полно.
\end{theorem}

\begin{proof}
	Пусть $\{f_n\}$ --- фундаментальная последовательность функционалов. Тогда для любого $\varepsilon>0$ существует $N$ такой, что для любых $n,m>N$ выполнено $\|f_n-f_m\|<\varepsilon$. Тогда для любого $x\in L$ имеем $$|f_n(x)-f_m(x)|\leq\|f_n-f_m\|\cdot\|x\|<\varepsilon \|x\|.$$ Таким образом, последовательность $f_n(x)$ сходится для любого $x$. Положим $$f(x)=\lim\limits_{n\rightarrow\infty} f_n(x).$$ Докажем, что $f(x)$ --- линейный непрерывный функционал. Проверим линейность $$f(\alpha x+\beta y)=\lim\limits_{n\rightarrow\infty} f_n(\alpha x+\beta y)=\lim\limits_{n\rightarrow\infty} (\alpha f_n(x)+\beta f_n(y))=$$ $$=\alpha\lim\limits_{n\rightarrow\infty} f_n(x)+\beta\lim\limits_{n\rightarrow\infty} f_n(y)=\alpha f(x)+\beta f(y).$$ Выберем $N$ так, что для любых $n>N$ и $p$ выполнено $\|f_{n+p}-f_n\|<1$. Тогда $\|f_{n+p}\|\leq\|f_n\|+1.$ Следовательно, $|f_{n+p}(x)|\leq(\|f_n\|+1)\|x\|.$ Устремляя $p$ к бесконечности, получаем $|f(x)|\leq(\|f_n\|+1)\|x\|$. Отсюда, $f(x)$ непрерывен. Зафиксируем $\varepsilon$, выберем $N$ так, что для любых $n>N$ и $p$ выполнено $\|f_{n+p}-f_n\|<\varepsilon$. Тогда для любого $x\in L$ выполнено $|f_{n+p}(x)-f_n(x)|<\varepsilon\|x\|$. Устремим $p$ к бесконечности, получим $|f(x)-f_n(x)|\leq\varepsilon\|x\|$. Таким образом, $\|f-f_n\|\leq\varepsilon$. Следовательно, $\{f_n\}$ сходится к $f$.
\end{proof}

\section{Эвклидовы и гильбертовы пространства}


\begin{definition}
	Пусть $L$ --- вещественное линейное пространство. \emph{Скалярным произведением} в $L$ называется действительная функция $(x,y)$ на $L\times L$, удовлетворяющая следующим условиям.
	\begin{enumerate}
		\item $(x,y)=(y,x)$;
		\item $(x+y,z)=(x,z)+(y,z)$;
		\item $(\alpha x,y)=\alpha (x,y)$;
		\item $(x,x)\geq 0$ причем $(x,x)=0$ только при $x=0$.
	\end{enumerate}
	Линейное пространство, в котором задано скалярное произведение, называется \emph{эвклидовым пространством}.
\end{definition}

Заметим, что скалярное произведение задает норму с помощью формулы $$\|x\|=\sqrt{(x,x)}.$$ Таким образом эвклидовы пространства являются нормированными.

\begin{definition}
	Полное сепарабельное эвклидово пространство называется \emph{гильбертовым пространством}.
\end{definition}

\begin{remark}
	Иногда в определении гильбертовых пространств требуют бесконечномерность.
\end{remark}

\begin{remark}
	Заметим, что нормированное пространство мы определили как над полем вещественных чисел, так и над полем комплексных чисел. Однако, эвклидово пространство и гильбертово пространство мы определили лишь над полем вещественных чисел. Более того, если определить скалярное произведение таким же образом, то $(ix,ix)=-(x,x)$. Таким образом, квадраты $x$ и $ix$ не могут быть одновременно положительными. Поэтому на комплексных пространствах вводят эрмитово скалярное произведение.
\end{remark}

\begin{definition}
	Пусть $L$ --- комплексное линейное пространство. \emph{Эрмитовым скалярным произведением} в $L$ называется комплексная функция $(x,y)$ на $L\times L$, удовлетворяющая следующим условиям.
	\begin{enumerate}
		\item $(x,y)=\overline{(y,x)}$;
		\item $(x+y,z)=(x,z)+(y,z)$;
		\item $(\alpha x,y)=\alpha (x,y)$;
		\item $(x,x)\geq 0$ причем $(x,x)=0$ только при $x=0$.
	\end{enumerate}
	Линейное пространство, в котором задано эрмитово скалярное произведение, называется \emph{эрмитовым пространством}.
\end{definition}

\begin{claim}
	\label{Er1} $(x,\alpha y)=\bar{\alpha}(x,y)$
\end{claim}

\begin{proof}
	Действительно, $$(x,\alpha y)=\overline{(\alpha y,x)}=\overline{\alpha(y,x)}=\bar{\alpha}\overline{(y,x)}=\bar{\alpha}(x,y).$$
\end{proof}

Теперь вернемся к вещественным пространствам.

\begin{theorem}[неравенство Коши--Буняковского]
	\label{Ev1}
	Пусть $L$ --- вещественное эвклидово пространство. Тогда $$|(x,y)|\leq\|x\| \|y\|.$$
\end{theorem}


\begin{proof}
	Рассмотрим $x+\alpha y$. Тогда $$0\leq (x+\alpha y,x+\alpha y)=(y,y)\alpha^2+2(x,y)\alpha+(x,x).$$ Это неравенство должно быть выполнено для любого $\alpha$. Следовательно, дискриминант должен быть отрицательным, т.е. $$4(x,y)^2-4(x,x)(y,y)\leq 0.$$ Отсюда, $|(x,y)|\leq\|x\| \|y\|.$
\end{proof}

\begin{definition}
	Два элемента $x,y$ называются \emph{ортогональными}, если $(x,y)=0$. Система ненулевых векторов $\{x_{\alpha}\}$ называется \emph{ортогональной}, если $(x_{\alpha},x_{\beta})=0$ при $\alpha\neq\beta$.
\end{definition}

\begin{definition}
	Система векторов $\{x_{\alpha}\}$ называется \emph{полной} в $L$, если наименьшее содержащее $\{x_{\alpha}\}$ замкнутое подпространство есть само $L$. Если ортогональная система $\{x_{\alpha}\}$ полна, то она называется \emph{ортогональным базисом}. Если при этом норма каждого элемента равна $1$, то система $\{x_{\alpha}\}$ называется \emph{ортонормированным базисом}.
\end{definition}

\begin{theorem}
	\label{Ev2}
	Пусть $f_1,f_2,\ldots, f_n,\ldots$ --- линейно независимая система элементов в эвклидовом пространстве $L$. Тогда в $L$ существует ортонормированная система $$\varphi_1, \varphi_2,\ldots, \varphi_n,\ldots$$ такая, что $$\varphi_n=a_{n1}f_1+a_{n2}f_2+\cdots+a_{nn}f_n,$$ причем $a_{nn}\neq 0$.
\end{theorem}

\begin{proof}
	Элемент $\varphi_1$ ищется в виде $\varphi_1=a_{11} f_1$. Находим $a_{11}$, получаем $$1=(\varphi_1,\varphi_1)=a_{11}^2(f_1,f_1).$$ Отсюда, $a_{11}=\frac{1}{(f_1,f_1)}$. Предположим, что $\varphi_1, \varphi_2,\ldots, \varphi_{n-1}$ уже построены. Тогда $$\varphi_n=f_n-b_{n1}\varphi_1-b_{n2}\varphi_2-\cdots-b_{n n-1}\varphi_{n-1}.$$ Находим $b_{ni}$ из условия $(\varphi_n,\varphi_i)=0$. Получаем $b_{ni}=(f_n,\varphi_i)$. Таким образом мы получили элемент $\varphi_n$, ортогональный ко всем $\varphi_1, \varphi_2,\ldots, \varphi_{n-1}$. Поделим его на $(\varphi_n,\varphi_n)$, получаем ортонормированную систему.
\end{proof}

\begin{corollary}
	\label{Ev3}
	В сепарабельном эвклидовом пространстве существует ортонормированный базис.
\end{corollary}

\begin{proof}
	Пусть $g_1,g_2,\ldots, g_n,\ldots$ --- счетное всюду плотное множество. Тогда система $g_1,g_2,\ldots, g_n,\ldots$ полна. Выберем из этой системы полную систему линейно независимых элементов $f_1,f_2,\ldots, f_n,\ldots$ . Для этого достаточно исключить все элементы $g_n$, являющиеся линейной комбинацией $g_1,g_2,\ldots, g_{n-1}$. Применим к полученной системе теорему \ref{Ev2}, получим ортонормированный базис.
\end{proof}


\begin{definition}
	Пусть $\{f_n\}$ --- ортонормированная система в эвклидовом пространстве $L$, $g\in L$. Число $c_n=(g,f_n)$ называется \emph{коэффициентом Фурье} элемента $g(x)$ по ортонормированной системе $\{f_n\}$. Ряд $\sum\limits_{n=1}^{\infty} c_n f_n$ называется \emph{рядом Фурье} элемента $g$ по ортонормированной системе $\{f_n\}$.
\end{definition}


\begin{theorem}[неравенство Бесселя]
	\label{Bes}
	Пусть $\{f_n\}$ --- ортонормированная система в эвклидовом пространстве $L$, $g\in L$. Тогда для любого $n$ справедливо неравенство $$\sum\limits_{k=1}^n c_k^2\leq \|g\|^2,$$ где $c_k=(f_k,g)$.
\end{theorem}

\begin{proof}
	Рассмотрим функцию $$g_n=\sum\limits_{k=1}^n c_kf_k.$$ Пусть $h_n=g-g_n$. Тогда $$(g_n,f_m)=\sum\limits_{k=1}^n c_k(f_k,f_m)=\begin{cases}c_m,\quad\text{если $m\leq n$} \\
			0,\quad\text{если $m>n$}.\end{cases}$$ Тогда $$(h_n,f_m)=c_m-(g_n,f_m)=\begin{cases}0,\quad\text{если $m\leq n$} \\
			c_m,\quad\text{если $m>n$}.\end{cases}$$ Отсюда, $$(g_n,h_n)=(h_n,g_n)=0.$$ Заметим, что $$(g_n,g_n)=\left(\sum\limits_{k=1}^n c_kf_k,\sum\limits_{l=1}^n c_lf_l\right)=\sum\limits_{k=1}^n\sum\limits_{l=1}^nc_k c_l(f_k,f_l)=\sum\limits_{k=1}^n c_k^2.$$ Следовательно, $$(g,g)=(g_n+h_n,g_n+h_n)=(g_n,g_n)+2(g_n,h_n)+(h_n,h_n)=$$ $$=(g_n,g_n)+(h_n,h_n)\geq (g_n,g_n)=\sum\limits_{k=1}^n c_k^2.$$
\end{proof}

Устремляя $n$ к бесконечности, получаем следующие следствия.

\begin{corollary}
	\label{Bes2}
	В условиях теоремы \ref{Bes} имеем $$\sum\limits_{k=1}^{\infty} c_k^2\leq(g,g).$$
\end{corollary}

\begin{corollary}
	\label{Bes3}
	В условиях теоремы \ref{Bes} $$\lim\limits_{n\rightarrow\infty} c_n=0.$$
\end{corollary}

\begin{definition}
	Ортонормированная система $\{f_n\}$ называется \emph{замкнутой}, если $$(g,g)=\sum\limits_{k=1}^{\infty} c_k^2.$$ Это равенство называется \emph{равенством Парсеваля}.
\end{definition}

\begin{theorem}
	\label{Bes4}
	При любых $a_1,a_2,\cdots, a_n$ выполнено $$\left\|g-\sum\limits_{k=1}^{n} c_k f_k\right\|\leq\left\|g-\sum\limits_{k=1}^{n} a_k f_k\right\|.$$
\end{theorem}

\begin{proof}
	Пусть $g_n=\sum\limits_{k=1}^{n} c_k f_k$. Заметим, что $(g-g_n,f_k)=0$ при всех $k\leq n$. Тогда $$\left\|g-\sum\limits_{k=1}^{n} a_k f_k\right\|^2=\left\|(g-g_n)+\sum\limits_{k=1}^{n}(c_k-a_k) f_k\right\|^2=$$ $$=\left\|g-g_n\right\|^2+\left\|\sum\limits_{k=1}^{n}(c_k-a_k) f_k\right\|^2=\left\|g-g_n\right\|^2+\sum\limits_{k=1}^{n}(c_k-a_k)^2\geq\|g-g_n\|^2.$$
\end{proof}

\begin{theorem}
	\label{Bes5}
	В сепарабельном эвклидовом пространстве $L$ всякая полная ортонормированная система функций является замкнутой, и обратно.
\end{theorem}

\begin{proof}
	Пусть $\{f_n\}$ --- замкнутая ортонормированная система и $g\in L$. Рассмотрим последовательность частичных сумм $$g_n=\sum\limits_{k=1}^n c_k f_k.$$ Тогда $$\|g-g_n\|^2=(g-g_n,g-g_n)=\|g\|-2(g,g_n)+\sum\limits_{k=1}^n c^2_k=$$ $$=\sum\limits_{k=1}^{\infty} c^2_k+\sum\limits_{k=1}^n c^2_k-2\left(g,\sum\limits_{k=1}^n c_k f_k\right)=\sum\limits_{k=1}^{\infty} c^2_k+\sum\limits_{k=1}^n c^2_k-2\sum\limits_{k=1}^n c_k(g,f_k)=$$ $$=\sum\limits_{k=1}^{\infty} c^2_k+\sum\limits_{k=1}^n c^2_k-2\sum\limits_{k=1}^n c^2_k=\sum\limits_{k=n+1}^{\infty} c^2_k.$$ Поскольку $\sum\limits_{k=1}^{\infty} c_k^2=(g,g)$, то $\sum\limits_{k=n+1}^{\infty} c^2_k$ стремиться к нулю при $n\rightarrow\infty$. Таким образом, последовательность $\{g_n\}$ сходиться к $g$. Следовательно, $\{f_n\}$ всюду плотно в $L$, т.е. система $\{f_n\}$ полна. Обратно, пусть $\{f_n\}$ полна. Тогда для любого $\varepsilon>0$ существует сумма $\sum\limits_{k=1}^{n} a_k f_k$ такая, что $$\|g-\sum\limits_{k=1}^{n} a_k f_k\|<\varepsilon.$$ Согласно теореме \ref{Bes4} имеем $$\|g-\sum\limits_{k=1}^{n} c_k f_k\|<\varepsilon.$$ Отсюда, $$\varepsilon>\|g-\sum\limits_{k=1}^{n} c_k f_k\|\geq\|g\|-\|\sum\limits_{k=1}^{n} c_k f_k\|=\|g\|-\sum\limits_{k=1}^{n} c^2_k.$$ В силу произвольности $\varepsilon>0$ мы получаем равенство Парсеваля.
\end{proof}

Рассмотрим более подробно гильбертовы пространства.

\begin{lemma}
	\label{LHil1}
	Пусть $X$ --- сепарабельное метрическое пространство. Тогда любое его подпространство $X'$ сепарабельно.
\end{lemma}

\begin{proof}
	Пусть $x_1,x_2,\ldots,x_n,\ldots$ --- счетное всюду плотное множество в $X$. Положим $$a_n=\inf\limits_{y\in X'}\rho(x_n,y).$$ Тогда для любых $m$ и $n$ существует точка $y_{mn}\in X'$ такая, что $$\rho(x_n,y_{mn})<a_n+\frac{1}{m}.$$ Пусть $y\in X'$. Заметим, что для любого $\varepsilon$ существует $x_n$ такой, что $\rho(x_n,y)<\varepsilon$. Следовательно, $a_n<\varepsilon$. Тогда $$\rho(y_{mn},y)\leq\rho(y_{mn},x_n)+\rho(y,x_n)<a_n+\frac{1}{m}+\varepsilon<2\varepsilon+\frac{1}{m}.$$ В силу произвольности $m$ можно считать, что $\frac{1}{m}<\varepsilon$. Получаем $\rho(y_{mn},y)\leq3\varepsilon$. Следовательно, $\{y_{mn}\}$ --- счетное всюду плотное множество в $X'$.
\end{proof}

\begin{theorem}
	\label{Hil1}
	Пусть $H$ --- гильбертого пространство, $M$ --- замкнутое подпространство в $H$. Тогда $M$ содержит ортонормированную систему $\{\varphi_n\}$, линейное замыкание которой  совпадает с $M$.
\end{theorem}

\begin{proof}
	Применим процесс ортогонализации к какой-либо счетной всюду плотной последовательности в $M$.
\end{proof}

\begin{claim}
	\label{Hil2}
	Пусть $M$ --- замкнутое подпространство в $H$. Обозначим через $M'$ множество элементов $g\in H$, ортогональных ко всем $f\in M$. Тогда $M'$ --- замкнутое подпространство в $H$.
\end{claim}

\begin{proof}
	Пусть $g_1,g_2\in M'$. Тогда для любого $f\in M$ имеем $$(\alpha_1g_1+\alpha_2g_2,f)=\alpha_1(g_1,f)+\alpha_2(g_2,f)=0.$$ Отсюда следует линейность $M'$. Докажем замкнутость. Пусть последовательность $g_n\in M'$ стремиться к $g$. Тогда $$(f,g)=\lim\limits_{n\rightarrow\infty}(f,g_n)=0.$$ Следовательно, $g\in M'$, и $M'$ замкнуто.
\end{proof}

Подпространство $M'$ называется \emph{ортогональным дополнением} подпространства $M$.

\begin{theorem}
	\label{Hil3}
	Пусть $H$ --- гильбертого пространство, $M$ --- замкнутое подпространство в $H$. Тогда любой элемент $f\in H$ единственным образом представляется в виде $f=g+h$, где $g\in M$, $h\in M'$.
\end{theorem}

\begin{proof}
	Пусть $\{\varphi_n\}$ --- полная ортогональная система в $M$. Положим $$g=\sum\limits_{n=1}^{\infty} c_n\varphi_n,\quad c_n=(f,\varphi_n).$$ Согласно неравенству Бесселя ряд $\sum\limits_{n=1}^{\infty} c^2_n$ сходится. Тогда $\sum\limits_{n=1}^{\infty} c_n\varphi_n$ тоже сходится и $g\in M$. Положим $h=f-g$. Заметим, что $(h,\varphi_n)=0$ для любого $\varphi_n$. С другой стороны, любой элемент $v\in M$ можно представить в виде $v=\sum\limits_{n=1}^{\infty} a_n\varphi_n$. Тогда $$(h,v)=\sum\limits_{n=1}^{\infty} a_n(v,\varphi_n)=0.$$ Таким образом, $h\in M'$. Предположим, что существует другое представление $f=g_1+h_1$, где $g_1\in M$, $h_1\in M'$. Тогда $$(g_1,\varphi_n)=(g,\varphi_n)=c_n$$ для всех $n$. Отсюда, $g=g_1$. Следовательно, $h=h_1$.
\end{proof}

\begin{corollary}
	\label{Hil4}
	Каждая ортонормированная система может быть расширена до системы, полной в $H$.
\end{corollary}

\begin{theorem}
	\label{Hil5}
	Пусть $\{f_n\}$ --- произвольная ортонормированная система в гильбертовом пространстве $H$. Пусть числовой ряд $\sum\limits_{k=1}^{\infty}c^2_k$ сходится. Тогда существует элемент $g\in H$ такой что $c_k=(g,f_k)$ и $$\sum\limits_{k=1}^{\infty}c^2_k=(f,f)=\|f\|.$$
\end{theorem}

\begin{proof}
	Положим $g_n=\sum\limits_{k=1}^{n}c_k f_k$. Тогда $$\|g_{n+p}-g_{n}\|^2=\|\sum\limits_{k=n+1}^{n+p}c_k f_k\|^2=\sum\limits_{k=n+1}^{n+p}c^2_k.$$ Поскольку ряд $\sum\limits_{k=1}^{\infty}c^2_k$ сходится, то последовательность $\{g_n\}$ фундаментальна. Пусть $g$ --- ее предел. Тогда $$(g,f_k)=(g_n,f_k)+(g-g_n,f_k),$$ где $n\geq k$. Заметим, что $(g_n,f_k)=c_k$ и $$|(g-g_n,f_k)|\leq\|g-g_n\|\|f_k\|=\|g-g_n\|\rightarrow 0,\quad n\rightarrow\infty.$$ Переходя к пределу при $n\rightarrow\infty$, получаем $(g,f_k)=c_k$. С другой стороны, $$\|g-g_n\|\|^2=\left(g-\sum\limits_{k=1}^{n}c_k f_k,g-\sum\limits_{k=1}^{n}c_k f_k\right)=(g,g)-\sum\limits_{k=1}^{n}c^2_k.$$ Таким образом, $$\sum\limits_{k=1}^{\infty}c^2_k=(f,f)=\|f\|.$$
\end{proof}

\begin{theorem}[теорема Рисса--Фишера]
	\label{RisFish}
	Всякое бесконечномерное гильбертого пространство изометрично пространству $l_2$.
\end{theorem}

\begin{proof}
	Пусть $H$ --- бесконечномерное гильбертово пространство, $\{f_n\}$ --- ортонормированный базис. Определим $J\colon H\rightarrow l_2$ по правилу $$J(g)=(c_1,c_2,\ldots,c_n,\ldots),$$ где $c_k=(g,f_k)$. Очевидно, что $$\|J(g)\|^2=\|g\|^2=\sum\limits_{k=1}^{\infty}c_k^2.$$ Более того, согласно теореме \ref{Hil5}, $J(H)=l_2$.
\end{proof}

\begin{corollary}
	\label{RisFish2}
	Все бесконечномерные гильбертовы пространства изометричны.
\end{corollary}

\begin{theorem}[теорема Рисса]
	\label{RisHil}
	Пусть $H$ --- гильбертово пространство. Тогда для любого $v\in H$ формула $f_v(x)=(x,v)$ задает линейный непрерывный функционал на $H$ и $\|f_v\|=\|v\|$. Обратно, всякий функционал $f\in H^*$ задается таким способом.
\end{theorem}


\begin{proof}
	Очевидно, что $f_v(x)=(x,v)$ задает линейный непрерывный функционал на $H$. Равенство $\|f_v\|=\|v\|$ следует из неравенства Коши--Буняковского $|(v,x)|\leq\|v\|\|x\|$ и равенства $f_v(v)=\|v\|^2$. Пусть $f\in H^*$. Если $f=0$, то возьмем $v=0$. Пусть $f\neq 0$ и $H_0$ --- ядро $f$. Тогда $H_0$ имеет коразмерность один (см. \ref{LP1}). Пусть $H'_0$ --- ортогональное дополнение к $H_0$. Тогда $H'_0$ имеет размерность один. Пусть $e\in H'_0$ --- единичный вектор. Положим $v=f(e) e$. Согласно теореме \ref{Hil3} любой элемент $x\in H$ имеет представление $x=ae+y$, где $y\in H_0$, $a\in\RR$. Тогда $f(x)=af(e)+f(y)=af(e)$. С другой стороны, $$(x,v)=(ae+y,f(e)e)=af(e)(e,e)+f(e)(y,e)=af(e).$$
\end{proof}

\begin{theorem}
	\label{Ev4}
	Нормированное пространство $L$ эвклидово тогда и только тогда, когда для любых двух элементов $f,g$ выполнено равенство $$\|f+g\|^2+\|f-g\|^2=2(\|f\|^2+\|g\|^2).$$
\end{theorem}

\begin{proof}
	Пусть $L$ --- эвклидово пространство. Тогда $$\|f+g\|^2+\|f-g\|^2=(f+g,f+g)+(f-g,f-g)=$$ $$=(f,f)+2(f,g)+(g,g)+(f,f)-2(f,g)+(g,g)=2(\|f\|^2+\|g\|^2).$$ Таким образом, мы доказали необходимость. Докажем достаточность. Положим $$(f,g)=\frac{1}{4}(\|f+g\|^2-\|f-g\|^2).$$ Докажем, что $(f,g)$ --- скалярное произведение. Очевидно, что $(f,g)=(g,f)$. Пусть $g=f$. Тогда $$(f,f)=\frac{1}{4}(\|f+f\|^2-\|f-f\|^2)=\frac{1}{4}\|2f\|^2=\|f\|^2.$$ Рассмотрим функцию $$\Phi(f_1,f_2,g)=4((f_1+f_2,g)-(f_1,g)-(f_2,g))=$$ $$\|f_1+f_2+g\|^2-\|f_1+f_2-g\|^2-\|f_1+g\|^2+\|f_1-g\|^2-\|f_2+g\|^2+\|f_2-g\|^2.$$ Нам нужно доказать, что $\Phi(f_1,f_2,g)=0$ для любых $f_1,f_2,g$. Заметим, что $$\|f_1+f_2+g\|^2=2\|f_1+g\|^2+2\|f_2\|^2-\|f_1+g-f_2\|^2,$$ $$\|f_1+f_2-g\|^2=2\|f_1-g\|^2+2\|f_2\|^2-\|f_1-g-f_2\|^2.$$ Получаем $$\Phi(f_1,f_2,g)=\|f_1-g-f_2\|^2-\|f_1+g-f_2\|^2+$$ $$+\|f_1+g\|^2-\|f_1-g\|^2-\|f_2+g\|^2+\|f_2-g\|^2.$$ Просуммировав выражения для $\Phi(f_1,f_2,g)$, получаем $$2\Phi(f_1,f_2,g)=\|f_1+f_2+g\|^2-\|f_1+f_2-g\|^2+\|f_1-g-f_2\|^2-\|f_1+g-f_2\|^2-$$ $$-2\|f_2+g\|^2+2\|f_2-g\|^2.$$ Заметим, что $$\|f_1+f_2+g\|^2+\|f_1-g-f_2\|^2=2\|f_2+g\|^2+2\|f_1\|^2,$$ $$\|f_1+f_2-g\|^2+\|f_1+g-f_2\|^2=2\|f_2-g\|^2+2\|f_1\|^2.$$ Получаем, $2\Phi(f_1,f_2,g)=0$. Рассмотрим $$\varphi(c)=(cf,g)-c(f,g).$$ Нужно доказать, что $\varphi(c)=0$ для любого $c$. Очевидно, что $\varphi(1)=0$. Заметим, что $$\varphi(0)=\frac{1}{4}(\|g\|^2-\|g\|^2)=0.$$ Более того, $$\varphi(-1)=(-f,g)+(f,g)=$$ $$=\frac{1}{4}(\|-f+g\|^2-\|-f-g\|^2)+\frac{1}{4}(\|f+g\|^2-\|f-g\|^2)=0.$$ Таким образом, $(-f,g)=-(f,g)$. Пусть $n\in\ZZ$. Тогда $$(nf,g)=\sign n(|n|f,g)=\sign n(f+\cdots+f,g)=\sign n |n|(f,g)=n(f,g).$$ Таким образом, $\varphi(n)=0$ для любого $n\in\ZZ$. Пусть $\frac{p}{q}\in\QQ$. Тогда $$(\frac{p}{q} f,g)=p(\frac{1}{q} f,g)=\frac{p}{q}q(\frac{1}{q} f,g)=\frac{p}{q}((\frac{1}{q} f,g)+\cdots(\frac{1}{q} f,g))=\frac{p}{q}(f,g).$$ Таким образом, $\varphi(c)=0$ для любого $c\in\QQ$. В силу непрерывности $\varphi(c)$ получаем $\varphi(c)=0$ для любого $c\in\RR$.
\end{proof}

\begin{example}
	Рассмотрим $n$-мерное пространство $\RR^n$. Определим норму $$\|x\|_p=\left(\sum\limits_{i=1}^n|x_i|^p\right)^{\frac{1}{p}}.$$ Пусть $$f=(1,1,0,0,\ldots,0),\quad g=(1,-1,0,0,\ldots,0).$$ Тогда $$f+g=(2,0,0,0,\ldots,0),\quad f-g=(0,2,0,0,\ldots,0).$$ С другой стороны, $$\|f\|_p=\|g\|_p=2^{\frac{1}{p}},\quad \|f+g\|=\|f-g\|=2.$$ Таким образом, $\RR^n$ с этой нормой не является эвклидовым при $p\neq 2$.
\end{example}

\section{Функции ограниченной вариации}

\begin{definition}
	Пусть функция $f(x)$ задана на отрезке (интервале, полуинтервале) $[a;b]$. Будем говорить, что функция $f(x)$ имеет \emph{ограниченную вариацию}, если $$V_a^b(f):=\sup\sum\limits_{i=1}^n|f(t_i)-f(t_{i-1})|<\infty,$$ где $\sup$ берется по всем наборам $t_0,t_1,\ldots,t_n\in[a;b]$.
\end{definition}

\begin{claim}
	\label{Var1} Пусть $f(x)$ --- функция ограниченной вариации на отрезке $[a;b]$. Тогда $f(x)$ ограничена на $[a;b]$. Более того, для любых точек $x_0,x\in [a;b]$ выполнено $|f(x)|\leq |f(x_0)|+V_a^b(f)$.
\end{claim}

\begin{proof}
	Действительно, $$|f(x)|-|f(x_0)|\leq |f(x)-f(x_0)|\leq \sup\sum\limits_{i=1}^n|f(t_i)-f(t_{i-1})|.$$
\end{proof}

\begin{claim}
	\label{Var2}
	Пусть $f,g$ --- функции ограниченной вариации на отрезке $[a;b]$. Тогда $f+g$ --- функция ограниченной вариации на отрезке $[a;b]$ и $\alpha f$ --- функция ограниченной вариации на отрезке $[a;b]$. Более того, $V_a^b(f+g)\leq V_a^b(f)+V_a^b(g)$ и $V_a^b(\alpha f)=|\alpha|V_a^b(f)$.
\end{claim}

\begin{proof}
	Имеем $$V_a^b(f+g)=\sup\sum\limits_{i=1}^n|f(t_i)+g(t_i)-f(t_{i-1})-g(t_{i-1})|\leq $$ $$\leq\sup\sum\limits_{i=1}^n(|f(t_i)-f(t_{i-1})|+|g(t_i)-g(t_{i-1})|)\leq $$ $$\leq\sup\sum\limits_{i=1}^n|f(t_i)-f(t_{i-1})|+\sup\sum\limits_{i=1}^n|g(t_i)-g(t_{i-1})|.$$ Аналогично, $$V_a^b(\alpha f)\sup\sum\limits_{i=1}^n|\alpha f(t_i)-\alpha f(t_{i-1})|=|\alpha|\sup\sum\limits_{i=1}^n|f(t_i)-f(t_{i-1})|.$$
\end{proof}

Таким образом, функции ограниченной вариации образуют линейное пространство.

\begin{theorem}
	\label{Var3}
	Пусть $f(x)$ --- функция ограниченной вариации на отрезке $[a;b]$. Тогда
	\begin{enumerate}
		\item функции $V(x)=V_a^x(f)$ и $U(x)=V(x)-f(x)$ неубывающие на $[a;b]$;
		\item $V_a^b(f)=V_a^c(f)+V_c^b(f)$ для любого $c\in[a;b]$.
	\end{enumerate}
\end{theorem}

\begin{proof}
	Поскольку добавление новой точки в набор $t_0,t_1,\ldots, t_n$ сумма $\sum\limits_{i=1}^n|f(t_i)-f(t_{i-1})|$ не уменьшается, то можно считать, что $c$ входит в набор $t_0,t_1,\ldots, t_n$. Пусть $t_k=c$. Тогда $$V_a^b(f)=\sup\sum\limits_{i=1}^n|f(t_i)-f(t_{i-1})|=$$ $$=\sup\sum\limits_{i=1}^k|f(t_i)-f(t_{i-1})|+\sup\sum\limits_{i=k+1}^n|f(t_i)-f(t_{i-1})|=V_a^c(f)+V_c^b(f).$$ Отсюда, мы получили утверждение (2) и то, что функция $V(x)=V_a^x(f)$ неубывающая. Заметим, что $$V(x)-V(y)=V_x^y(f)\geq |f(x)-f(y)|\geq f(x)-f(y).$$ Таким образом, $U(x)=V(x)-f(x)$ неубывает на $[a;b]$.
\end{proof}

\begin{corollary}
	\label{Var4}
	Любая функция ограниченной вариации может быть представлена как разность двух ограниченных неубывающих функций.
\end{corollary}

Пусть $f(x)$ --- функция на $[a;b]$, $u(x)$ --- функция ограниченной вариации на отрезке $[a;b]$. Пусть $$V=\{a=t_0<t_1<\cdots<t_n=b,\xi_1,\xi_2,\ldots,\xi_n\}$$ --- размеченное разбиение. Пусть $\Delta_i u=u(t_i)-u(t_{i-1})$. Положим $$\sigma_{u,V}(f)=\sum\limits_{i=1}^n f(\xi_i)\Delta_i u.$$ Тогда $\sigma_{u,V}(f)$ называется \emph{интегральной суммой Стильтьеса}. Если существует предел $$I_u(f)=\lim\limits_{\Delta_V\rightarrow 0} \sigma_{u,V}(f),$$ то функция $f(x)$ называется \emph{интегрируемой по функции} $u(x)$, а величина $I_u(f)$ называется \emph{интегралом Стильтьеса} от функции $f(x)$ по функции $u(x)$.

\chapter{Теория операторов}

\section{Определения и основные свойства}

\begin{definition}
	Пусть $L$ и $L'$ --- два линейных пространства. \emph{Линейным оператором} $A$, действующим из $L$ в $L'$, называется отображение $y=Ax$ ($x\in L$, $y\in L'$), удовлетворяющее условиям.
	\begin{enumerate}
		\item $A(\alpha x)=\alpha Ax$;
		\item $A(x_1+x_2)=Ax_1+Ax_2$.
	\end{enumerate}
\end{definition}

Множество всех $x\in L$, для которых отображение $A$ определено, называется \emph{областью определения} оператора $A$. Предположим, что $L,L'$ --- нормированные пространства. Оператор $A$ \emph{непрерывен} в точке $x_0\in L$, если для любого $\varepsilon>0$ существует $\delta$ такое, что для любого $x\in L$ с условием $\|x-x_0\|<\delta$ выполнено $\|Ax-Ax_0\|<\varepsilon$. Оператор $A$ ограничен, если он определен на всем $L$ и каждое ограниченное множество переводит в ограниченное множество.


\begin{claim}
	\label{Op1}
	Для того, чтобы линейный оператор $A$ был ограничен необходимо и достаточно, чтобы существовала константа $C$ такая, что $\|Ax\|\leq C\|x\|$ для любого $x$.
\end{claim}

\begin{proof}
	Пусть $A$ ограничен. Пусть $C=\sup\limits_{x:\|x\|= 1}\|Ax\|$. Докажем, что $\|Ax\|\leq C\|x\|$ для любого $x$. Предположим, что существует $x\in L$ такой, что $\|A x\|>C\|x\|$. Рассмотрим $y=\frac{x}{\|x\|}$. Заметим, что $\|y\|=1$. С другой стороны, $$\|Ay\|=\frac{1}{\|x\|}Ax>\frac{C\|x\|}{\|x\|}.$$ Противоречие. Предположим, что существует константа $C$ такая, что $\|Ax\|\leq C\|x\|$ для любого $x$. Пусть $M\subset L$ --- ограниченное множество, но $AM\subset L'$ не ограничено. Поскольку $M$ --- ограниченное множество, то существует константа $K$ такая, что $\|x\|<K$ для всех $x\in M$. С другой стороны, $AM\subset L'$ не ограничено, следовательно, существует $x\in M$ такой, что $\|Ax\|>CK>C\|x\|$. Противоречие.
\end{proof}

Число $$\sup\limits_{x:\|x\|= 1}\|Ax\|=\sup\limits_{x\neq 0}\frac{\|Ax\|}{\|x\|}$$ называется \emph{нормой} оператора и обозначается $\|A\|$.

\begin{claim}
	\label{Op2} Для линейного оператора следующие условия равносильны:
	\begin{enumerate}
		\item $A$ ограничен;
		\item $A$ непрерывен;
		\item $A$ непрерывен в некоторой точке.
	\end{enumerate}
\end{claim}

\begin{proof}
	Пусть $A$ ограничен. Тогда $\|Ax-Ay\|=\|A(x-y)\|\leq\|A\|\|x-y\|$. Отсюда следует непрерывность $A$. Пусть оператор $A$ непрерывен в точке $x_0$. Из равенства $Ax=A(x-x_0)+A x_0$ следует непрерывность $A$ в нуле. Тогда для любого $\varepsilon>0$ существует $\delta$ такое, что для любого $x\in L$ с условием $\|x\|<\delta$ выполнено $\|Ax\|<\varepsilon$. Положим $\varepsilon=1$. Тогда $\|Ax\|<\frac{1}{\delta}$ при $\|x\|<1$. Следовательно, $A$ ограничен.
\end{proof}

\begin{lemma}
	\label{BerBan} Пусть $X$ --- полное метрическое пространство и $X=\bigcup\limits_{n=1}^{\infty} X_n$, где $X_n$ --- замкнутые множества. Тогда хотя бы для одного $X_n$ существует шар $B_r(a)$ радиуса $r$ с центром в $a$ такой, что $B_r(a)\subset X_n$.
\end{lemma}

\begin{proof}
	Предположим противное. Пусть $y_1\not\in X_1$. Тогда существует шар $B_1=B_{r_1}(y_1)$, радиуса $r_1$ такой, что $B_1\cap X_1=\emptyset$. Если $B_1\not\subset X_2$, то существует $y_2\in B_1$ такой, что $y_2\not\in X_2$. Тогда существует шар $B_2=B_{r_2}(y_2)$, радиуса $r_2<\frac{r_1}{2}$ такой, что $B_2\cap X_2=\emptyset$. Аналогично, существует $B_3=B_{r_3}(y_3)$ такой, что $B_3\cap X_3=\emptyset$, $r_3<\frac{r_2}{2}$, $y_3\in B_2$ и т.д. Мы получили последовательность вложенных шаров $B_1\supset B_2\supset\cdots\supset B_n\supset\cdots$ такую, что их радиусы стремятся к нулю, а их пересечение не лежит в $\bigcup\limits_{n=1}^{\infty} X_n$ (см. \ref{Pol2}). Противоречие.
\end{proof}

\begin{theorem}[теорема Банаха--Штейнгауза]
	\label{Ban-Sh}
	Пусть дано семейство $\{A_{\alpha}\}$ --- ограниченных линейных операторов на банаховом пространстве $L$, принимающие значения в нормированном пространстве $L'$. Предположим, что $\sup\limits_{\alpha}\|A_{\alpha}x\|<\infty$ для любого $x\in L$. Тогда $\sup\limits_{\alpha}\|A_{\alpha}\|<\infty$.
\end{theorem}

\begin{proof}
	Рассмотрим $$M_n=\{x\in L\mid \|A_{\alpha}x\|\leq n\text{ для всех $\alpha$}\}.$$ Заметим, что все $M_n$ замкнуты и покрывают все пространство. Тогда существует шар $B_r(a)$ радиуса $r$ с центром в $a$ такой, что $B_r(a)\subset M_n$. Поскольку $Ax=A(x-a)+Aa$ и $\sup\limits_{\alpha}\|A_{\alpha}a\|<\infty$, то получаем равномерную ограниченность операторов $\{A_{\alpha}\}$ на шаре $B_r(0)$. Следовательно, $\{A_{\alpha}\}$ равномерно ограниченно на единичном шаре.
\end{proof}

\begin{corollary}
	\label{Ban-Sh2}
	Пусть $L$ и $L'$ --- банаховы пространства, $A_n\colon L\rightarrow L'$ --- непрерывные операторы, причем для каждого $x$ существует $Ax=\lim\limits_{n\rightarrow\infty}A_n x$. Тогда $A$ --- непрерывный оператор.
\end{corollary}

\begin{proof}
	Очевидно, что $A$ --- линейное отображение. По теореме Банаха--Штейнгауза имеем $\sup\limits_n\|A_n\|\leq C<\infty$. Тогда $$\|Ax\|=\lim\limits_{n\rightarrow\infty}\|A_n x\|\leq C\|x\|,$$ т.е. $\|A\|\leq C$.
\end{proof}

\begin{definition}
	Оператор $A$ называется \emph{обратимым}, если для любого $y\in L'$ уравнение $Ax=y$ имеет единственное решение в $L$. Если $A$ обратим, то каждому $y\in L'$ можно поставить в соответствие $x\in L$, являющееся решением уравнения $Ax=y$. Оператор, осуществляющий это соответствие называется \emph{обратным} к $A$ и обозначается $A^{-1}$.
\end{definition}


\begin{theorem}
	\label{Op3}
	Оператор $A^{-1}$, обратный к линейному оператору $A$, также линеен.
\end{theorem}

\begin{proof}
	Пусть $y_1,y_2\in L'$. Положим $Ax_1=y_1$ и $Ax_2=y_2$, т.е. $A^{-1}y_1=x_1$, $A^{-1}y_2=x_2$. Тогда $A(x_1+x_2)=Ax_1+Ax_2=y_1+y_2$. Отсюда, $$A^{-1}(y_1+y_2)=x_1+x_2=A^{-1}y_1+A^{-1}y_2.$$ Аналогично, $A(\alpha x_1)=\alpha(Ax_1)=\alpha y_1$. Отсюда, $A^{-1}(\alpha y_1)=\alpha x_1=\alpha A^{-1}y_1$.
\end{proof}

\begin{lemma}
	\label{Op4} Пусть $M$ --- всюду плотное множество в банаховом пространстве $L$. Тогда любой ненулевой элемент $y\in L$ можно разложить в ряд $$y=y_1+y_2+\cdots+y_n+\cdots,$$ где $y_n\in M$ и $\|y_n\|\leq\frac{3\|y\|}{2^n}$.
\end{lemma}

\begin{proof}
	Выберем $y_1\in M$ так, что $\|y-y_1\|<\frac{\|y\|}{2}$. Выберем $y_2\in M$ так, что $\|y-y_1-y_2\|<\frac{\|y\|}{4}$, $y_3\in M$ так, что $$\|y-y_1-y_2-y_3\|<\frac{\|y\|}{8}$$ и т.д. Таким образом, $$\|y-y_1-y_2-\cdots-y_n\|\leq\frac{\|y\|}{2^n}.$$ Очевидно, что $$\left\|y-\sum\limits_{i=1}^n y_i\right\|\rightarrow 0,\text{ при } n\rightarrow\infty.$$ С другой стороны, $$\|y_1\|=\|y_1-y+y\|\leq\|y_1-y\|+\|y\|\leq\frac{3\|y\|}{2},$$
	$$\|y_2\|=\|y_2+y_1-y+y-y_1\|\leq\|y_2+y_1-y\|+\|y-y_1\|\leq\frac{3\|y\|}{4},$$ и т.д. Наконец, $$\|y_n\|=\|y_n+\cdots+y_2+y_1-y+y-y_1-y_2-\cdots-y_{n-1}\|\leq$$ $$\leq\|y_n+\cdots+y_2+y_1-y\|+\|y-y_1-y_2-\cdots-y_{n-1}\|\leq\frac{3\|y\|}{2^n}.$$
\end{proof}

\begin{theorem}
	\label{Op5}
	Пусть $A$  --- линейный ограниченный оператор, взаимно однозначно отображающий банахово пространство $L$ на банахово пространство $L'$. Тогда оператор $A^{-1}$, обратный к линейному оператору $A$, также ограничен.
\end{theorem}

\begin{proof}
	Положим $$M_n=\{y\in L'\mid \|A^{-1}y\|\leq n\|y\|\}.$$ Тогда $L'=\bigcup M_n$. По теореме Бэра \ref{Pol3} хотя бы одно множество $M_n$ плотно в некотором шаре $B_0$. Внутри шара $B_0$ рассмотрим шаровой слой $$P=\{z\mid\beta<\|z-y_0\|<\alpha\},$$ $0<\beta<\alpha$, $y_0\in M_n$. Пусть $P_0=\{z\mid\beta<\|z\|<\alpha\}.$ Покажем, что в $P-0$ плотно некоторое множество $M_N$. Пусть $z\in P\cap M_n$. Тогда $z-y_0\in P_0$. Заметим, что $$\|A^{-1}(z-y_0)\|\leq \|A^{-1}z\|+\|A^{-1}y_0\|\leq n(\|z\|+\|y_0\|)=$$ $$=n(\|z-y_0+y_0\|+\|y_0\|)\leq n(\|z-y_0\|+2\|y_0\|)=$$ $$=n\|z-y_0\|\left(1+\frac{2\|y_0\|}{\|z-y_0\|}\right)\leq n\|z-y_0\|\left(1+\frac{2\|y_0\|}{\beta}\right).$$ Возьмем целое число $N$, большее $n\left(1+\frac{2\|y_0\|}{\beta}\right).$ Тогда $z-y_0\in M_N$. Поскольку $M_n$ плотно в $P$, то $M_N$ плотно в $P_0$. Пусть $y\in L'$. Заметим, что существует $\lambda$ такое, что $\beta<\|\lambda y\|<\alpha$, т.е. $\lambda y\in P_0$. Так как $M_N$ плотно в $P_0$, то существует последовательность $y_n\in M_N$, сходящаяся к $\lambda y$. Тогда последовательность $\frac{1}{\lambda}y_n$ сходится к $y$. Заметим, что если $y_n\in M_N$, то и $\frac{1}{\lambda}y_n\in M_N$ при $\lambda\neq 0$. Таким образом, $M_N$ плотно в $L'$.
	
	Пусть $y\in L'$. Согласно лемме \ref{Op4} существует ряд из элементов $y_n\in M_N$ сходящийся к $y$ т.е. $$y=y_1+y_2+\cdots+y_n+\cdots.$$ Более того, $\|y_n\|<\frac{3\|y\|}{2^n}$. Пусть $x_1,x_2,\ldots,x_n,\ldots$ --- прообразы $y_1,y_2,\ldots,y_n,\ldots$. Тогда $$\|x_n\|=\|A^{-1}y_n\|\leq N\|y_n\|<\frac{3N\|y\|}{2^n}.$$ Следовательно, ряд $\sum x_n$ сходится. Более того $$\|x\|\leq\sum\limits_{n=1}^{\infty}\|x_n\|\leq 3N\|y\|\sum\limits_{n=1}^{\infty}\frac{1}{2^n}=3N\|y\|.$$ Отсюда, $$Ax=Ax_1+Ax_2+\cdots+Ax_n+\cdots=y_1+y_2+\cdots+y_n+\cdots=y.$$ Тогда $x=A^{-1}y$ и $$\|A^{-1}y\|=\|x\|\leq 3N\|y\|.$$ Следовательно, $A^{-1}$ ограничен.
\end{proof}

\begin{definition}
	Пусть $A$  --- линейный ограниченный оператор, отображающий пространство $L$ в пространство $L'$. Пусть $g$ --- линейный непрерывный функционал на $L'$, т.е. $g\in L'^*$. Тогда мы можем определить функционал $f$ на $L$, как $f(x)=g(Ax)$. Не сложно проверить, что $f$ --- линейный непрерывный функционал. Таким образом, $f\in L^*$. Каждому функционалу $g\in L'^*$ мы поставили в соответствие функционал $f\in L^*$. Таким образом, мы получили отображение $A^*\colon L'^*\rightarrow L^*$. Это отображение линейно, т.е. $A^*$ --- линейный оператор. Он называется \emph{сопряженным} оператором.
\end{definition}

Очевидно, что $(A+B)^*=A^*+B^*$, $(\alpha A)^*=\alpha A^*$.

\begin{theorem}
	\label{Op6}
	Пусть $A$  --- линейный ограниченный оператор, отображающий банахово пространство $L$ на банахово пространство $L'$. Тогда оператор $A^*$ также ограничен и $$\|A^*\|=\|A\|.$$
\end{theorem}

\begin{proof}
	Пусть $x\in L$, $g\in L'^*$. Заметим, что $$|A^*g(x)|=|g(Ax)|\leq\|g\|\|A\|\|x\|.$$ Отсюда, $\|A^*g\|\leq\|A\|\|g\|$. Таким образом, $\|A^*\|\leq\|A\|.$ Пусть $y_0=\frac{Ax}{\|Ax\|}$. Тогда $\|y_0\|=1$. По теореме Хана--Банаха существует функционал $g$, что $\|g\|=1$ и $g(y_0)=1$, т.е. $g(Ax)=\|Ax\|$. Тогда $$\|Ax\|=g(Ax)=|A^*g(x)|\leq\|A^*g\|\|x\|\leq \|A^*\| \|g\|\|x\|=\|A^*\|\|x\|.$$ Отсюда, $\|A\|\leq\|A^*\|$.
\end{proof}

Пусть $A$ --- линейный ограниченный оператор, действующий в гильбертовом пространстве $H$. Согласно теореме Рисса \ref{RisHil} любой функционал имеет вид $f(x)=(x,y)$. Тогда сопряженный оператор $A^*$ можно задать равенством $(Ax,y)=(x,A^*y)$.

\begin{definition}
	Линейный ограниченный оператор $A$ называется \emph{самосопряженным}, если $(Ax,y)=(x,Ay)$ для любых $x,y\in H$.
\end{definition}

\section{Спектр линейного оператора}

Пусть $A$ --- линейный оператор в конечномерном пространстве. Число $\lambda$ называется \emph{собственным значением} оператора $A$, если уравнение $Ax=\lambda x$ имеет ненулевое решение. Совокупность всех собственных значений называется \emph{спектром} оператора $A$, а все остальные значения $\lambda$ --- \emph{регулярными}. Заметим, что в конечномерном пространстве существуют лишь два случая:\\
1) уравнение $Ax=\lambda x$ имеет ненулевое решение; оператор $(A-\lambda I)^{-1}$ при этом не существует (здесь $I$ --- тождественный оператор);\\
2) существует ограниченный оператор $(A-\lambda I)^{-1}$.

В бесконечномерном пространстве существует еще один случай:\\
3) оператор $(A-\lambda I)^{-1}$ существует, но не ограничен.

Рассмотрим оператор $A$ на бесконечномерном пространстве $L$. Число $\lambda$ называется \emph{регулярным} для оператора $A$, если оператор $(A-\lambda I)^{-1}$ определен на всем $L$ и непрерывен. Оператор $R_{\lambda}(A)=(A-\lambda I)^{-1}$ называется \emph{резольвентой}. Совокупность всех $\lambda$, которые не являются регулярными, называется \emph{спектром} оператора $A$. Очевидно, что спектру принадлежат все собственные значения оператора $A$. Их совокупность называется \emph{точечным спектром}. Остальная часть спектра, т.е. совокупность всех $\lambda$, для которых $(A-\lambda I)^{-1}$ существует, но не непрерывен, называется \emph{непрерывным спектром}. Будем обозначать множество регулярных чисел через $\varrho(A)$, а спектр $\sigma(A)$.

\begin{theorem}
	\label{Sp1}
	Пусть $A$  --- линейный ограниченный оператор, отображающий банахово пространство $L$ на банахово пространство $L'$, обладает ограниченным обратным оператором $A^{-1}$. Пусть $D$  --- линейный ограниченный оператор, отображающий банахово пространство $L$ на банахово пространство $L'$, такой что $\|D\|<\frac{1}{\|A^{-1}\|}$. Тогда оператор $A+D$ также ограничен и обладает ограниченным обратным.
\end{theorem}

\begin{proof}
	Согласно теореме \ref{Op5} нужно показать, что уравнение $Ax+Dx=y$ однозначно разрешимо для любого $y$. Это уравнение равносильно уравнению $$A^{-1}(Ax+Dx)=A^{-1}y.$$ Отсюда, $A^{-1}y-A^{-1}Dx=x.$ Заметим, что отображение $F(x)=A^{-1}y-A^{-1}Dx$ сжимающее. Действительно, $$\|F(x)-F(z)\|=\|A^{-1}y-A^{-1}Dx-A^{-1}y+A^{-1}Dz\|=$$ $$=\|A^{-1}D(z-x)\|\leq\|A^{-1}\|\|D\|\|z-x\|.$$ Поскольку $\|A^{-1}\|\|D\|<1$, то отображение $F(x)$ сжимающие. Тогда, согласно теореме \ref{Sj2}, уравнение $Ax+Dx=y$ имеет ровно одно решение.
\end{proof}

\begin{corollary}
	\label{Sp2}
	Пусть $\lambda\in\varrho(A)$. Тогда при достаточно малом $\delta$, число $\lambda+\delta\in\varrho(A)$.
\end{corollary}

Таким образом, регулярные точки образуют открытое множество. Следовательно, спектр --- замкнутое множество.

\begin{theorem}
	\label{Sp3}
	Пусть $A$  --- линейный ограниченный оператор, отображающий банахово пространство $L$ на себя. Тогда при $|\lambda|>\|A\|$, имеем $\lambda\in\varrho(A)$. Более того, $$R_{\lambda}(A)=-\sum\limits_{k=0}^{\infty}\frac{A^k}{\lambda^{k+1}}.$$
\end{theorem}

\begin{proof}
	Поскольку $\|-\lambda I\|=|\lambda|$, то $\|(-\lambda I)^{-1}\|=\frac{1}{|\lambda|}$. Тогда, согласно теореме \ref{Sp1}, оператор $-\lambda I+A$ обладает ограниченным обратным. Поскольку $$A-\lambda I=-\lambda(I-\frac{A}{\lambda}),$$ то достаточно доказать, что $$(I-A)^{-1}=\sum\limits_{k=0}^{\infty} A^k,$$ если $\|A\|<1$. Заметим, что $$\sum\limits_{k=0}^{\infty} \|A^k\|=\sum\limits_{k=0}^{\infty}\|A\|^k<\infty.$$ Поскольку $L$ полно, то сумма $\sum\limits_{k=0}^{\infty} A^k$ является ограниченным линейным оператором. Для любого $n$ получаем $$(I-A)\left(\sum\limits_{k=0}^{n} A^k\right)=\sum\limits_{k=0}^{n} A^k(I-A)=\sum\limits_{k=0}^{n} (A^k-A^{k+1})=I-A^{n+1}.$$ Переходя к пределу при $n\rightarrow\infty$, получаем $$(I-A)\left(\sum\limits_{k=0}^{\infty} A^k\right)=I.$$
\end{proof}


\begin{corollary}
	\label{Sp4}
	Пусть $\lambda_0\in\varrho(A)$. Тогда при $|\lambda-\lambda_0|<\|R_{\lambda_0}(A)\|^{-1}$, имеем $\lambda\in\varrho(A)$. Более того, $$R_{\lambda}(A)=\sum\limits_{k=0}^{\infty}(\lambda-\lambda_0)^k(R_{\lambda_0}(A))^{k+1}.$$
\end{corollary}

\begin{proof}
	Сходимоть ряда $\sum\limits_{k=0}^{\infty}(\lambda-\lambda_0)^k(R_{\lambda_0}(A))^{k+1}$ доказывается точно так же, как в теореме \ref{Sp3}. Для его суммы имеем, $$(A-\lambda I)\sum\limits_{k=0}^{\infty}(\lambda-\lambda_0)^k(R_{\lambda_0}(A))^{k+1}=$$ $$=\sum\limits_{k=0}^{\infty}(\lambda-\lambda_0)^k(R_{\lambda_0}(A))^{k+1}(A-\lambda_0I-(\lambda-\lambda_0)I)=$$ $$=\sum\limits_{k=0}^{\infty}\left((\lambda-\lambda_0)^k(R_{\lambda_0}(A))^{k}-(\lambda-\lambda_0)^{k+1}(R_{\lambda_0}(A))^{k+1}\right)=I.$$
\end{proof}

\begin{theorem}
	\label{Sp5}
	Спекр любого оператора $A$ в комплексном банаховом пространстве $L$ является непустым компактом в круге радиуса $\|A\|$ с центром в нуле комплексной плоскости.
\end{theorem}

\begin{proof}
	Мы уже доказали, что $\sigma(A)$ замкнуто и $$\sigma(A)\subset\{z\in\CC:|z|\leq\|A\|\}.$$ Предположим, что $R_{\lambda}(A)$ существует для всех $\lambda$. Пусть $\varphi$ --- линейный непрерывный функционал на пространстве операторов. Положим $f(\lambda)=\varphi(R_{\lambda}(A))$. Согласно \ref{Sp4} $f(\lambda)$ --- аналитическая функция на всем $\CC$. Более того, согласно теореме \ref{Sp3} $|f(\lambda)|\rightarrow 0$ при $\lambda\rightarrow\infty$. По теореме Лиувилля $f(\lambda)\equiv 0$. Отсюда, $R_{\lambda}(A)=0$ при всех $\lambda$.
\end{proof}

\begin{definition}
	\emph{Спектральный радиус} оператора $A$ зададим формулой $$r(A)=\inf\|A^n\|^{\frac{1}{n}}.$$
\end{definition}

\begin{theorem}
	\label{Sp6}
	Имеет место равенство
	$$r(A)=\lim\limits_{n\rightarrow\infty}\|A^n\|^{\frac{1}{n}}.$$ Более того, $$r(A)=\max\{|z|:z\in\sigma(A)\}.$$
\end{theorem}

\begin{proof}
	Пусть $\varepsilon>0$. Выберем такое $p\in\NN$, что $\|A^p\|^{\frac{1}{p}}\leq r(A)+\varepsilon$. Пусть $n=kp+m$, где $0\leq m\leq p-1$. Тогда $$\|A^n\|\leq\|A^p\|^k\|A^m\|\leq M\|A^p\|^k,$$ где $M=\max\limits_{m=1,\ldots,p-1}\|A^m\|$. Отсюда, $$r(A)\leq\|A^n\|^{\frac{1}{n}}\leq M^{\frac{1}{n}}\|A^p\|^{\frac{k}{n}}\leq M^{\frac{1}{n}}(r(A)+\varepsilon)^{\frac{kp}{n}}.$$ Поскольку $M^{\frac{1}{n}}\rightarrow 1$, $\frac{kp}{n}\rightarrow 1$ при $n\rightarrow\infty$, то $$r(A)\leq\lim\limits_{n\rightarrow\infty}\|A^n\|^{\frac{1}{n}}\leq r(A)+\varepsilon.$$ В силу произвольности $\varepsilon$, получаем $$r(A)=\lim\limits_{n\rightarrow\infty}\|A^n\|^{\frac{1}{n}}.$$ Покажем, что при $\lambda>r(A)$ оператор $A-\lambda I$ обратим. Поскольку $A-\lambda I=\lambda(\frac{A}{\lambda}-I)$, то мы можем считать, что $\lambda=1$, а $r(A)<1$. Пусть $\varepsilon$ такое, что $r(A)+\varepsilon<1$. Заметим, что существует $N\in\NN$ такое, что для любого $n>N$ выполнено $\|A^n\|\leq (r(A)+\varepsilon)^n$. Тогда ряд $\sum\limits_{k=0}^{\infty} A^k$ сходится. Аналогично, как при доказательстве теоремы \ref{Sp3}, получаем, что $(A-I)^{-1}=\sum\limits_{k=0}^{\infty} A^k$.
	
	Докажем, что существует точка на окружности радиуса $r(A)$ есть точка спектра. Предположим противное. Поскольку $\sigma(A)$ --- замкнутое множество, то существует $r<r(A)$ такое, что для всех $\lambda$ с условием $|\lambda|>r$ существует $R_{\lambda}(A)$. Пусть $\varphi$ --- линейный непрерывный функционал на пространстве операторов. Тогда функция $f(\lambda)=\varphi(R_{\lambda}(A))$ голоморфна при $|\lambda|>r$. Мы знаем, что вне круга радиуса $\|A\|$ эта функция задается рядом Лорана (см. \ref{Sp3}) $$f(\lambda)=-\sum\limits_{k=0}^{\infty}\frac{\varphi(A^k)}{\lambda^{k+1}}.$$ В силу единственности разложения этот ряд задает $f(\lambda)$ при $|\lambda|>r$. Пусть $r<\lambda<r(A)$. Тогда, согласно теореме Банаха--Штейнгауза, существует $C$ такое, что для любого $k$ имеем $\|\lambda^{-k-1}A^k\|\leq C$. Следовательно, $\|A^k\|^{\frac{1}{k}}\leq C^{\frac{1}{k}}\lambda^{1+\frac{1}{k}}$. Отсюда, $r(A)\leq\lambda$. Противоречие.
\end{proof}

\section{Компактные операторы}

\begin{definition}
	Пусть $X$ и $Y$ --- банаховы пространства. Линейный оператор $A\colon X\rightarrow Y$ называется \emph{компактным}, если он каждое ограниченное множество переводит в относительно компактное.
\end{definition}

\begin{claim}
	\label{Kom1}
	Линейный оператор $A\colon X\rightarrow Y$ компактен тогда и только тогда, когда он переводит единичный шар в $X$ в относительно компактное множество в $Y$.
\end{claim}

\begin{theorem}
	\label{Kom2}
	Пусть $A$ и $B$ --- компактные операторы. Тогда $A+B$ и $\alpha A$ --- также компактные операторы.
\end{theorem}

\begin{proof}
	Пусть $U$ --- ограниченное множество. Тогда $(A+B)(U)\subset A(U)+B(U)$ и $(\alpha A)(U)=\alpha(A(U))$. Поскольку $A(U)$ и $B(U)$ относительно компактны, то $A(U)+B(U)$ также относительно компактно. Действительно, пусть $\{x_n+y_n\}$ --- последовательность в $A(U)+B(U)$. Выберем сходящуюся подпоследовательность $\{x_{i_n}\}$ в $\{x_n\}$, затем выберем сходящуюся подпоследовательность $\{y_{j_n}\}$ в $\{y_{i_n}\}$. Тогда последовательность $\{x_{j_n}+y_{j_n}\}$ сходится. Если последовательность $\{x_n\}\subset A(U)$ сходится, то сходится и последовательность $\{\alpha x_n\}\subset \alpha A(U)$. Следовательно, $A+B$ и $\alpha A$ --- компактные операторы.
\end{proof}

\begin{theorem}
	\label{Kom3}
	Пусть $\{A_n\}$ --- последовательность компактных операторов, сходящихся по норме к оператору $A$. Тогда $A$ --- компактный оператор.
\end{theorem}

\begin{proof}
	Нам достаточно доказать, что для любой ограниченной последовательности $\{x_n\}$ ($\|x_n\|<C$), из последовательности $\{Ax_n\}$ можно выбрать сходящуюся подпоследовательность. Поскольку $A_1$ компактен, то можно выбрать подпоследовательность $x_1^{(1)},x_2^{(1)},\ldots,x_n^{(1)},\ldots$ такую, что последовательность $A_1x_n^{(1)}$ будет сходиться. Из этой последовательности можно выбрать подпоследовательнось $x_1^{(2)},x_2^{(2)},\ldots,x_n^{(2)},\ldots$ такую, что последовательность $A_2x_n^{(2)}$ будет сходиться и т.д. Возьмем последовательность $x_1^{(1)},x_2^{(2)},\ldots,x_n^{(n)},\ldots$. Докажем, что последовательность $Ax_n^{(n)}$ сходится. Для этого достаточно доказать, что она фундаментальная. Действительно, $$\|Ax_n^{(n)}-Ax_m^{(m)}\|=\|Ax_n^{(n)}-A_kx_n^{(n)}+A_kx_n^{(n)}-A_kx_m^{(m)}+A_kx_m^{(m)}-Ax_m^{(m)}\|\leq$$ $$\leq\|Ax_n^{(n)}-A_kx_n^{(n)}\|+\|A_kx_n^{(n)}-A_kx_m^{(m)}\|+A_kx_m^{(m)}-Ax_m^{(m)}\|.$$ Выберем $k$ так, что $\|A-A_k\|<\frac{\varepsilon}{C}$. Тогда $$\|Ax_n^{(n)}-A_kx_n^{(n)}\|=\|(A-A_k)x_n^{(n)}\|\leq\|A-A_k\|\|x_n\|<\frac{\varepsilon}{C}C=\varepsilon.$$ Аналогично, $\|Ax_n^{(n)}-A_kx_n^{(n)}\|<\varepsilon$. Теперь выберем $N$ так, что для всех $n>N$ и $m>N$ было выполнено $\|A_kx_n^{(n)}-A_kx_m^{(m)}\|<\varepsilon$. Тогда $\|Ax_n^{(n)}-Ax_m^{(m)}\|<3\varepsilon$. Следовательно, последовательность $Ax_n^{(n)}$ сходится.
\end{proof}

\begin{corollary}
	\label{Kom4}
	Множество компактных операторов образует замкнутое линейное подпространство.
\end{corollary}

\begin{proof}
	Следует из теорем \ref{Kom2} и \ref{Kom3}.
\end{proof}

\begin{theorem}
	\label{Kom5}
	Пусть $A\colon X\rightarrow Y$ и $B\colon Y\rightarrow Z$ --- линейные операторы на баноховых пространствах $X,Y,Z$. Тогда $BA$ --- компактный оператор, если один из операторов компактен, а второй ограничен.
\end{theorem}

\begin{proof}
	Пусть $U\subset X$ --- ограниченное множество. Предположим $A$ ограничен, $B$ компактен. Тогда $A(U)$ также ограничено. Следовательно, $B(A(U))$ относительно компактно, т.е. $BA$ --- компактный оператор. Предположим $A$ компактен, $B$ ограничен. Тогда $A(U)$ относительно компактно. Поскольку $B$ непрерывный оператор (т.е. $B$ переводит фундаментальную последовательность в фундаментальную последовательность), то $B(A(U))$ также относительно компактно. Следовательно, $BA$ --- компактный оператор.
\end{proof}

\begin{theorem}
	\label{Kom6}
	Пусть $A\colon X\rightarrow Y$ --- линейный оператор на банаховых пространствах $X,Y$. Тогда $A$ компактен тогда и только тогда, когда $A^*\colon Y^*\rightarrow X^*$ компактен.
\end{theorem}

\begin{proof}
	Предположим, что $A\colon X\rightarrow Y$ --- компактный оператор. Пусть $V,V^*$ --- единичные шары в $X$ и в $Y^*$ соответственно. Пусть $f\in V^*$. Тогда $A^*f(x)=f(Ax)$. Заметим, что $A^*(V^*)$ равномерно ограниченно и равномерно непрерывно. Поскольку $A(V)$ --- относительно компактно, то по теореме Арцела $A^*(V^*)$ относительно компактно.
	
	Предположим, что $A^*$ --- компактный оператор. Тогда $A^{**}$ --- компактный оператор. Мы знаем, что существуют изометричные вложения $J_1\colon X\rightarrow X^{**}$, $J_2\colon Y\rightarrow Y^{**}$. Более того, $$A^{**}(J_1x)(f)=J_1x(A^*(f))=(A^*(f))(J_1x)=f(Ax)=$$ $$=f(J_2Ax)=(J_2Ax)(f).$$ Отсюда следует, что $A$ --- компактный оператор.
\end{proof}



\begin{thebibliography}{lll}
	\bibitem{K}
	Келли Дж.Л. \emph{Общая топология}.
	\bibitem{KF}
	А.Н. Колмогоров, С.В. Фомин, \emph{Элементы теории функции и функционального анализа}.
	\bibitem{MF}
	Мищенко А.С., Фоменко А.Т. \emph{Курс дифференциальной геометрии и топологии}.
	\bibitem{En}
	Энгелькинг Р. \emph{Общая топология}.
	
\end{thebibliography}


\end{document}
\bibitem{KSB}
